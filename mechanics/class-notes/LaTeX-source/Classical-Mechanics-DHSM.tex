%%%%%%%%%%%%%%%%%%%% book.tex %%%%%%%%%%%%%%%%%%%%%%%%%%%%%
%
% sample root file for the chapters of your "monograph"
%
% Use this file as a template for your own input.
%
%%%%%%%%%%%%%%%% Springer-Verlag %%%%%%%%%%%%%%%%%%%%%%%%%%%%%%


% RECOMMENDED %%%%%%%%%%%%%%%%%%%%%%%%%%%%%%%%%%%%%%%%%%%%%%%%%%%
\documentclass[graybox,envcountchap,sectrefs]{svmonoMuga}

% choose options for [] as required from the list
% in the Reference Guide

\usepackage{mathptmx}
\usepackage{helvet}
\usepackage{courier}
\usepackage{type1cm}

\usepackage{makeidx}         % allows index generation
\usepackage{graphicx}        % standard LaTeX graphics tool
                             % when including figure files
\usepackage{multicol}        % used for the two-column index
\usepackage[bottom]{footmisc}% places footnotes at page bottom
\usepackage{subfigure}
\usepackage{amssymb,amsmath}
 \usepackage{enumerate}
\usepackage{ntheorem}
%\usepackage{movie15}

 %   \usepackage{createspace}
%    \usepackage[size=pocket,noicc]{createspace}

%\usepackage{graphicx}
\usepackage[utf8]{inputenc}
%\usepackage{amsmath}
%\usepackage{mathptmx}
%\usepackage{hyperref}
\usepackage{amsfonts}
%\usepackage{amssymb}
\usepackage[colorlinks={true},linkcolor={blue},citecolor={red}]{hyperref}

%\graphicspath{{Images/}}
\graphicspath{{/Users/Mugalan/Documents/Research/Papers/Physics/Images/}}

\newtheorem{assumption}{Assumption}[chapter]
\newtheorem{axiom}{Axiom}[chapter]
\newtheorem{solutiontoexercise}{Solution to Exercise}[chapter]
\newtheorem{exerciseprofs}{Exercise in Prof. Sivasegaram's tutorial on steady flow:\hspace{0.2cm}Q}
\newtheorem{exerciseprofu}{Exercise in Prof. Sivasegaram's tutorial on un-steady flow:\hspace{0.2cm}Q}
\newtheorem{solnexerciseprofs}{Solution to exercise in Prof. Sivasegaram's tutorial on steady flow:\hspace{0.2cm}Q}
\newtheorem{solnexerciseprofu}{Solution to exercise in Prof. Sivasegaram's tutorial on un-steady flow:\hspace{0.2cm}Q}


\usepackage{fancyhdr}
\pagestyle{fancy}
\fancyhead{} % clear all header fields
\renewcommand{\headrulewidth}{0.5pt} % no line in header area
%\fancyfoot{} % clear all footer fields
\fancyhead[RE,LO]{\small Lecture notes by D. H. S. Maithripala, Dept. of Mechanical Engineering, University of Peradeniya} % page number in "outer" position of footer line
%\fancyfoot[LE,RO]{Page \thepage} % other info in "inner" position of footer line


\allowdisplaybreaks
% see the list of further useful packages
% in the Reference Guide


\makeindex             % used for the subject index
                       % please use the style svind.ist with
                       % your makeindex program

%%%%%%%%%%%%%%%%%%%%%%%%%%%%%%%%%%%%%%%%%%%%%%%%%%%%%%%%%%%%%%%%%%%%%

\begin{document}

\author{{Maithripala D. H. S., \\
Dept. of Mechanical Engineering, \\
Faculty of Engineering, \\
University of Peradeniya, \\
Sri Lanka.}
}
\title{Lecture Notes on Classical Mechanics}
\subtitle{Class notes for ME211, ME320, ME518\\
\mbox{}\\
\small{Supplementary .ipynb interactive Notebook:\\
\url{https://github.com/mugalan/lessons/blob/main/mechanics/class-notes/Mugas_Classical_Mechanics.ipynb}}\\}
\maketitle
%\frontmatter%%%%%%%%%%%%%%%%%%%%%%%%%%%%%%%%%%%%%%%%%%%%%%%%%%%%%%%%%%%%%%%
\preface
This is a compilation of notes that originated as class notes for ME2204 Engineering Mechanics at the department 
of Mechanical and Manufacturing Engineering, Faculty of Engineering, University of Ruhuna during the period of 2006 to 2008.  I am currently  using it as supplementary notes for ME320 and ME518 at the 
University of Peradeniya. They are far from complete and I will be frequently 
updating them as time permits.
I am deeply indebted to all the students who have suffered through them and have tried out the exercises and have provided me with valuable input. I am sure there are many errata 
and will greatly appreciate if you can please bring them to my notice by sending an e-mail to mugalan at gmail.com. %

\vspace{\baselineskip}
\begin{flushright}\noindent
Peradeniya, Sri Lanka,\hfill {\it D. H. S. Maithripala}\\
\today \\
\end{flushright}




\tableofcontents








\chapter{Galilean Mechanics}


%\chapter{Introduction}
\begin{figure}
\centering
\includegraphics[width=6cm]{Justus_Sustermans_-_Portrait_of_Galileo_Galilei_1636}\label{Fig:Galileo}
\caption{Galileo Galilei --- 15 February 1564 to 8 January 1642: \href{https://en.wikipedia.org/wiki/Galileo_Galilei}{Wikipedia} }
\end{figure}

%\section{Introduction}
Mechanics deals with the scientific description of the world as we perceive it. The human infatuation with the subject pre-dates written history and has given rise to the well accepted customary approach of searching for  scientific laws, in the form of mathematical expressions, to describe and generalize naturally observed phenomena. The sole test of the ``truth'' of such laws was (and 
still is) experiment. Typically one would imagine, conceptualize and arrive at mathematical expressions to describe and generalize observed phenomena and then devise 
experiments to verify their validity. They are considered to be true as long as there are no experiments that contradict them.
Such laws may fail to be ``true'' by virtue of such experiments being inaccurate. Accuracy depends on precision. Up until the 20th century, Galilean mechanics (or what is popularly 
known as Newtonian mechnics\footnote{Galileo Galilei's contribution to the formulation of the laws as stated by Newton is monumental.}) and Maxwell's electromagnetism were 
found to accurately describe the phenomena of nature observed to within the precision allowed by the instruments of that time. As technology developed it became possible, around 
the start of the 20th century, to conduct experiments and observe nature with increasingly high precision. Scientists started observing phenomena that were not quite accurately 
described by Newton's laws and Maxwell's laws. Inaccuracies were observed, when Galilean principles were used to describe the motion of atomistic and sub-atomistic particles,  such as electrons, moving close to the speed of light. The attempt to accurately describe, the motion of objects that move close to the speed of light gave birth to 
Einstein's \emph{relativistic mechanics}, while the attempt to accurately describe the motion of microscopic particles gave birth to \emph{quantum mechanics}. To this date no experiments nor 
observations have contradicted the validity of these two scientific principles. Einstein extended his theory of relativity to incorporate gravitational interaction between objects and termed it the theory of \emph{general relativity}. The theory of special relativity, Maxwell's electro-magnetism, and quantum mechanics have been 
combined into one single framework called \emph{quantum field theory} while the unification of Einstein's general relativity and quantum mechanics remains an open challenge that 
prevents the formulation of one single principle (a theory of everything) that could explain all observed phenomena.

The objects of motion that we encounter in most of our Engineering practice are much larger than microscopic objects and move at speeds much slower than the 
speed of light. The motion of such objects are sufficiently accurately described by Galilean mechanics. Galilean mechanics rests on the 
belief that all \textit{objects are made of 
impenetrable interacting particles of matter} and that only one thing can be at a given place at a given time instant \cite{MM}. This notion requires the acceptance of the fundamental concept that all observers agree that \textit{time} is; independent of space, one-dimensional, continuous, isotropic, and homogeneous and that
\textit{space} is; three-dimensional, continuous, isotropic, and homogeneous. 
The notion of time is related to memory. Memory allows one to prescribe a sequence of 
events. The duration between events is measured by time. Since events are related to motion, so is time, and hence motion is essential to the definition of time. We measure time by comparing motions. In classical mechanics the 
best way to look at space and time is as tools that allow one to describe the relation between objects and hence motion. 

The aim of mechanics is to describe the relation between objects using the language of mathematics. Transforming the observations into a mathematical expression involving numbers requires measurements. Measurements depend on the measuring system or in other words the observer. Since motion is believed to be universal the laws that govern nature should also be observer independent. \emph{Thus mechanics can be considered to be the search and study of the scientific laws of motion in the form of mathematical expressions that are observer independent}. This means that half of the problem lies in figuring out the capabilities each observer has and the measurements and observations that different observers can agree upon. Properties and concepts that every observer can agree upon are called \textit{observer invariant} or simply \textit{intrinsic}. 
Below we will explore these notions that will eventually help us mathematically describe the motion of everyday observed objects. This study is what is called Galilean mechanics.


The application of the Galilean laws of mechanics is generally divided
into three branches depending on the type of objects under consideration; \textit{rigid-body mechanics, deformable-body mechanics} and \textit{fluid mechanics}. In these notes we 
will concentrate on learning the basics of rigid-body mechanics.
The study of rigid-body mechanics begins with describing the motion of a single \textit{particle of matter}. A general rigid body is considered to consist of a large number of such particles where the distance between each of the particles 
remain fixed. The geometric description of the motion of these objects is what is generally known as \textit{Kinematics} while the study of the cause of motion is referred to as 
\textit{Kinetics}. We will explore these ideas a bit more closely in the following sections.





\section{Fundamental Laws of Galilean Mechanics}\label{Secn:GalileanMechanics}

Describing the motion of a particle depends on the measurements that an observer makes (quantitative observations). We will say that a particular measuring system represents an observer. In general, different measurement systems will yield different descriptions of the motion. For example a particle that appears to be fixed with respect to one observer will appear to be moving for another observer that is moving with respect to the first one. Thus in general the description of position of a given particle depends on the observer. A natural question that arises is the following: do there exists notions and properties of nature that do not depend on the observer? If so what are they? 

\subsection{Inertial Observers}
Our general experience is that motion appears to be continuous, direction independent, and the same every where on earth. We will take this as to be true. 
That is we assume that
\begin{svgraybox}
\begin{axiom}{\sf Fundamental Assumptions of Galilean Space-Time:}\label{ax:GalileanSpaceTime} 
There exists a class of observers called \textit{inertial observers} who agree that
\begin{enumerate}[(i)]
\item time is independent of space,
\item time is one dimensional, continuous, isotropic, homogeneous, and the difference in time between any two particular `instants' is the same,
\item space is three dimensional, continuous, isotropic, homogeneous, and the distance between any two particular `points' in space is the same.
\end{enumerate}
\end{axiom}
\end{svgraybox}
Galilean mechanics rests upon these assumptions and hold true as long as they remain valid.
These fundamental assumptions about classical space-time are in fact assumptions about the measuring systems the observers have. First of these assumptions imply that there exists inertial observer independent units for the measurement of time or in other words that a \emph{universal clock} exists. The second of the above assumptions implies that the notion of distance is inertial observer independent and hence that straight lines, parallel lines, and perpendicular lines in space can be defined in an inertial observer independent manner. A space where these notions hold is called an \textit{Euclidean space}\footnote{A space where Euclidean geometry holds.}. 
\begin{svgraybox}
Thus the above Axiom supposes that there exists observers, referred to as inertial observers, who see that time is universal and that the space we live in is Euclidean.
\end{svgraybox} 


\begin{figure}[ht]
\begin{center}
\includegraphics[width=1.5in]{eFramePoint}
\renewcommand{\baselinestretch}{1}\selectfont
\caption{The representation of point in space using an orthonormal frame $\mathbf{e}$.}
\label{Fig:PointDefn}
\renewcommand{\baselinestretch}{1.5}\selectfont
\end{center}
\end{figure}


Below we will explore further the consequences of this assumption.
Let $\mathbf{e}$ denote an inertial observer. The observer $\mathbf{e}$ can describe a point $P$ in the 3D space that we live in by picking a point $O$ in space and setting three mutually perpendicular unit length axis\footnote{Show that this can be done since $\mathbf{e}$ sees space to be Euclidean.} at $O$. The basic assumption that space is isotropic and homogeneous allows us to pick any arbitrary point and any such mutually perpendicular axis. Such a set of axis is called an \textit{ortho-normal frame of reference}. Labelling the axis $\mathbf{e}_1, \mathbf{e}_2, \mathbf{e}_3$ to give a right handed orientation we can symbolically represent the frame as a row matrix 
$\mathbf{e}=[\mathbf{e}_1\:\:\: \mathbf{e}_2\:\:\: \mathbf{e}_3]$ where the $\mathbf{e}_1,\mathbf{e}_2$ and $\mathbf{e}_3$ are to be taken as symbols and nothing more. In this note we will always assume that the orthonormal frames are right hand oriented. Notice that with abuse of notation we have used $\mathbf{e}$ to denote the frame that corresponds to the observer $\mathbf{e}$.
Using such a frame,  any point $P$ in 3D-Euclidean space can be uniquely described using only the three measurements (numbers) $x_1,x_2$ and $x_3$.  These three numbers 
describe respectively the distance to the point along the $\mathbf{e}_1,\mathbf{e}_2,$ and $\mathbf{e}_3$ directions as shown in figure \ref{Fig:PointDefn}. Conversely the assumption that 3D-Euclidean space is continuous implies that any ordered triple of real numbers 
$(x_1,x_2,x_3)\in \mathbb{R}^3$ can be used to represent a unique point in 3D-Euclidean space. Symbolically we describe this identification as
\[
OP \triangleq  x_1\mathbf{e}_1+x_2\mathbf{e}_2+x_3\mathbf{e}_3=\underbrace{\left[ \begin{array}{ccc}\mathbf{e}_1 & \mathbf{e}_2 & \mathbf{e}_3 \end{array} \right]}_{\mathbf{e}} \,\underbrace{\left[ \begin{array}{c} x_1\\x_2\\x_3 \end{array} \right]}_x =\mathbf{e}\,x.
\]
In these notes the matrix
\[
x=\left[ \begin{array}{c} x_1\\x_2\\x_3 \end{array} \right],
\]
will be referred to as the Euclidean \textit{representation} matrix, or simply the representation, of the position of the point $P$ with respect to the frame $\mathbf{e}$. The components $ (x_1,x_2,x_3)\in \mathbb{R}^3$
will be referred to as the position components of $P$, or the \textit{Euclidean coordinates} of $P$  with respect to $\mathbf{e}$ and again with a bit of abuse of notation we will also denote this ordered triple by the same symbol, $x$, that denotes it matrix counterpart given above. We will use the terms measurement system, frame, coordinates, and observer to mean the same thing.

Consider two points $P$ and $Q$ in 3D-Euclidean space with Euclidean coordinates 
$x\triangleq (x_1,x_2,x_3)\in \mathbb{R}^3$ and $y\triangleq (y_1,y_2,y_3)\in \mathbb{R}^3$ respectively with respect to some frame $\mathbf{e}$.
Recall that the standard \textit{inner product} in $\mathbb{R}^3$ is defined by 
\[
<<x\,,\,y>>\triangleq x_1y_1+x_2y_2+x_3y_3.
\]
Then, by virtue of the Pythagorean theorem for 3D-Euclidean space, one sees that the measured distance between the two points $P$ and $Q$, in 3D-Euclidean space is equal to
\begin{align}
d(P,Q)&\triangleq ||x-y|| = \sqrt{<<(x-y)\,,\,(x-y)>>}=\sqrt{(x_1-y_1)^2+(x_2-y_2)^2+(x_3-y_3)^2}.\label{eq:EuclideanDistance}
\end{align}
Using the inner product we can also  define the angle between the lines $OP$ and $OQ$ by the relationship
\[
\theta \triangleq \cos^{-1}\left(\frac{<<x\,,\,y>>}{||x||\cdot ||y||}\right).
\]

\begin{svgraybox}
Thus we see that the fundamental assumptions of classical space-time imply that an inertial observer can construct an orthonormal frame in space and use it as its measurement system to define distances between points in space in such a way that all inertial observers will agree upon this measurement. In other words for every inertial observer there exists a globally defined coordinate system for space such that the distance between any two points with coordinates, $x\triangleq (x_1,x_2,x_3)\in \mathbb{R}^3$ and $y\triangleq (y_1,y_2,y_3)\in \mathbb{R}^3$, is given by (\ref{eq:EuclideanDistance}).
\end{svgraybox}
The construction of the orthonormal frame and the use of the clock allows an observer $\mathbf{e}$ to assign the ordered quadruple $(t,x)\in \mathbb{R}^4$ where $t\in \mathbb{R}$ and  $x\in \mathbb{R}^3$ to a space-time event in a unique way. A different measurement system, $\mathbf{e}'$ may provide a different identification  $(\tau,\xi)\in \mathbb{R}^4$ where $\tau\in \mathbb{R}$ and  $\xi\in \mathbb{R}^3$. When comparing the motion described by the two observers we need to know how the two representations (coordinates) are related to each other.  That is we must find the functions $\tau(t,x)$ and $\xi(t,x)$.
The homogeneity assumption of space-time implies that $\tau(t_1+T,x_1+a)-\tau(t_2+T,x_2+a)=\tau(t_1,x_1)-\tau(t_2,x_2)$, and $\xi(t_1+T,x_1+a)-\xi(t_2+T,x_2+a)=\xi(t_1,x_1)-\xi(t_2,x_2)$ for all $a, T$ and $t_1,t_2,x_1,x_2$. This implies that necessarily $\tau=a+b t+c x$ and $\xi=\gamma+\beta t+R x$ where $c,\gamma, \beta\in \mathbb{R}^3$ and $a,b\in \mathbb{R}$ are constant and $R$ is a $3\times 3$ constant matrix\footnote{Showing this only requires the knowledge of partial derivatives.}.
 

The assumption that time is independent of space implies that $c=0$ and the assumption that all inertial observers see the same intervals of time means that necessarily  $b=1$ and hence that $\tau=t+a$. Hence all inertial observers measure time up to an ambiguity of an additive constant and thus  without loss of generality we may assume that all observers have synchronized their clocks and hence that $a=0$. This also implies that a \emph{universal clock} exists.  
Furthermore the assumption that space intervals are inertial observer independent implies that, $||\xi(t,x_1)-\xi(t,x_2)||=||x_1-x_2||$. Thus $||R (x_1-x_2)||=||x_1-x_2||$ for all $x_1, x_2$. Thus necessarily $R$ must be an orthogonal\footnote{A matrix that satisfies the properties $R^TR=RR^T=I$ is called an orthogonal transformation.} constant transformation. 
Since the space is observed to be homogeneous by all inertial observers without loss of generality we may choose $\gamma=0$ \footnote{Note that choosing $\gamma=0$ amounts to assuming that the origin of the spatial frames of both observers coincide at the time instant $t=0$ and does not sacrifice any generality since the space is homogeneous we can parallel translate the frames until they coincide at the time instant $t=0$.}.  Thus we see that the representation of the same space-time event by two different inertial observers are related by $(\tau,\xi)=(t,\beta t+Rx)$. You are asked to complete the details of these arguments in exercise-\ref{HomgenetyInertialFrame}.

Since in the following sections we will see that the orthonormal frames are related to each by such an orthogonal transformation it follows from $\xi=Rx$ that the frame used by $\mathbf{e}'$ to make spatial measurements is also an orthonormal frame.
Let $O'$ be the origin of the orthonormal frame used by $\mathbf{e}'$. If the space-time event $O'$ has the representation $(t,o)$ according to the observer $\mathbf{e}$, it has the representation $(t,\beta t+ Ro)=(t,0)$ according to the observer $\mathbf{e}'$. Thus we have that $\beta=-R\dot{o}=-Rv$ where $v=\dot{o}$ and hence that the velocity of the center of the $\mathbf{e}'$ frame with respect to the $\mathbf{e}$ frame, given by $v=\dot{o}=-R^T\beta$, must be a constant. That is we see that all inertial observers must necessarily translate at constant speed with respect to each other without rotation.
This also shows that the representation of a space-time event denoted by $(t,x)$ according to $\mathbf{e}$ must necessarily have the representation $\left(t,R(x-vt)\right)$ for some constant $v\in \mathbb{R}^3$ according to any other inertial frame $\mathbf{e}'$. Space is homogeneous only for such observers. In particular we can see that this is not the case for observers rotating with respect to an inertial observer $\mathbf{e}$. That is a rotating observer will not observe space to be homogeneous\footnote{Show that this is true.}. Since $R$ is a constant, without loss of generality, one can always pick the orthonormal frame used by $\mathbf{e}'$ to be parallel to the one used by $\mathbf{e}$ so that $R=I_{3\times 3}$. Then we see that $\xi(t)=x(t)-vt$ in parallel translating inertial frames. It is traditional to refer to parallel frames that translate at constant velocities with respect to each other as \textit{inertial frames}.
\begin{svgraybox}
In summary Axiom-\ref{ax:GalileanSpaceTime} implies that there exists a special class of observers called inertial observers who see that time is a universal quantity and that a special class of spatial coordinates called Euclidean coordinates for 3D-space exists such that the distance between any two points in space is given by (\ref{eq:EuclideanDistance}) in an inertial observer independent manner. From a physical point of view it follows that any two such observers must necessarily be moving with constant relative velocity with respect to each other without rotation.
\end{svgraybox}
%%%%%%%%%%%

\subsection{Description of Motion}
\begin{figure}[ht]
\begin{center}
\includegraphics[width=3in]{VelocityDefn}
\renewcommand{\baselinestretch}{1}\selectfont
\caption{The Velocity of a Particle described with respect to a frame $\mathbf{e}$.}
\label{Fig:VelocityDefn}
\renewcommand{\baselinestretch}{1.5}\selectfont
\end{center}
\end{figure}

Consider a particle moving in 3D-Euclidean space. We have seen that the position $P(t)$ of the particle at a given time $t$ can be expressed by the Euclidean representation matrix $x(t)$ with respect to some orthonormal
frame $\mathbf{e}$. Since the position of the particle is changing with time, $x(t)$ is a function of time and describes a curve in $\mathbb{R}^3$ that is parameterized by time $t$.
The velocity of a point $P$ is always defined with respect to an orthonormal reference frame. 
Specifically 
with respect to the $\mathbf{e}$ frame it is defined to be the infinitesimal change of position with respect to $\mathbf{e}$. That is
\[
\dot{x}(t)\triangleq \lim_{\delta t \rightarrow 0}\frac{x(t+\delta t)-x(t)}{\delta t}=\left[\begin{array}{c} \dot{x}_1\\ \dot{x}_2\\ \dot{x}_3\end{array}\right].
\]
Observe that by definition $\dot{x}(t)$ gives the tangent to the curve at $P(t)$ (Refer to figure \ref{Fig:VelocityDefn}).
The components
\[
\dot{x}_i(t)=\lim_{\delta t \rightarrow 0}\frac{x_i(t+\delta t)-x_i(t)}{\delta t}\:\:\:\:\: i=1,2,3
\]
represent the infinitesimal change of position in the $\mathbf{e}_i$ direction.
Thus the \textit{correct} way to visualize $\dot{x}(t)$ is to consider it as the description of a point in a frame that is parallel to $\mathbf{e}$ but with origin at $P(t)$ (Refer to figure 
\ref{Fig:VelocityDefn}) or alternatively as arrows at $P(t)$.
The magnitude of the velocity in the $\mathbf{e}$ frame is defined to be
\[
\textsl{v}\triangleq||\dot{x}||=\sqrt{\dot{x}_1^2+\dot{x}_2^2+\dot{x}_3^2}.
\]



The acceleration of a particle is also defined in terms of orthonormal frames. In an orthonormal frame $\mathbf{e}$ it is defined to be the infinitesimal change of velocity in the $\mathbf{e}$ frame,
\[
\ddot{x}(t)\triangleq\lim_{\delta t \rightarrow 0}\frac{\dot{x}(t+\delta t)-\dot{x}(t)}{\delta t}=\left[\begin{array}{c} \ddot{x}_1\\ \ddot{x}_2\\ \ddot{x}_3\end{array}\right].
\]
\\
Animals have inbuilt sensors that allow them to measure accelerations. For example, the human ear contains accelerometers that allow them to sense the direction of the 
gravitational accelerations. It is interesting to note that animals can not sense absolute velocities and thus can not distinguish between being at rest or being in constant velocity motion.


Consider a certain space-time event $A$ and let $\mathbf{e}$ and $\mathbf{e}'$ be two inertial observers with parallel frames. Let $(t,x)\in \mathbb{R}^4$ and $(\tau,\xi)\in \mathbb{R}^4$ be the representation of the space-time event $A$ made by 
$\mathbf{e}$ and $\mathbf{e}'$ respectively. Since all inertial observers agree that space-time is homogeneous we saw previously that the two representations are related by $\tau=t$ and that
$\xi=x-vt$ for some constant $v\in \mathbb{R}^3$. Thus the representation of the velocity of a particle in the two frames are related by 
$\dot{\xi}=\dot{x}-v$ and the acceleration of a particle in the two frames are related by $\ddot{\xi}=\ddot{x}$.
Therefore we see that if a particular observer, $\mathbf{e}$, sees that an object is moving at a certan acceleration then all parallel inertial observers with respect to $\mathbf{e}$ will also observe that the object is moving at that same acceleration. This allows us to conclude that even though the position and velocity of a particle are not the same for all inertial observers with parallel frames the acceleration of a particle is observed to be the same for all inertial observers with parallel frames. Hence we can conclude the following:
\begin{svgraybox}
The isotropy, homogeneity, and continuity of space-time imply that the acceleration of a particle is a parallel translating inertial observer independent quantity. That is, if we represent the motion of a particle using any inertial coordinate system with parallel frames the acceleration computed in all of these frames will be the same.
\end{svgraybox}






\subsection{The Apparent Cause of Change of Motion}\label{Secn:Kinetics}

Kinetics deal with the apparent causes of motion. Galileo Galilei in the 17th century made the observation that a person doing experiments, with moving objects, below the deck in a ship traveling at 
constant velocity, without rocking, on a smooth sea;  would not be able to tell whether the ship was moving or was stationary\footnote{It is interesting to note that animals can only feel accelerations and not absolute velocities. Thus one could not differentiate between being in a ship moving at constant 
velocity and being in a ship at rest.}. Thus he concluded that laws of nature that describe the 
motion of objects must be the same in all inertial frames.  At the end of the previous section we have seen that acceleration of a given particle is seen to be the same for all inertial observers with parallel frames.
Therefore, in order to be the same in all inertial frames, the laws of nature that govern particle motion must necessarily depend only on the acceleration. 


Based on experiments, involving colliding rolling balls on a smooth horizontal surface, it was observed that objects generally tend to move at constant velocities or remain at rest 
unless brought into interaction with other objects. This idea was generalized by Galileo in his \emph{principle of inertia} where he stated that, all inertial observers will see that, an object that does not interact with any other 
object will move at a constant speed or will remain at rest\footnote{Newton's First law.}. Thus for an isolated matter particle we would have that $\ddot{x}=0$ in any and every inertial frame and thus that it is an invariant property observed by all inertial observers. This also means that all inertial observers will agree that an isolated matter particle is either at rest or is moving in a straight line. It was also observed that there was a certain resistance to change in this steady motion and that this resistance depended on the amount of matter present in the object. Bigger objects would tend to change its motion slower than smaller objects that were made of the same material. The measure of the degree of this resistance to change in 
motion is referred to as the \textit{inertia} of the object. Careful experiments with two colliding balls on a smooth surface indicated that a unique number can be ascribed to each ball such that if for each ball you multiply this number with the velocity of the ball, and added the results together, this sum will always remain a constant irrespective of whether they collide or not. It is easy to see that all inertial observers will also agree on this statement even though the constant they obtain will be different. Thinking of interacting particles, Galileo hypothesised that this number that multiplies the velocity of the particle was the same for all inertial observers and must be an intrinsic property of the object.  This is one of the crucial assumptions of Galilean mechanics:

\begin{svgraybox}
\begin{axiom}{\sf The Principle of Conservation of Linear Momentum:} 
Assume that there exists an inertial observer independent unique property called mass that can be assigned to each and every particle in the Universe and that all particles interact with each other. Define {linear momentum} of a particle in an inertial frame to be the mass times velocity of the particle in the inertial frame. An isolated set of particles interact with each other in such a way that the total sum of the linear momentum of the set of particles always remains constant when observed in any inertial frame. 
\end{axiom}
\end{svgraybox}
\noindent The assumptions on classical space-time stated in Axiom-\ref{ax:GalileanSpaceTime} along with the above principle of conservation of linear momentum are the fundamental principles (laws) on which Galilean mechanics is founded upon. 

Let us investigate what, additional information, the principle of conservation of linear momentum gives us about a system of interacting but isolated set of particles $P_1,P_2,\cdots,P_n$. That is, a set of particles that interact with themselves but do not interact with any other particles of the universe. This is an idealization of a real situation where the external interactions are much weaker compared to the internal ones. Let $m_i$ be the mass of $P_i$ and $\dot{x}_i$ be the velocity of $P_i$ in some inertial frame $\mathbf{e}$. Then the conservation of linear momentum in $\mathbf{e}$ gives us that $\sum_{i=1}^{n}m_i\dot{x}_i=\mathrm{constant}$. The principle of conservation of linear momentum says that in a different inertial frame $\mathbf{e}'$ this quantity still remains constant as well. However in general the two constants will not be the same. That is, although two observers in two different inertial frames will agree that the total linear momentum of the system of particles is conserved, in general they will measure two different values for this constant. Thus though conservation of linear momentum is inertial frame invariant, the total linear momentum is not.

Differentiating the expression $\sum_{i=1}^{n}m_i\dot{x}_i=\mathrm{constant}$ one sees that in the $\mathbf{e}$-frame
\begin{equation}\label{eq:Force}
\sum_{i=1}^{n}m_i\ddot{x}_i=0.
\end{equation}
Observe that now all inertial observers will agree that this sum is equal to zero. The expression (\ref{eq:Force}) implies that an isolated particle ($n=1$) will move at constant velocity in any inertial frame (\textit{Galileo's principle of inertia or what is commonly known as Newton's $1^{st}$ law of motion}).
The above sum can be re-arranged to give
\[
m_j\ddot{x}_j=-\sum_{\stackrel{i=1}{i\neq j}}^{n}m_i\ddot{x}_i.
\]
This says that the mass times the acceleration of the $j^{\mathrm{th}}$ particle is not free but is constrained due to the interaction it has with the rest of the particles. This constraint imposed on the mass times the acceleration of the $j^{\mathrm{th}}$ particle is defined to be the \textit{force} acting on the  $j^{\mathrm{th}}$ particle and in this case is given by$f_j=-\sum_{\stackrel{i=1}{i\neq j}}^{n}m_i\ddot{x}_i$. Then the above expression gives $m_j\ddot{x}_j=f_j$.  This is nothing but the statement of \textit{Newton's $2^{nd}$ law of motion}. This also shows that forces observed is the same in all inertial frames.
Considering two interacting particles (\ref{eq:Force}) also shows that mutual particle interactions are equal and opposite (\textit{Newton's $3^{rd}$ law of motion}). Thus we see that the three Laws of Newton are a direct consequence of the principle of conservation of linear momentum in inertial frames.  However note that it does not tell us that the mutual interaction between two particles must lie in the direction of the line joining the two particles. Nevertheless from a macroscopic point of view it is an empirically observed fact that this in fact is true and will be taken as a separate hypothesis of the nature of force:
\begin{svgraybox}
\begin{axiom}\label{axiom:DirectionOfForces}
Assume that, taken pairwise, mutual particle interaction forces act along the straight line that joins the two particles.
\end{axiom} 
\end{svgraybox}
We will see later that this assumption is crucial in ensuring that a quantity known as angular momentum of a set of isolated but interacting set of particles be conserved. Therefore one may replace this hypothesis with the equivalent hypothesis that the particles interact in such a way that the total angular momentum of the universe is conserved.


The quantity, \textit{force}, that effects a change in the motion of a particle thus is a consequence of its interaction with other particles such that the total linear momentum of all the interacting particles remain constant in any inertial frame. Four types of fundamental particle interactions have been observed so far. They are the strong 
(nuclear), electro-magnetic, electro-weak, and gravitational interactions. When taken pairwise these interactions are assumed to act along the line joining the two interacting particles. Since we have seen that particle accelerations are inertial observer invariant so are these forces. The change of motion that a matter particle or an object undergoes is due to the manifestation of one or many of these interactions.


All experiments conducted verified the principle of conservation of linear momentum and the resulting Newton's laws up until the end of the 19th century to the precision allowed by the instruments of that day.  Towards the end of the 19th century and around the early period of the 20th century the advances in technology brought about by the development of Galilean mechanics and the Maxwell's laws of electro-magnetism made it possible to observe and measure phenomena at very small time and length scales and at much faster speeds and larger distances. This capability began to unearth certain phenomena; regarding objects that were very small and that moved very close to the speed of light, and at the same time about the motion of very large objects that were very far from earth, that could not be described accurately by the afore referred Galilean laws of motion. The search to explain these phenomena gave birth to Einstein's principle of relativity and to the principles of quantum mechanics. Most of the objects of motion that we are interested in Engineering are macroscopic bodies (that is much larger than the afore referred microscopic objects), moving at speeds far less than the speed of light. It still remains valid that Galilean laws of mechanics predict the behavior of such objects to a sufficiently high degree of precision compared to the size and speed of the objects.
 
 
\begin{svgraybox}
In summary the fundamental notions on which Galilean mechanics is founded upon are:
\begin{enumerate}[i.]
\item objects are made up of impenetrable particles that interact with each other in an observer independent manner,
\item when taken pairwise particles interactions lie along the straight line that joins the two particles,
\item there exists a class of observers called inertial observers who agree that space is 3D and Eulcledian, and that time is 1D and universal,
\item associated with each particle there exists an inertial observer independent quantity called mass,
\item the total linear momentum of all the particles in the Universe is always conserved when viewed in any inertial frame.
\end{enumerate}


Based on these it can be deduced that all particles interact in a manner that is independent of the inertial frame of reference and these interactions are the causes of change in motion. The change in motion of a particle in any given inertial frame $\mathbf{e}$ is described by the mathematical expression 
\begin{equation}\label{eq:NewtonsLaw}
m\ddot{x}=f(t),
\end{equation}
where $f(t)$ is defined to be the force acting on the particle that arises due to its interaction with the rest of the particles in the universe and $m$ is the observer invariant property of the particle called mass. \end{svgraybox}


It is interesting to note, according to the expression (\ref{eq:NewtonsLaw}), that measuring accelerations allows one to measure forces. Even though they can be theoretically estimated using the knowledge of 
the fundamental interactions of particles, it turns out that this is the only way we can measure forces. Thus the above expression, if you may, can be taken to be the definition of force.
Since the knowledge of the fundamental interactions allow us to estimate forces, equation (\ref{eq:NewtonsLaw}) can be used to predict the motion of the point particle. That is what 
is monumental about this law. From a mathematical perspective equation (\ref{eq:NewtonsLaw}) describes a second order differential equation and solving it for $x(t)$ requires the 
knowledge of the initial conditions $x(t_0)=x_0$ and $\dot{x}(t_0)=v_0$ and therefore in order to predict the motion of a particle what one needs is the knowledge of the initial state, 
$x(t_0)=x_0$ and $\dot{x}(t_0)=v_0$, and the knowledge of the force, $f(t)$, at all times, $t$.

\begin{example}\label{Example:Force}
Consider the problem of a horizontal spring with one end fixed to a support and the other end fixed to an object, of mass $m$, that moves on a smooth horizontal table. We assume that the object is symmetric and small so that we can approximate it as a point particle with mass $m$. If we 
give an initial horizontal displacement to the object we know empirically that the object will exhibit a simple harmonic motion if the air and other resistances on the object are
negligible and the motion is small.
That is, if $x(t)$ is the displacement of the object from the un-stretched position and if the air resistance on the object is negligible and the motion is small, the position $x(t)$ of the object $P$ at a given time $t$ is described sufficiently accurately by the second order differential equation
\[
m\ddot{x}(t)=-k\,x(t).
\]
Observing this expression it is evident that the mass times the acceleration of the object is constrained and is equal to $-k\,x(t)$. Thus we could call $-k\,x(t)$ the force exerted on the object due to its collective interaction with all the particles that makeup the
spring\footnote{This is known as the Hooke's law.}. Considering the fact that a spring is made of atoms that interact in a manner where they repulse 
each other when they are too close and attract when they are sufficiently away\footnote{This arises due to the electro-magnetic interactions due to the electrons and protons that 
makeup the particle.} we may also theoretically estimate this law. These are the two fundamental ways in which forces are determined in practice.
\end{example}

%%%%%%%%%%%%%%%%%%%%%%%%%%%%%




\subsection{The Motion of a Set of Interacting Particles}\label{Secn:InteractingParticleMotion}
In this section we will investigate the additional consequences of the law of conservation of linear momentum has on the motion of a set of particles $P_1,P_2,\cdots,P_n$. For example this could be a body of fluid particles (a deformable body) or a body of particles rigidly fixed with respect to each other (a rigid body). 

Let us begin by consider the description of the motion of a single particle $P_i$ of mass $m_i$ in an inertial frame $\mathbf{e}$ with origin $O$. Denote by $x_i$ and $\dot{x}_i$ its position and velocity respectively in the inertial frame $\mathbf{e}$. By definition the linear momentum of particle $i$ in the inertial frame $\mathbf{e}$ is $p_i\triangleq m_i\dot{x}_i$.
\begin{svgraybox}
From (\ref{eq:NewtonsLaw}) we see that 
\[
\dot{p}_i=f_i,
\]
where $f_i$ is the force acting on the particle due to its interaction with all the other particles in the universe.
This expression states that the rate of change of linear momentum of a particle in an inertial frame is equal to the force acting on the particle. This is as an equivalent statement of Newton's second law. 
\end{svgraybox}
The law of conservation of momentum tells us that the total linear momentum of a collection of interacting but otherwise isolated set of particles $P_1,P_2,\cdots,P_n$ is conserved in any inertial frame 
$\mathbf{e}$. That is in particular
\[
p\triangleq \sum_{i=1}^np_i=\sum_{i=1}^nm_i\dot{x}_i=\mathrm{constant}.
\]

When there are other external influences on the set of particles the total linear momentum of the set of particles is not conserved. To see this we will split the interaction force that the $i^{\mathrm{th}}$ particle experiences as 
\begin{align*}
f_i=f_i^e+\sum_{j\neq i}^nf_{ij}
\end{align*}
where $f_{ij}$ is the force on $i$ due to its interaction with $j$.
Note that since particle interactions are equal and opposite we have that the interaction of $j$ on $i$ denoted by $f_{ij}$ is equal and opposite to the interaction of $i$ on $j$ denoted by $f_{ji}$. That is $f_{ij}=-f_{ji}$. Hence we have that
\begin{align*}
\sum_{i=1}^nf_i=\sum_{i=1}^nf_i^e+\sum_{i=1}^n\sum_{j\neq i}^nf_{ij}= \sum_{i=1}^nf_i^e\triangleq f^e
\end{align*}
where $f^e$ represents the total resultant of the external interactions acting on the set of particles $P_1,P_2,\cdots,P_n$.
Thus we have $\dot{p}=f^e$.
This also implies the following. Let $\bar{x}$ be the representation of the \emph{center of mass} $O_c$ of the particles in the $\mathbf{e}$ frame.
That is let
\begin{align}
\bar{x}&\triangleq \frac{\sum_{i=1}^{n}m_ix_i}{\sum_{i=1}^{n}m_i}.
\end{align}
Defining $M\triangleq \sum_{i=1}^{n}m_i$ and
differentiating this we get
$M\dot{\bar{x}}= {\sum_{i=1}^{n}m_i\dot{x}_i}= {p}$
and differentiating again 
we get $M\ddot{\bar{x}}= f^e$.
This only describes the center of mass motion of the set of particles and not the motion that is relative to the center of mass that we clearly observe for instance with rigid body motion in space. In order to capture this motion the notion of the rotational ability of a particle at $P_i$ about a point $O'$ in space is defined as follows. 
Let $OO'=\mathbf{e}\,o$.
\begin{svgraybox}
The \textit{angular momentum} of a particle $P_i$ in the $\mathbf{e}$ frame about the point $O'$ (fixed or otherwise) is defined to be
\begin{align}
\pi_i &\triangleq (x_i-o)\times m_i\dot{x}_i.\label{eq:AngularMomentum_i}
\end{align}
\end{svgraybox}
Differentiating this we have 
\begin{align}
\dot{\pi}_i&= (\dot{x}_i-\dot{o})\times m_i\dot{x}_i+(x_i-o)\times m_i\ddot{x}_i=-\dot{o}\times m_i\dot{x}_i+(x_i-o)\times f_i.\label{eq:ChangeAngMomentum}
\end{align}
\begin{svgraybox}
The quantity 
\begin{align}
\tau_i &\triangleq (x_i-o)\times f_i,\label{eq:ForceMomentum}
\end{align} 
is defined to be the \textit{moment of the force} about $O'$. Combining this with (\ref{eq:ChangeAngMomentum}) we have
\begin{align}
\dot{\pi}_i&= -\dot{o}\times m_i\dot{x}_i+\tau_i.\label{eq:ChangeAngMomentum1}
\end{align}
\end{svgraybox}



Recall that since the force acting on a particle is due to its interaction with the  particles under consideration and the rest of the particles in the universe we have $f_i=f_i^e+\sum_{j\neq i}^nf_{ij}$. Since  $f_{ji}=-f_{ij}$ and they lie along the straight line joining the two particles we also have that
$(x_i-o)\times  f_{ij}=-(x_j-o)\times  f_{ji}$ which implies that 
\begin{align*}
\sum_{i=1}^n\tau_i=\sum_{i=1}^n(x_i-o)\times f_i=\sum_{i=1}^n(x_i-o)\times  f_{i}^e+\sum_{i=1}^n\sum_{j\neq i}^n(x_i-o)\times  f_{ij}=\sum_{i=1}^n(x_i-o)\times  f_{i}^e.
\end{align*}
We will define the quantity 
\begin{align*}
{\tau^e}\triangleq \sum_{i=1}^n(x_i-o)\times  f_{i}^e.
\end{align*}
to be the \emph{total moment of the external forces acting about the point $O'$}.
Thus the rate of change of total angular momentum $\pi=\sum_{i=1}^n\pi_i$ about a point $O'$ is
\begin{align*}
\dot{\pi}&=\sum_{i=1}^n\dot{\pi}_i= -\dot{o}\times \sum_{i=1}^nm_i\dot{x}_i+\sum_{i=1}^n\sum_{j\neq i}^n(x_i-o)\times  f_{ij}+\sum_{i=1}^n(x_i-o)\times  f_{i}^e,\\
&=-\dot{o}\times \sum_{i=1}^nm_i\dot{x}_i+\tau_e=-M\dot{o}\times \dot{\bar{x}}+\tau_e.
\end{align*}

This expression tells us that $\dot{\pi}=\tau^e$  if $\dot{o}=0$ or if  $\dot{\bar{x}}=0$ or if $O'=O_c$. However note that $\dot{o}=0$ or  $\dot{\bar{x}}=0$ are conditions that not all inertial observers will agree upon. Thus we have the following conclusion that all inertial observers will agree upon:  the total angular momentum of a set of interacting but otherwise isolated set of particles  about the center of mass of the set of particles remain constant when observed in any inertial frame. Notice that the assumption that pairwise particle interactions that lie along a straight line that joins the two particles was crucial for this conclusion. Thus one may replace that assumption with the sometimes more widely used assumption that total angular momentum of the universe is conserved. 



\begin{svgraybox}
In summary, what we have seen is that, in general if $P_1,P_2,\cdots,P_n$ are a set of particles that are interacting with themselves and the rest of the universe, then the law of conservation of linear momentum of the universe along with the assumption that pairwise particle interactions lie along the straight line joining the two particles imply that:
\begin{enumerate}[(a)]
\item the rate of change of total momentum of the set of particles is equal to the total resultant of the external forces acting on the system of particles. That is
\begin{align}
\dot{p}=f^e.\label{eq:TotalRateOfP}
\end{align}
\item the set of particles move in such a way that its center of mass moves according to the motion of a particle of mass $M=\sum_{i=1}^nm_i$ that is under the influence of the force $f^e=\sum_{i=1}^nf_i^e$. That is
\begin{align}
M\ddot{\bar{x}}=f^e.\label{eq:TotalRateOfCM}
\end{align}
\item the rate of change of the total angular momentum of the system of particles about its center of mass is equal to the total resultant of the moments of the external forces acting on the system. That is
\begin{align}
\dot{\pi}=\tau^e. \label{eq:RateofchangeofTotalPi}
\end{align}
\end{enumerate}
\end{svgraybox}



\emph{We emphasize that these conclusions are valid for any collection of particles such as in a deformable body or a rigid body}. A straight forward corollary of (\ref{eq:TotalRateOfCM}) and  (\ref{eq:RateofchangeofTotalPi}) for an isolated but mutually interacting set of particles is that the velocity of the center of mass of the particles remains constant and that the total angular momentum of the set of particles about its center of mass is conserved  in any inertial frame . 

%%%%%%%%%%%%%%%%%%%%%%%%%%%%




%%%%%%%%%%%%%%%%%%%%%%%%%

\subsection{Kinetic Energy}

Another fundamental property of motion is the \emph{energy} associated with it. Consider a particle of mass $m$ moving under the influence of a force $f$ (arising due to the interactions it has with the rest of the universe). The kinetic energy of the particle is defined with respect to an \textit{inertial frame} $\mathbf{e}$ and is given by the relationship
\begin{equation*}
\mathrm{KE} \triangleq \frac{1}{2}m ||\dot{x}(t)||^2 =\frac{1}{2m} ||p(t)||^2.
\end{equation*}
Observe that this quantity changes from inertial frame to inertial frame and hence is not a quantity that is invariant for all inertial observers. In general, unlike total linear momentum and total angular momentum, the total kinetic energy of a system of isolated particles need not remain constant. However differentiating the Kinetic energy associated with given particle we see that along the motion of a particle
\begin{equation*}
\dfrac{d }{dt} \mathrm{KE}= m \langle \langle \ddot{x}(t),\dot{x}(t)\rangle\rangle=\langle \langle f(t),\dot{x}(t)\rangle\rangle.
\end{equation*}
The quantity 
\[
W(t_1,t_2)=\int_{t_1}^{t_2}\langle \langle f(t),\dot{x}(t)\rangle\rangle\,dt
\]
is defined to be the \emph{work done by the force} $f$ acting on the particle during the time interval $[t_1,t_2]$. 
\begin{svgraybox}
Thus we have that the rate of change of kinetic energy of a particle is equal to the rate of work done by the force acting on the particle. This is nothing but a statement of conservation of energy of a set of interacting particles. The rate of work done is called the \emph{power} of the force.
\end{svgraybox}


%%%%%%%%%%%%%%%%%%%

\subsection{Particle Collisions and Thermalization}
In what follows we consider what happens to the total kinetic energy of a set of colliding particles that are isolated from the rest of the universe. We will assume that the particles interact with each other only when they collide with each other.
Let us first consider what happens when only two particles collide with each other. Let the two particles have a mass of $m_1$ and $m_2$ respectively. Let $\mathbf{e}$ be an inertial frame and $\mathbf{b}$ be a frame that moves parallel to $\mathbf{e}$ with origin coinciding with the center of mass of the two particles. We will call the frame $\mathbf{b}$ the center of mass frame. Let $x_1, x_2$ be the Euclidean representation of the two particles in $\mathbf{e}$ and let $X_1,X_2$ be the representation of the two particles in $\mathbf{b}$. Let $o$ be the representation of the center of mass of the particles in the $\mathbf{e}$ frame. Then $x_i=o+X_i$.
Since the origin of $\mathbf{b}$ is at the center of mass of the two particles we have
\begin{align}
m_1X_1+m_2X_2 &=0.\label{eq:cmp1}
\end{align}
Differentiating this we have
\begin{align}
m_1\dot{X}_1+m_2\dot{X}_2&=0.\label{eq:cmv1}
\end{align}
Thus the 
total linear momentum of the particles in the $\mathbf{e}$ frame can be expressed as
\begin{align}
m_1\dot{x}_1+m_2\dot{x}_2&= (m_1+m_2)\dot{o}
\end{align}
and the total kinetic energy of the particles can be expressed as
\begin{align}
\mathrm{KE}&=\frac{m_1}{2}||\dot{x}_1||^2+\frac{m_2}{2}||\dot{x}_2||^2,\\
&=\frac{(m_1+m_2)}{2}||\dot{o}||^2+\frac{m_1}{2}||\dot{X}_1||^2+\frac{m_2}{2}||\dot{X}_2||^2.
\end{align}

Let us consider what happens in a collision. Denote by a superscript $'$ the variables that describe the motion of the particle after collision. Principle of conservation of momentum implies that
\begin{align*}
(m_1+m_2)\dot{o}&=(m_1+m_2)\dot{o}'.
\end{align*}
This shows that the \textit{center of mass velocity is unaltered by the collision}\footnote{Note that as shown by (\ref{eq:TotalRateOfCM}) this conclusion holds in general as well.}.
Let us assume that the \textit{kinetic energy is not lost in the collision}\footnote{Such collisions are called elastic collisions.}. Thus $\mathrm{KE}=\mathrm{KE}'$. This gives us
\begin{align*}
\frac{(m_1+m_2)}{2}||\dot{o}||^2+\frac{m_1}{2}||\dot{X}_1||^2+\frac{m_2}{2}||\dot{X}_2||^2 &=
\frac{(m_1+m_2)}{2}||\dot{o}'||^2+\frac{m_1}{2}||\dot{X}_1'||^2+\frac{m_2}{2}||\dot{X}_2'||^2 .
\end{align*}
Since $\dot{o}=\dot{o}'$ we have
\begin{align*}
\frac{m_1}{2}||\dot{X}_1||^2+\frac{m_2}{2}||\dot{X}_2||^2 &=
\frac{m_1}{2}||\dot{X}_1'||^2+\frac{m_2}{2}||\dot{X}_2'||^2 .
\end{align*}
From (\ref{eq:cmv1}) we have
\begin{align*}
\dot{X}_2&=-\frac{m_1}{m_2}\dot{X}_1,\:\:\:\:\:\:\dot{X}_2'=-\frac{m_1}{m_2}\dot{X}_1',
\end{align*}
and hence
\begin{align*}
||\dot{X}_1||^2 &=||\dot{X}_1'||^2,\:\:\:\:\:\: ||\dot{X}_2||^2 =||\dot{X}_2'||^2.
\end{align*}
This shows that, in a perfectly elastic collision, the magnitude of the velocities of each of the particles do not change when viewed in the center of mass frame. Expression (\ref{eq:cmp1}) also tells us that in the center of mass frame the two particles appear to move in a straight line through the origin. Thus what changes in a collision is only the angle of this straight line. The cosine of the angle between this line and the velocity of the center of mass can be found from $\dot{o}\cdot(\dot{X}_2-\dot{X}_1)$
\begin{align*}
\dot{o}\cdot(\dot{X}_2-\dot{X}_1)&=\dot{o}\cdot(\dot{x}_2-\dot{x}_1)=\left(\frac{m_1\dot{x}_1+m_2\dot{x}_2}{(m_1+m_2)}\right)\cdot(\dot{x}_2-\dot{x}_1)\\
&=\frac{m_2||\dot{x}_2||^2-m_1||\dot{x}_1||^2+(m_1-m_2)\dot{x}_1\cdot\dot{x}_2}{m_1+m_2}.
\end{align*}

Let us consider the case where the two particles have been contained in space so that the particles would have been colliding elastically with the walls and with each other for a long time. Since the only thing that changes in elastic particle collisions is their relative direction of motion viewed in a center of mass frame of the two particles we can 
hypothesize that the average of the angle between the center of mass motion and the motion relative to the center of mass, over time, should be zero. That is $
\langle \dot{o}\cdot (\dot{x}_2-\dot{x}_1)\rangle =0$.
Here we have used the customary notation of angle brackets to mean the time average over a long period of time.
This in essence assumes that \textit{on average there is no preferred directions of motion}\footnote{Justify using physically reasonable arguments why this assumption must be true.}. Thus it is also true that $
\langle \dot{x}_1\cdot \dot{x}_2\rangle=0$.
Under this assumption it follows that over a long period of time there is no correlation between the center of mass motion and the motion with respect to the center of mass. Then the above expression tells us that
\begin{align}
\left\langle \frac{m_1||\dot{x}_1||^2}{2}\right\rangle
=\left\langle \frac{m_2||\dot{x}_2||^2}{2}\right\rangle
\end{align}
Which tells us that \textit{after many collisions we could expect that the average kinetic energy of the two particles to be the same}. When the system consists of a large number of particles, taking particles pairwise into consideration we may conclude that after many collisions the average kinetic energy of all the interacting particles become the same. This is the condition that is known as \textit{thermal equilibrium}.

Let us conclude this section by summarizing what we have learnt in this section.
\begin{svgraybox}
\begin{enumerate}[(a)]
\item The law of conservation of linear momentum implies that the center of mass motion of a set of colliding particles do not change over time. 
\item In addition if the particle collisions are elastic then the average kinetic energy of all the colliding particles remain constant over time. This condition is known as \textit{thermal equilibrium}. 
\end{enumerate}
\end{svgraybox}
%%%%%%%%%%%%%%%%%%%%%%%%%%%%%%%%%%%%%%%%%%%%

%%%%%%%%%%%%%%%
\section{Description of Motion in Moving Frames}\label{Secn:RelativeMotion}

\begin{figure}[ht]
\begin{center}
\includegraphics[width=3.5in]{GeneralMovingFrame}
\renewcommand{\baselinestretch}{1}\selectfont
\caption{The description of motion of a particle $P(t)$ in two different orthonormal frames $\mathbf{e}$ and $\mathbf{b}$ representing two different observers.}
\label{Fig:GeneralMovingFrame0}
\renewcommand{\baselinestretch}{1.5}\selectfont
\end{center}
\end{figure}

We can easily see that the construction of orthonormal frames is not unique and that, in general, different observers can have different orthonormal frames. For instance consider figure-\ref{Fig:GeneralMovingFrame} where two observers have defined two right hand oriented orthonormal frames $\mathbf{e}=[\mathbf{e}_1\:\:\: \mathbf{e}_2\:\:\: \mathbf{e}_3]$ and $\mathbf{b}=[\mathbf{b}_1\:\:\: \mathbf{b}_2\:\:\: \mathbf{b}_3]$ respectively. The position of the particle $P(t)$ at a particular instant of time $t$ is described by the two observers using the Euclidean representation matrices $x=[x_1\:\:\:x_2\:\:\:x_3]^T$ and $X=[X_1\:\:\:X_2\:\:\: X_3]^T$ respectively. Let $o$ be the Euclidean representation of the point $O'$ in the orthonormal frame $\mathbf{e}$. That is let $OP=\mathbf{e}x$, $OO'=\mathbf{e}o$ and $O'P=\mathbf{b}X$. We are interested in determining the relationship between the observed motion in the two different frames.


\begin{figure}[ht]
\begin{center}
\includegraphics[width=3.0in]{MovingFrame}
\renewcommand{\baselinestretch}{1}\selectfont
\caption{Description of the point $P(t)$ in three orthonormal frames $\mathbf{e}$, $\mathbf{e}'$, and $\mathbf{b}$ representing two different observers.}
\label{Fig:GeneralMovingFrame}
\renewcommand{\baselinestretch}{1.5}\selectfont
\end{center}
\end{figure}

Let us begin by finding the relationship between the two representations $x$ and $X$. 
Introduce another frame $\mathbf{e}'=[\mathbf{e}'_1\:\:\:
 \mathbf{e}'_2\:\:\: \mathbf{e}'_3]$, as shown in figure-\ref{Fig:GeneralMovingFrame}, with origin coinciding with $O'$ such that its axis are of unit length and are also parallel to $\mathbf{e}$. Let the representation of the point $P$ in the $\mathbf{e}'$ frame be $x'$. That is let $O'P=\mathbf{e}'x'$.
From the Euclidean assumption of space it follows that
\begin{align*}
OP&=OO'+O'P=\mathbf{e}\, o+\mathbf{e}'x'=\mathbf{e}x
\end{align*}
Since $\mathbf{e}$ and $\mathbf{e}'$ are parallel to each other we have that the two representations 
$x$ and $x'$ are related by
\begin{align}
x=o+x'.\label{eq:ParallelFramesPosition}
\end{align}
However, what we are really interested in is the relationship between $x$ and $X$. 

Observe that one can represent each of the axis $\mathbf{b}_i$ using the $\mathbf{e}'$ frame as follows.
\begin{align*}
\mathbf{b}_1= r_{11}\mathbf{e'}_1+r_{21}\mathbf{e'}_2+r_{31}\mathbf{e'}_3,\\
\mathbf{b}_2= r_{12}\mathbf{e'}_1+r_{22}\mathbf{e'}_2+r_{32}\mathbf{e'}_3,\\
\mathbf{b}_3= r_{13}\mathbf{e'}_1+r_{23}\mathbf{e'}_2+r_{33}\mathbf{e'}_3.
\end{align*}
This can be expressed in the matrix form
\begin{align*}
\underbrace{[\mathbf{b}_1\:\:\: \mathbf{b}_2\:\:\: \mathbf{b}_3]}_{\mathbf{b}}=\underbrace{[\mathbf{e'}_1\:\:\: \mathbf{e'}_2\:\:\: \mathbf{e'}_3]}_{\mathbf{e}'}\underbrace{\begin{bmatrix}r_{11}&r_{12}&r_{13}\\
r_{21}&r_{22}&r_{23}\\
r_{31}&r_{32}&r_{33}\\
\end{bmatrix}}_{R}.
\end{align*}
It can be shown that the assumption of Euclidean 3D-space implies that $\mathbf{b}=\mathbf{e}'\,R$ where $R$ is a $3\times 3$  special orthogonal matrix\footnote{In exercise-\ref{ex:RotationMatrix} you are asked to prove this. The space of all $3\times 3$ special orthogonal matrices is denoted by $\mathrm{SO}(3)$.
Recall that a special orthogonal matrix $R$ satisfies $RR^T=R^TR=I_{3\times 3}$ and $\det{(R)}=1$.}.
Since $O'P=\mathbf{e}'x'=\mathbf{b}\,X=\mathbf{e}'\,RX$ we have that 
\begin{align}
x'&=RX.\label{eq:RotatedFramesPosition}
\end{align}
\begin{svgraybox}
 Thus we have that the two 
representations of the point $P$ in the two frames $\mathbf{e}$ and $\mathbf{b}$ as depicted in figure-\ref{Fig:GeneralMovingFrame0} are related by
\begin{align}
x&=o+RX.\label{eq:Moving2FixedGeneral0}
\end{align}
\end{svgraybox}
The above expression can also be expressed in matrix form as:
\begin{align}
\left[\begin{array}{c}x \\ 1
\end{array}\right]&=\left[\begin{array}{cc}
R & o \\ 0 & 1\end{array}\right]
\left[\begin{array}{c} X \\ 1
\end{array}\right].
\end{align}
This expression will be very useful when one deals with kinematic chains such as robot arms.
\begin{svgraybox}
Notice that the preceding discussion shows that the relationship between two right hand oriented orthonormal frames $\mathbf{e}$ and $\mathbf{b}$ is uniquely given by a $(o,R)\in\mathbb{R}\times SO(3)$ and that conversely any $(o,R)\in\mathbb{R}\times SO(3)$ defines the relationship between two right hand oriented orthonormal frames $\mathbf{e}$ and $\mathbf{b}$ in a unique fashion.
\end{svgraybox}
%%%%%%%%%%%%%%%%%%%%%%%%%%%%%%%%%%%


\subsection{Kinematics in  Moving Frames}
In general, since the position representation are different from frame to frame, the velocities and accelerations expressed in one frame will be different form those expressed with 
respect to another. Thus it is important to always specify the orthonormal frame with which they are expressed. Consider the problem of describing the motion of a point $P(t)$ that is moving with respect to both frames $\mathbf{e}$ and $\mathbf{b}$ and let $\mathbf{b}$ be translating and rotating with respect to $\mathbf{e}$. Thus we have that all matrices $x(t),X(t)$ and $o(t),R(t)$ are changing with respect to time.  Differentiating the expression (\ref{eq:Moving2FixedGeneral0}) relating the two position representations we find that the two velocities $\dot{x}$ and $\dot{X}$ measured in the two frames are related by
\[
\dot{x}=\dot{o}+\dot{R}X+R\dot{X}.
\]
Similarly the two accelerations measured in the two frames $\ddot{x}$ and $\ddot{X}$ are related by
\[
\ddot{x}=\ddot{o}+\ddot{R}X+2\dot{R}\dot{X}+R\ddot{X}.
\]

In the following we proceed to find if $\dot{R}$ and $\ddot{R}$ can expressed a little more conveniently.
In exercise-\ref{ex:RotationMatrix} we have seen that $R^T(t)R(t)=I$. Thus it follows that $\dot{R}^TR+R^T\dot{R}=0$,
and hence that
\[
R^T\dot{R}=-(R^T\dot{R})^T=\widehat{\Omega},
\]
where $\widehat{\Omega}$ is a skew symmetric matrix. The space of all $3\times 3$ skew-symmetric matrices is denoted by $\mathrm{so}(3)$. 
\begin{svgraybox}
Thus we have that if $R(t)\in \mathrm{SO}(3)$ then
\begin{align}
\dot{R}&=R\widehat{\Omega},\label{eq:RdotEqn}
\end{align}
where  $\widehat{\Omega}(t)\in \mathrm{so}(3)$.
\end{svgraybox}
Differentiating it twice we have $\ddot{R}=R(\widehat{\Omega}^2+\dot{\widehat{\Omega}})$.
Substituting these expressions for $\dot{R}$ and $\ddot{R}$ in the above expressions relating velocities and accelerations in the two frames $\mathbf{e}$ and $\mathbf{b}$ we have,
\begin{align}
\dot{x}&=\dot{o}+R\left(\widehat{\Omega} X+\dot{X}\right),\label{eq:VelocityMoving2Fixed2}\\
\ddot{x}&= \ddot{o}+R\left(\widehat{\Omega}^2(t) X+2\,\widehat{\Omega}\dot{X}+\dot{\widehat{\Omega}}X+\ddot{X}\right).\label{eq:InertialAccInBodyFrame}
\end{align}

\begin{figure}[ht]
\begin{center}
\begin{tabular}{ccc}
\includegraphics[width=2in]{TwoDRotatn1V2} & \includegraphics[width=2in]{TwoDRotatn2V2} & \includegraphics[width=2in]{TwoDRotatn3V2}\\
Rotation about $\mathbf{e}_1$ & Rotation about $\mathbf{e}_2$ & Rotation about $\mathbf{e}_3$ 
\end{tabular}
\renewcommand{\baselinestretch}{1}\selectfont
\caption{Rotated Frames}
\label{Fig:TwoDRotatnFrames}
\renewcommand{\baselinestretch}{1.5}\selectfont
\end{center}
\end{figure}



Thus we see that in order to find these relationships one needs to compute the matrices $\widehat{\Omega},\widehat{\Omega}^2$ and $\dot{\widehat{\Omega}}$. 
As an illustration let us consider, easy to visualize, three special frame rotations. We will also see that these three special type of rotating frames will become useful when representing the motion of complicated systems as well.
Consider the three rotating orthonormal frames $\mathbf{a},\mathbf{b},\mathbf{c}$ that are related to a fixed frame $\mathbf{e}$ as shown in figure \ref{Fig:TwoDRotatnFrames}. Each of the frames $\mathbf{a},\mathbf{b},\mathbf{c}$ correspond to a simple counter clockwise rotation about the $i^{\mathrm{th}}$ axis of $\mathbf{e}$ by an angle equal to $\theta_i(t)$. Let $\mathbf{a}=\mathbf{e}\, R_1{(\theta_1)}$, $\mathbf{b}=\mathbf{e} \,R_2{(\theta_2)}$,
and $\mathbf{c}=\mathbf{e}\, R_3{(\theta_3)}$. In exercise-\ref{ex:RotatedFrames} you are asked to show using direct calculations that the following expressions hold.


{\small
\begin{align}
R_1{(\theta_1)}=\left[\begin{array}{ccc}
1 & 0 & 0\\
0 & \cos{\theta_1} & -\sin{\theta_1}\\
0 & \sin{\theta_1} & \cos{\theta_1}
\end{array}\right],\:\:\:\:
R_2{(\theta_2)}=\left[\begin{array}{ccc}
\cos{\theta_2} & 0 & \sin{\theta_2}\\
0 & 1 & 0\\
- \sin{\theta_2}& 0 & \cos{\theta_2}
\end{array}\right],\:\:\:\:
R_3{(\theta_3)}=\left[\begin{array}{ccc}
\cos{\theta_3} & -\sin{\theta_3} & 0\\
\sin{\theta_3} & \cos{\theta_3} &0\\
0 & 0 & 1
\end{array}\right],\label{eq:RotatedFrames}
\end{align}
}
and
\[
R_1^T\dot{R}_1=\widehat{\Omega}_1=\left[\begin{array}{ccc}
0 & 0 & 0\\
0 & 0 & -\dot{\theta_1}\\
0 & \dot{\theta_1} & 0
\end{array}\right],\:\:\:\:
R_2^T\dot{R}_2=\widehat{\Omega}_2=\left[\begin{array}{ccc}
0 & 0 & \dot{\theta_2}\\
0 & 0 & 0\\
- \dot{\theta_2}& 0 & 0
\end{array}\right],\:\:\:\:
R_3^T\dot{R}_3=\widehat{\Omega}_3=\left[\begin{array}{ccc}
0 & -\dot{\theta}_1 & 0\\
\dot{\theta}_1 & 0 &0\\
0 & 0 & 0
\end{array}\right]
\]
and
\[
\widehat{\Omega}_1^2=-\dot{\theta}_1^2\left[\begin{array}{ccc}
0 & 0 & 0\\
0 & 1 & 0\\
0 & 0 & 1
\end{array}\right],\:\:\:\:
\widehat{\Omega}_2^2=-\dot{\theta}_2^2\left[\begin{array}{ccc}
1 & 0 & 0\\
0 & 0 & 0\\
0 & 0 & 1
\end{array}\right],\:\:\:\:
\widehat{\Omega}_3^2=-\dot{\theta}_3^2\left[\begin{array}{ccc}
1 & 0 & 0\\
0 & 1 & 0\\
0 & 0 & 0
\end{array}\right].
\]

Having seen how to calculate $\widehat{\Omega}$ and $\widehat{\Omega}^2$ and noticing that they have a pattern we may ask what general properties the $3\times 3$ skew-symmetric matrices have. In Section-\ref{Secn:PropertiesOfRotations} we will investigate in detail several properties of $3\times 3$ special orthogonal matrices and $3\times 3$ skew-symmetric matrices in order to facilitate these computations on one hand and on the other hand 
to get a deeper understanding of the physical meaning of $R\in \mathrm{SO}(3)$ and $\widehat{\Omega}\in \mathrm{so}(3)$.
%%%%%%%%%%%%%%%%%%%%%%%%%%%%%%%%%%%%%%%%


\subsection{Infinitesimal Rotations and Angular Velocity}\label{Secn:PropertiesOfRotations}
In this section we will take a closer look at the physical meaning of the skew symmetric matrix $\widehat{\Omega}=R^T\dot{R}$. To do so we will have to first obtain a better understanding of special orthogonal matrices. A given special orthogonal matrix $R\in \mathrm{SO}(3)$ can be viewed in at least three different ways. We have seen before that the relationship between two right hand oriented orthonormal frames with coinciding origin is uniquely determined by a special orthogonal matrix and that conversely every special orthogonal matrix uniquely defines a relationship between two such frames. Below we will see that there are two other ways of looking at a special orthogonal matrix. In one respect it can be seen as a coordinate transformation while in another respect we can view it as an action on 3-dimensional Euclidean space by rigid rotations.

First to see how it represents a coordinate transformation consider the expression 
(\ref{eq:RotatedFramesPosition}) a bit more closely. 
What this says is that $R$ can be thought of as a coordinate transformation that relates the $\mathbf{e}'$-frame coordinates of the point $P$, given by the matrix $x'$ to the $\mathbf{b}$-frame coordinates of the point $P$, given by the matrix $X$. This idea can be extended to any intrinsic property\footnote{A property that does not depend on the choice of coordinates used to represent it is referred to as an intrinsic property.} of the particle such as velocity, momentum, or force, that can be considered as a \textit{arrow in space with the foot coinciding with $O'$\footnote{Also referred to as a directed line segment. This is what you would have traditionally learnt as vectors at the secondary school level.}} in the following manner. 
\begin{svgraybox}
Let $\mathbf{e}$ and $\mathbf{b}$ be two orthonormal frames with coinciding origin and let $\mathbf{b}=\mathbf{e}\,R$ for some $R\in SO(3)$.
If \textit{gamma} is some intrinsic property we may represent it by a point $G$ with representation $\gamma$ in the $\mathbf{e}$-frame or with $\Gamma$ in the $\mathbf{b}$-frame. Then we have from (\ref{eq:RotatedFramesPosition}) that the two representations of the intrinsic quantity are related by
\begin{align}
\gamma&=R\Gamma.\label{eq:IntrinsicProperty}
\end{align}
In fact, insisting that this relationship holds can be taken to be the meaning of being intrinsic or coordinate independent.
\end{svgraybox}







\begin{figure}[ht]
\begin{center}
\begin{tabular}{c}
\includegraphics[width=4in]{RotatingBox4.png} 
\end{tabular}
\renewcommand{\baselinestretch}{1}\selectfont
\caption{The $\mathbf{b}$-frame is fixed on the box and the $\mathbf{e}$-frame is some `fixed' reference frame.}
\label{Fig:RotatingBox}
\renewcommand{\baselinestretch}{1.5}\selectfont
\end{center}
\end{figure}

On the other hand a given $R\in \mathrm{SO}(3)$ 
can be viewed as a \textit{map that acts} on a point $P$ in space to give a new point $P_R$ in the following manner. Let $\mathbf{e}$ be some `fixed' frame and let $x$ be the representation of $P$ in the $\mathbf{e}$ frame. 
Let $P_R$ be the point in space that has the representation $Rx$ in the $\mathbf{e}$-frame. That is let $OP=\mathbf{e}\, x$ and $OP_R=\mathbf{e}\,(R x)$.
This allows one to  consider $R\in \mathrm{SO}(3)$ as a transformation that  takes $P$ to a new point $P_R$ in space by mapping $x \to Rx$ and identifying $P_R$ with the point in space that has the representation $Rx$ in the $\mathbf{e}$-frame. Let $Q$ be another point in space that has the representation $y$ with respect to the $\mathbf{e}$-frame. Then since $R^TR=RR^T=I_{3\times 3}$ we see that $||Rx-Ry||=||x-y||$ and $\langle\langle Rx,Ry\rangle\rangle=\langle\langle x,y\rangle\rangle$  and hence that this map preserves lengths and angles in space. Such maps that transform points in space to other points in space in such a way that it preserves distances between points and angles between lines are called \textit{isometries}. Let us apply this map to all points in space and see how they transform by considering an illustration.
Consider the set of points defined by a cube in space as shown in the left hand side of the figure-\ref{Fig:RotatingBox}. The cube is chosen such that the point $P$ coincides with the vertex of this cube that is diagonally opposite the vertex at the centre of the frame $O$ as shown in the left hand side of figure-\ref{Fig:RotatingBox}. 
Since the map that takes $x \to Rx$ preserves lengths and angles in space we see that when the points defining the cube are transformed by $R$ to the new points, using the above recipe,  the new transformed points will also correspond to a cube that is identical to the initial cube with the exception of it now being `rotated' about the vertex $O$. This situation is shown in the right hand side of figure-\ref{Fig:RotatingBox}. Let $\mathbf{b}$ be a frame such that it is fixed with respect to the cube such that initially both $\mathbf{b}$ and $\mathbf{e}$ coincide. It is now easy to see that the new orientation of the frame $\mathbf{b}$ fixed to the cube is related to the frame $\mathbf{e}$ by the relationship $\mathbf{b}=\mathbf{e}R$. 
\begin{svgraybox}
Thus we see that a given $R\in \mathrm{SO}(3)$ can be uniquely identified with a `rigid rotation' of space and conversely that every rigid rotation of space about a fixed point can be identified with an $R\in \mathrm{SO}(3)$. This also shows that the configuration of a \textit{rigid body} moving such that one of its points remains fixed in space can be uniquely identified with an $R\in \mathrm{SO}(3)$.
\end{svgraybox}

%%%%%%%%%%%%%%%%%%%%%%
%\subsection{Angular Velocity}\label{Secn:AngularVelocity}
Let us now revert our attention to the $3\times 3$ skew symmetric matrix $\widehat{\Omega}=R^T\dot{R}$. We will see that it can be interpreted as the angular velocity of the frame $\mathbf{b}$ about $\mathbf{e}$. To do so we will first need to be familiar with several properties of the space of $3\times 3$ skew symmetric matrices, $\mathrm{so}(3)$.

It is straightforward to see that $\mathrm{so}(3)$ is a three dimensional real vector space under matrix addition and scalar multiplication\footnote{See exercise  \ref{ex:SkewNCross}.}. Thus it is isomorphic\footnote{An isomorphism is a continuous one-to-one and onto map where the inverse is also continuous.} to $\mathbb{R}^3$. That is, there is a one-to-one and onto correspondence between elements of $\mathbb{R}^3$ and elements of $\mathrm{so}(3)$. The isomorphism $\:\:\:\widehat{}\:\:: \mathbb{R}^3\to \mathrm{so}(3)$ that is explicitly defined by,
\begin{equation}\label{eq:SkewSymmetric0}
\widehat{\Omega}=\left[ \begin{array}{ccc} 0 & -\Omega_3 & \Omega_2 \\ \Omega_3 & 0 & -\Omega_1 \\ -\Omega_2 & \Omega_1 & 0\end{array}\right],
\end{equation}
for $\Omega =(\Omega_1,\Omega_2,\Omega_3)\in \mathbb{R}^3$ gives one such identification. It is easy to verify that $\:\:\:\widehat{}\:\:: \mathbb{R}^3\to \mathrm{so}(3)$ is linear. That is $\widehat{X+Y}=\widehat{X}+\widehat{Y}$ and $\widehat{\alpha X}=\alpha \widehat{X}$ for any $X,Y\in \mathbb{R}^3$ and $\alpha\in \mathbb{R}$.
It is also easy to directly verify that this isomorphism satisfies
\begin{align}
\widehat{\Omega}X&=\Omega \times X,\label{eq:HatNCross}\\
\langle\langle X,Y\rangle\rangle &=-\frac{1}{2}\mathrm{trace}{(\widehat{X}\widehat{Y})},
\end{align}
for $X,Y,\Omega\in \mathbb{R}^3$.

In exercises \ref{ex:SkewNCross} -- \ref{ex:XhatSqrd} you are asked to show the following very useful and  interesting properties of $3\times 3$ skew-symmetric matrices:
\begin{align}
\widehat{RX}&=R\widehat{X}R^T,\label{eq:AdjointAction}\\
\widehat{X}^2&= XX^T-||X||^2I_{3\times3}. \label{eq:OmegaHatSqrd}
\end{align}
for any $X\in \mathbb{R}^3$ and $R\in \mathrm{SO}(3)$.
%%%%%%%%%%%%%%%%%%%%%%%%%%%%%%%%


Let us now consider smooth rotations that are parameterised by a parameter $t$ that we may consider to be time. Let $R(t)$ be a smooth curve in the space $\mathrm{SO}(3)$ such that $R(0)=I_{3\times 3}$. Then from the above discussion we see that $R(t)$ represents a smooth rigid rotation of space for all $t$. Let $\mathbf{b}(t)=\mathbf{e}R(t)$ be the corresponding rotating frame. Since $R(0)=I_{3\times 3}$ we see that $\mathbf{b}(0)=\mathbf{e}$. Let $P(t)$ be a point in space that corresponds to $P(0)$ being `rotated' by $R(t)$. That is if $X$ is the representation of the point $P(0)$ in the $\mathbf{e}=\mathbf{b}(0)$ frame then $R(t)X$ is the representation of the point $P(t)$ in the 
$\mathbf{e}$-frame.
 Thus since rotations by $R$ preserve angles and lengths in space, $P(t)$ will appear to be fixed as viewed in the $\mathbf{b}(t)$ frame and will be equal to $X$. That is, the representation $X$ of the point $P(t)$ in the $\mathbf{b}(t)$ frame will not depend on $t$. Let $x(t)$ be the representation of the point $P(t)$ in the $\mathbf{e}$-frame. Then $x(t)=R(t)X$.
The velocity of the point $P(t)$ in the $\mathbf{e}$-frame is thus given by $\dot{x}(t)=\dot{R}(t)X$.
Previously we have seen that $R^T(t)\dot{R}(t)=\widehat{\Omega}(t)$ is always a skew-symmetric matrix. Thus we have that the velocity of the point $P(t)$ as expressed in the $\mathbf{e}$-frame has the representation
$\dot{x}(t)={R}(t)\widehat{\Omega}(t)X$.
Therefore from (\ref{eq:HatNCross}) we see that the velocity of the point $P(t)$ can be expressed in the $\mathbf{e}$-frame as 
\begin{align}
\dot{x}&=\dot{R}X=R\widehat{\Omega}X=R(\Omega \times X)=(R\Omega) \times (RX)=(R\Omega) \times x
=\omega\times x,\label{eq:Dotx}
\end{align}
where we have used the property $R(X\times Y)=RX\times RY$ and have set $\omega\triangleq R\Omega$ in the last equality. Notice that $\omega$ is the $\mathbf{e}$-frame representation of the quantity that has the representation $\Omega$ in the $\mathbf{b}$-frame. Also notice that $||\omega||=||R\Omega||=||\Omega||$.  


Since $\omega(t) \times \omega(t)=0$ we see that all points in space that lie along the direction $\omega(t)$ as viewed in the $\mathbf{e}$-frame have zero velocity when $R(t)$ acts on them by a `rotation'.
\begin{figure}[ht]
\begin{center}
\begin{tabular}{c}
\includegraphics[width=3in]{AngularVelocity.png} 
\end{tabular}
\renewcommand{\baselinestretch}{1}\selectfont
\caption{The meaning of angular velocity.}
\label{Fig:AngularVelocity}
\renewcommand{\baselinestretch}{1.5}\selectfont
\end{center}
\end{figure}
On the other hand by the definition of the cross product in $\mathbb{R}^3$ and the last equality of the expression (\ref{eq:Dotx}) we have
\[
\dot{x}=||\omega||\,||x||\,\sin{\theta}\,\mathbf{n} \,
\]
where $\theta$ is the angle between $\omega$ and $x$ in the $\mathbf{e}$-frame as shown in figure-\ref{Fig:AngularVelocity} and $\mathbf{n}=\omega/||\omega||$ is an orthonormal direction segment that is both mutually perpendicular to the direction given $\omega=R\Omega$ in the $\mathbf{e}$-frame and $OP$. 
Thus we see that $P(t)$ is instantaneously rotating about $\omega(t)$ as viewed in the $\mathbf{e}$-frame. Since $X$ was arbitrary we see that this is true for every point in space. Which shows that under the `rotation' by $R(t)$ every point in space is instantaneously rotating about $\omega$ with an angular rate of rotation equal to $||\omega||=||\Omega||$ as viewed in the frame $\mathbf{e}$. 
\begin{svgraybox}
The above discussion motivates one to define $\omega(t)=R(t)\Omega(t)$ to be the \textit{angular velocity} of the frame $\mathbf{b}$ with respect to the frame $\mathbf{e}$ represented in the $\mathbf{e}$-frame. We will call it the \textit{spatial angular velocity} of the frame $\mathbf{b}$ with respect to $\mathbf{e}$ and since $\Omega$ is its $\mathbf{b}$-frame representation, we will call $\Omega$ the \textit{body angular velocity} of the frame $\mathbf{b}$ with respect to $\mathbf{e}$.
\end{svgraybox}


%%%%%%%%%%%%%%%%%%%%%%%%%%%%%%%%%%


%%%%%%%%%%%%%%%%%%%%%%%%%%%%%%%%

\subsection{Angular Momentum in Moving Frames}\label{Secn:AngularMomentumMF}
We observe that the angular momentum $\pi_i$ of a particle $P_i$ about the origin of the moving $\mathbf{b}$-frame, $O'$, can be expressed as 
\begin{align*}
\pi_i&= m_i(x_i-o)\times \dot{x}_i=m_iR\left(X_i\times (\Omega\times X_i+\dot{X}_i+R^T\dot{o})\right),\\
&=R\left(-m_i\widehat{X}_i^2\Omega+m_iX_i\times( R^T\dot{o}+ \dot{X}_i)\right),
\end{align*}
where the last equality follows from
\begin{align*}
 X_i\times \Omega\times X_i&=-X_i\times X_i \times \Omega= -\widehat{X}_i^2\Omega.
 \end{align*}
\begin{svgraybox}
The quantity
\begin{align}
\mathbb{I}_i &\triangleq -m_i\widehat{X}_i^2=m_i\left(||X_i||^2I_{3\times 3} - X_iX_i^T\right),\label{eq:MomentOfInertiaP}
\end{align} 
is defined as the \textit{moment of inertia} of the particle $P$ about the point $O'$ in the frame $\mathbf{b}(t)$.
Using this we can now express the angular momentum of $P$ about $O'$ as 
\begin{align}
\pi_i&= R\left(\mathbb{I}_i\Omega+m_iX_i\times( R^T\dot{o}+ \dot{X}_i)\right).\label{eq:BodyPi}
\end{align}
The above expression shows that 
\begin{align}
\Pi_i\triangleq \left(\mathbb{I}_i\Omega+m_iX_i\times(\dot{X}_i+ R^T\dot{o})\right),
\end{align}
is the moving $\mathbf{b}$-frame representation of the angular momentum of $P_i$ about $O'$.
\end{svgraybox}

In what follows we consider the case where the particle $P_i$ appears fixed in the moving frame $\mathbf{b}$. That is when $\dot{X}_i=0$. In this case, differentiating (\ref{eq:BodyPi}) and using the Jacobi property of cross products, 
\begin{align*}
A\times B\times C+B\times C\times A+C\times A\times B=0
\end{align*} 
we find that
{\small
\begin{align*}
\dot{\pi}_i&=R\left(\mathbb{I}_i\dot{\Omega}-\mathbb{I}_i{\Omega}\times \Omega+-m_i\,(R^T\dot{o})\times\Omega\times X_i+m_iX_i\times R^T\ddot{o}\right).
\end{align*}
}
On the other hand we have from (\ref{eq:ChangeAngMomentum}) that 
\begin{align*}
\dot{\pi}_i&= R\left(-m_i(R^T\dot{o})\times \Omega \times {X}_i+X_i\times F_i\right),
\end{align*}
where $F_i=R^Tf_i$ is the representation of the force acting on the particle $p$ in the $\mathbf{b}$-frame and since $X_i\times F_i=R^T((x_i-o)\times f_i)$ the quantity $T_i=X_i\times F_i$ is the representation of the moment of the force acting on $p$ about the point $o$ with respect to the moving frame $\mathbf{b}(t)$. Thus combining the last two expressions we have

\begin{align}
\mathbb{I}_i\dot{\Omega}=\mathbb{I}_i\Omega \times \Omega -m_iX_i\times R^T\ddot{o}+X_i\times F_i.\label{eq:RateOfChangeBodyPi}
\end{align}

\begin{svgraybox}
In summary in the case where the particle $P_i$ is fixed with respect to the frame $\mathbf{b}$ we have that 
\begin{align*}
\pi_i&= R\left(\mathbb{I}_i\Omega+m_iX_i\times R^T\dot{o}\right),\\
%\dot{\pi}_i&= R\left(-m_i(R^T\dot{o})\times \Omega \times {X}_i+X_i\times F_i\right),\\
%\dot{\pi}_i&=R\left(\mathbb{I}_i\dot{\Omega}-\mathbb{I}_i{\Omega}\times \Omega-m_i\,(R^T\dot{o})\times\Omega\times X_i+m_iX_i\times R^T\ddot{o}\right),\\
\mathbb{I}_i\dot{\Omega}&=\mathbb{I}_i\Omega \times \Omega -m_iX_i\times R^T\ddot{o}+X_i\times F_i
\end{align*}
\end{svgraybox}
We will see that the last expression above will play a crucial role in deriving Euler's rigid body equations of motion.
%%%%%%%%%%%%%%%%%%%%

\subsection{Kinetic Energy in Moving Frames}\label{Secn:KineticEnergyMF}
Recall that the kinetic energy $KE_i$ of a particle of mass $m_i$ is defined with respect to an \textit{inertial frame} $\mathbf{e}$ and is given by the relationship
\begin{equation*}
\mathrm{KE}_i \triangleq \frac{1}{2}m_i ||\dot{x}_i(t)||^2,
\end{equation*}
where $||\cdot||$ is the Euclidian norm in $\mathbb{R}^3$. 
Since $||R X||=||X||$ we also have that the kinetic energy of the particle can also be expressed as 
\begin{align*}
\mathrm{KE}_i&=\frac{1}{2}m ||\dot{x}_i||^2=\frac{1}{2}m ||R^T \dot{x}_i||^2=\frac{1}{2}m ||R^T\dot{o}+\widehat{\Omega} X_i+\dot{X}_i||^2,\\
&=\frac{1}{2}m \left( ||\dot{o}||^2+2\dot{o}^TR(\widehat{\Omega} X_i+\dot{X}_i)+||\widehat{\Omega} X_i||^2+2\dot{X}_i^T\widehat{\Omega}X_i+||\dot{X}_i||^2\right).
\end{align*}
Note that $||\widehat{\Omega}X_i||^2=||\widehat{X}_i\Omega||^2=-\Omega^T\widehat{X}_i^2\Omega$. Using the substitution $\mathbb{I}_i=-m_i\widehat{X}_i^2=m_i(||X_i||^2I_{3\times 3}-X_iX_i^T)$ the kinetic energy of the particle can be expressed as
\begin{equation}\label{eq:KineticEnergyOfaParticleMoving}
\mathrm{KE}_i=\frac{1}{2}\left( m_i||\dot{o}||^2+2m_i\dot{o}^TR(\widehat{\Omega} X_i+\dot{X}_i)+\Omega^T\mathbb{I}_i\Omega+2m_i\dot{X}_i^T\widehat{\Omega}X_i+m_i||\dot{X}_i||^2\right).
\end{equation}
\begin{svgraybox}
If the particle $P$ is fixed with respect to the moving frame $\mathbf{b}$ then ${X}_i=\mathrm{constant}$ and hence
\begin{align}
\mathrm{KE}_i&=\frac{1}{2} \left( m_i||\dot{o}||^2+2m_i\dot{o}^TR({\Omega} \times X_i)+\Omega^T\mathbb{I}_i\Omega
\right),\nonumber\\
&=\frac{1}{2} \left( m_i||V_{o}||^2+2m_iV_o({\Omega} \times X_i)+\Omega^T\mathbb{I}_i\Omega
\right),\label{eq:KineticEnergyOfaParticleFixedInRotFrame}
\end{align}
where we have used the identity $V_o=R^T\dot{o}$ in the last expression.
\end{svgraybox}



%%%%%%%%%%%%%%%%%%%%%%%%%%%%
%%%%%%%%%%%%%%%%%%%%%%%%%%%%%%%%%%

\subsection{Newton's Law in Moving Frames}
Recall that Galilean laws of mechanics states that the total linear momentum of a set of interacting but otherwise isolated set of particles is conserved when observed in any inertial frame. Thus Newton's second law hold only in inertial frames. Let $\mathbf{e}$ be an inertial frame and let the representation of the position of a particle $m$ in the $\mathbf{e}$-frame be $x$. Let the force acting on the particle due to the interaction it has with the other particles of the Universe have the representation $f$ in the $\mathbf{e}$-frame. Then Newton's second law gives $f=m\ddot{x}$. If an observer makes measurements with respect to a different frame $\mathbf{b}$ that is translating and rotating with respect to $\mathbf{e}$ then it is natural to ask what Newton's second law looks like with respect to the measurements made with respect to the 
$\mathbf{b}$-frame. 

 
At the end of section-\ref{Secn:RelativeMotion} we see that the acceleration of the particle in the 
$\mathbf{e}$-frame is related to the $\mathbf{b}$-frame quantities by (\ref{eq:InertialAccInBodyFrame}). 
Let $F$ be the representation of the force acting on the particle in the $\mathbf{b}$-frame. That is let $F=R^Tf$. Then we have the following:
\begin{svgraybox}
Let $\mathbf{e}$ be an inertial frame and let $\mathbf{b}$ be a translating and rotating frame a shown in figure-\ref{Fig:GeneralMovingFrame0}. Denote the representation of the origin of the $\mathbf{b}$-frame with respect to the $\mathbf{e}$-frame be $o$ and let $R\in \mathrm{SO}(3)$ be the rotation matrix that relates the $\mathbf{b}$ frame to the $\mathbf{e}$ by the relationship $\mathbf{b}=\mathbf{e}\,R$.  Let
the representation of the position of a particle $m$ in the $\mathbf{e}$ frame be $x$ while let its representation in the $\mathbf{b}$-frame be $X$ and the force acting on the particle due to the interaction it has with the other particles of the Universe have the representation $f$ in the $\mathbf{e}$-frame.  Newton's second law expressed using the moving $\mathbf{b}$-frame quantities are 
\begin{align}
F&= mR^T\ddot{o}+m\widehat{\Omega}^2 X+2m\,\widehat{\Omega}\dot{X}+m\dot{\widehat{\Omega}}X+m\ddot{X},\label{eq:NewtonInMovingFrames}
\end{align}
where $F=R^Tf$ is the representation of the physical force acting on the particle in the $\mathbf{b}$-frame.
\end{svgraybox}

Notice that this equation is completely  expressed using only the $\mathbf{b}(t)$-frame representation of the force given by $F(t)$, the skew-symmetric matrix $\widehat{\Omega}=R^T\dot{R}$,  the position given by $X(t)$ and the derivatives of the position $\dot{X}$ and $\ddot{X}$. Thus this expression, \text{if you may}, can be considered to be the `appropriate version' of the Newton's equations in the rotating frame $\mathbf{b}(t)$.

Imagine the situation where the observer is unaware of the motion of its frame-$\mathbf{b}$ and thinks of it as an inertial frame\footnote{Like for instance when we think of an earth fixed frame to be inertial.}. Then the observer, having taken a mechanics class during her undergraduate program, will interpret mass times acceleration measured in her reference frame $\mathbf{b}$ to be the force felt in $\mathbf{b}$. That is, she will think that
\begin{align}
m\ddot{X}&=F-\left(mR^T\ddot{o}+m\widehat{\Omega}^2 X+2m\,\widehat{\Omega}\dot{X}+m\dot{\widehat{\Omega}}X\right),
\end{align}
is the force acting on the particle as expressed in her frame $\mathbf{b}$.
However the quantity $F(t)=R^T(t)f(t)$ is the only physically meaningful force that she feels.
Thus an observer moving with the rotating frame will, in addition to the fundamental interacting forces, feel that there exists another resultant `\textit{apparent force}':
\begin{align}
F_{app} &\triangleq  
\underbrace{-m\;R^T(t)\ddot{o}(t)}_{\mbox{\it Einstein}}\:\underbrace{-m\;\widehat{\Omega}^2(t) X(t)}_{\mbox{\it Centrifugal}}\:\underbrace{-\:2m\;\widehat{\Omega}(t)\dot{X}(t)}_{\mbox{\it Coriolis}}\:\underbrace{- \:m\;\dot{\widehat{\Omega}}(t) X(t)}_{\mbox{\it Euler}},\label{eq:AppearentForcesMoving}
\end{align}
simply due to its ignorance of the motion of its frame.
The first term $-m\;R^T(t)\ddot{o}(t)$ is known as the \textit{Einstein force}, the second term $-m\;\widehat{\Omega}^2(t) X(t)$ is known as the \textit{Centrifugal} force, the third term $-2m\;\widehat{\Omega}(t)\dot{X}(t)$ is known as the \textit{Coriolis} force and the last term 
$- m\;\dot{\widehat{\Omega}}(t) X(t)$ is known as the \textit{Euler} force. Observe that all these apparent forces have mass as a multiplicative factor. 
Note that the Einstein apparent force is observed due to the \textit{translational ignorance} of the one's reference frame while the Centrifugal, Coriolis, and Euler forces are observed due to the \textit{rotational ignorance} of the reference frame.

Using these expression we can explain many physical effects. In exercise \ref{ex:KinematicsNKineticsInElevator} you are asked to explain why a person standing on a scale 
inside an elevator sees his or her weight doubled as the elevator accelerates up at a rate of $g$ and sees the weight reduced to zero if the elevator decelerates at a rate of $g$ 
where $g$ is the gravitational acceleration. You are also asked to show that if, for some reason, the gravitational force field vanished and that the elevator was moving up at an 
acceleration of $g$ then the scale would show the correct weight of the person. This last observation shows that a person inside the elevator can not distinguish between the 
following two cases:
\begin{description}
\item[a.)] Gravity is present and the elevator is standing still (or moving at constant velocity).
\item[b.)] Gravity is absent and the elevator is accelerating upwards at a rate of $g$.
\end{description}
It is this observation that led Einstein to the conclusions of General Relativity and in particular that gravity is an apparent force !!!


%%%%%%%%%%%%%%%%%%%%%%%%%%%%%%%%%%%
In the following sections we will demonstrate the value of equation (\ref{eq:NewtonInMovingFrames}) in writing down the equations of motion. In particular in section-\ref{Secn:PolarCoords} we will see how to describe the motion of a particle constrained to move in two dimensions using polar coordinates and in section-\ref{Secn:EffectOfEarthRotation} we will use it to explain some of the apparent effects of Earth's rotation. 

\begin{svgraybox}
From an applications point of view the use of (\ref{eq:NewtonInMovingFrames}) in predicting the motion of objects moving under complicated geometric constraints is invaluable since in such a case representing position, velocity and the fundamental constraint force interactions is mostly convenient in a frame where the object appears fixed. Then Newton's equation (\ref{eq:NewtonInMovingFrames}) in the frame where the object appears fixed reduce to
\begin{align}
F&= mR^T\ddot{o}+m(\widehat{\Omega}^2 +\dot{\widehat{\Omega}})X.\label{eq:NewtonInBodyFixedFrames}
\end{align}
In this case what remains is the computation that relates the frame in which the object appears fixed to an inertial frame; namely $R$ and $\widehat{\Omega}=R^T\dot{R}$.
\end{svgraybox}
In section-\ref{Secn:BeadOnRotatingHoop} we show an example of this approach in describing the motion of a bead constrained to move on a rotating hoop. We also invite you to try this approach in describing similarly constrained motion that is described in exercises \ref{Ex:RotatingDiskMassInSlot}-\ref{ex:MEMSgyro}.

%%%%%%
\subsubsection{Description of Particle Motion in a Plane using Polar Coordinates}\label{Secn:PolarCoords}
For a particle restricted to move in 2-dimensions, it is sometimes convenient to write down the motion in polar coordinates $(r,\theta)$. This amounts to 
observing the motion in a moving frame  $\mathbf{b}(t)=[\mathbf{e}_r(t)\:\:\: \mathbf{e}_{\theta}(t)\:\:\: \mathbf{e}_{z}(t)]$ (refer to figure \ref{Fig:RotatingFrameWithBody}) where
$\mathbf{e}_r$ aligns along the particle $P$ at all times.  \begin{figure}[ht]
\begin{center}
\includegraphics[width=3in]{ParticleRotatingWithPoint}
\renewcommand{\baselinestretch}{1}\selectfont
\caption{}
\label{Fig:RotatingFrameWithBody}
\renewcommand{\baselinestretch}{1.5}\selectfont
\end{center}
\end{figure}
Consider the orthonormal frame $\mathbf{b}(t)=[\mathbf{e}_r\:\:\:\:\mathbf{e}_{\theta}\:\:\:\:\mathbf{e}_{z}]$. 
Let $\mathbf{e}=[\mathbf{e}_1\:\:\:\mathbf{e}_2\:\:\:\mathbf{e}_3]$ be an Earth fixed frame. Then $\mathbf{b}(t)=\mathbf{e}R(t)$ where
\[
R(t)=\left[\begin{array}{ccc}\cos{\theta} & -\sin{\theta} & 0\\
\sin{\theta} & \cos{\theta} & 0\\
0 & 0 & 1
\end{array}\right].
\]
Thus we have
\[
\widehat{\Omega}=\begin{bmatrix}0 & -\dot{\theta} &0\\
\dot{\theta} & 0 & 0\\
0 & 0 &0\end{bmatrix},\:\:\:\:\:
\dot{\widehat{\Omega}}=\begin{bmatrix}0 & -\ddot{\theta} &0\\
\ddot{\theta} & 0 & 0\\
0 & 0 &0\end{bmatrix},\:\:\:\:\:\widehat{\Omega}^2=-\dot{\theta}^2\begin{bmatrix}1 & 0 &0\\
0 & 1 & 0\\
0 & 0 &0\end{bmatrix}.
\]
The representation of $P$ in this frame is
\[
X=\left[\begin{array}{c} r\\ 0 \\ 0
\end{array} \right]
\]
and hence we see that
\[
\dot{X}=\left[\begin{array}{c} \dot{r}\\
0 \\ 0 \end{array} \right]\:\:\:\:\ddot{X}=\left[\begin{array}{c} \ddot{r}\\
0 \\ 0 \end{array} \right].
\]




From Newton's equations in the $\mathbf{b}$-frame given by (\ref{eq:NewtonInMovingFrames}) we have
\[
m\left[\begin{array}{c} \ddot{r}\\
0\\ 0\end{array} \right]=-\left[\begin{array}{c} -m r\dot{\theta}^2\\
0 \\ 0\end{array} \right]-
 \left[\begin{array}{r} 0\\
2m\dot{r}\dot{\theta}\\ 0\end{array} \right]-
  \left[\begin{array}{c} 0\\
mr\ddot{\theta}\\ 0\end{array} \right]
+\left[\begin{array}{c}F_{r}\\
{F}_{\theta}\\ F_z\end{array} \right].
\]


Observe that the apparent force known as the Centrifugal force is $mr\dot{\theta}^2$ and is in the $\mathbf{e}_r$ direction, the Coriolis force is $-2m\dot{r}\dot{\theta} $ in the $\mathbf{e}_{\theta}$ direction and the Euler force is $-mr\ddot{\theta}$ in the $\mathbf{e}_{\theta}$ direction and we recover what we have learnt in our junior level physics classes.
Simplifying the above equations we have that
\begin{align*}
m\ddot{r} - mr\dot{\theta}^2&= F_r, \\
mr\ddot{\theta} +2m\dot{r}\dot{\theta}&= F_{\theta}\\
F_z &= 0.
\end{align*}

Observe that if we were to constrain the motion of the particle to a circle, then $r$ is a constant and thus we must necessarily exert a physical force $F_r=-mr\dot{\theta}^2$ in the $\mathbf{e}_r$ direction (radial direction) and a force 
$F_{\theta}=mr\ddot{\theta}$ in the $\mathbf{e}_\theta$ direction (tangential direction)  to enforce this constraint.
Observe that the radial force that we must exert is equal to the apparent force we call centrifugal force and the tangential force we must exert is equal to the negative of the apparent force we call Euler's force.
Compare the results of this with those obtained in exercise \ref{ex:ParticleOnCircle}.
%%%%%%%%%%%%%%%%%%%%%%%%%%%%%%%%%%%%%%%%%

\subsection{Example: Bead on a Rotating Hoop}\label{Secn:BeadOnRotatingHoop}
As an illustration of the use of (\ref{eq:NewtonInMovingFrames}) that represents Newton's equations in a moving frame we will consider the problem of analyzing the motion of a bead on a rotating hoop\footnote{This section was typed and illustrated by Mr. K. G. B. Gamagedara E/09/078.}. A schematic of the system is shown in figure-\ref{Fig:BeadOnHoop0}. 
\begin{figure}[h]
\begin{center}
\includegraphics[width=2in]{BallHoop2}
\renewcommand{\baselinestretch}{1}\selectfont
\caption{Bead on a Rotating Hoop.}
\label{Fig:BeadOnHoop0}
\renewcommand{\baselinestretch}{1.5}\selectfont
\end{center}
\end{figure}



Let us choose frames as shown in Figure-\ref{fig:hoop_top0} and Figure-\ref{fig:hoop_front0} and assume that the frame $\mathbf{e}$ is an inertial frame. We will denote by $P$ the position of the bead and by $O$ the origin of the $\mathbf{e}$-frame. Let $\mathbf{c}$
be another orthonormal frame such that it moves with respect to $\mathbf{e}$ in such a way that $\mathbf{c}_3\equiv \mathbf{e}_3$ and $\mathbf{c}_1$ is always orthogonal to the plane of the hoop as shown in figure-\ref{fig:hoop_top0}.
Then the two frames $\mathbf{c}$ and $\mathbf{e}$ are related by $\mathbf{c}=\mathbf{e}\,R_3(\theta)$ where we use the customary notation $R_i(\theta)$ to denote a rotation about the $i^\mathrm{th}$ axis by an angle equal to $\theta$. Specifically in this case,
\begin{align*}
R_3(\theta)=\begin{bmatrix}
\cos{\theta}&-\sin{\theta}&0\\
\sin{\theta}&\cos{\theta}&0\\
0&0&1
\end{bmatrix}.
\end{align*}


\begin{figure}[hbtp]
\minipage{0.5\textwidth}
  \begin{center}
  \includegraphics[scale=.5]{hoop_top.png}
  \caption{The view of the system from the top.}
  \label{fig:hoop_top0}
  \end{center}
\endminipage\hfill
\minipage{0.5\textwidth}
  \begin{center}
  \includegraphics[scale=.4]{hoop_front.png}
  \caption{View of the system from a direction perpendicular to the plane of the hoop.}
  \label{fig:hoop_front0}
  \end{center}
\endminipage\hfill
\end{figure}


Let $\mathbf{b}$
be another orthonormal frame such that it moves with respect to $\mathbf{c}$ in such a way that $\mathbf{b}_1\equiv \mathbf{c}_1$ and $\mathbf{b}_2$ is always along $OP$ as shown in figure-\ref{fig:hoop_front0}.
Then the two frames $\mathbf{b}$ and $\mathbf{c}$ are related by $\mathbf{b}=\mathbf{c}\,R_1(\phi)$ where
\begin{align*}
R_1(\phi)=\begin{bmatrix}
1&0&0\\
0&\cos{\phi}&-\sin{\phi}\\
0&\sin{\phi}&\cos{\phi}
\end{bmatrix}.
\end{align*}

Thus we have that, the two frames $\mathbf{b}$ and $\mathbf{e}$ are related by $\mathbf{b}=\mathbf{e}\,R_3(\theta)R_1(\phi)=\mathbf{e}\,R$ which gives that $R=R_3(\theta)R_1(\phi)$.

The representation of the position of $P$, in the $\mathbf{b}$ frame is seen to be, 
\begin{align*}
X&=\begin{bmatrix}
0\\r\\0
\end{bmatrix},
\end{align*}
and is independent of time and hence we have that $\dot{X}=\ddot{X}=0$. In fact it is this very convenient reason why we chose the $\mathbf{b}$-frame to move with the particle.
The Newton's equation (\ref{eq:NewtonInMovingFrames}) in the $\mathbf{b}$-frame then becomes, $F=m(\widehat{\Omega}^2+\dot{\widehat{\Omega}})X$,
where $F$ is the representation, in $\mathbf{b}$, of the fundamental forces acting on the bead due to its interaction with the rest of the Universe.

Notice that it is convenient to express the constraint interactions in the frame $\mathbf{b}$ while it is convenient to express the gravitational interaction in the $\mathbf{c}$-frame. Thus if $f$ is the representation of the force in the $\mathbf{e}$-frame and $F$ is the representation of the force in the $\mathbf{b}$-frame we have that
\begin{align*}
\mathbf{e}\,f=\mathbf{b}\,F&=
\mathbf{b}\begin{bmatrix}
F_{N1}\\ F_{N2}\\ 0
\end{bmatrix}
+\mathbf{c}
\begin{bmatrix}
0\\0\\-mg
\end{bmatrix}=\mathbf{b}\begin{bmatrix}
F_{N1}\\ F_{N2}\\ 0
\end{bmatrix}
+\mathbf{b}\,R_1^T(\phi)
\begin{bmatrix}
0\\0\\-mg
\end{bmatrix}
\end{align*}
Hence we have
\begin{equation*}
F=
\begin{bmatrix}
F_{N1}\\ F_{N2}\\ 0
\end{bmatrix}
+R_1^T(\phi)
\begin{bmatrix}
0\\0\\-mg
\end{bmatrix}=
\begin{bmatrix}
F_{N1}\\ F_{N2}-mg\sin{\phi}\\ -mg\cos{\phi}
\end{bmatrix}.
\end{equation*}



Now that we have found $F$ what remains to write down the Newton's equations $F=m(\widehat{\Omega}^2+\dot{\widehat{\Omega}})X$ explicitly in the moving frame $\mathbf{b}$ is the computation of $\dot{\widehat{\Omega}}$ and ${\widehat{\Omega}}^2$. We know that
$\widehat{\Omega}=R^T\dot{R}$.
Since
\begin{align*}
\dot{R}&=R_3\widehat{\Omega}_3R_1+R_3R_1\widehat{\Omega}_1=R(R_1^T\widehat{\Omega}_3R_1+\widehat{\Omega}_1)
\end{align*}
we have that
\begin{align*}
\widehat{\Omega}=R_1^T\widehat{\Omega}_3R_1+\widehat{\Omega}_1.
\end{align*}
Recall from (\ref{eq:AdjointAction}) we have that $\widehat{R\Omega}=R\widehat{\Omega}R^T$. Thus using the linearity of the $\:\widehat{}\:$ isomorphism we find that the $3\times 1$ version of $\widehat{\Omega}$ is explicitly given by
\begin{align*}
\Omega &= R_1^T(\phi)\Omega_3+\Omega_1=
\begin{bmatrix}
\dot{\phi} \\ \dot{\theta}\sin\phi \\ \dot{\theta}\cos\phi
\end{bmatrix}
\end{align*}
and hence that
\begin{align*}
\dot{\Omega} &= 
\begin{bmatrix}
\ddot{\phi} \\ \ddot{\theta}\sin\phi+\dot{\theta}\dot{\phi}\cos\phi \\ \ddot{\theta}\cos\phi-\dot{\theta}\dot{\phi}\sin\phi
\end{bmatrix}.
\end{align*}
From these we have that the corresponding skew-symmetric matrices are given by
\begin{align*}
\widehat{\Omega} &= 
\begin{bmatrix}
0 & -\dot{\theta}\cos\phi & \dot{\theta}\sin\phi \\ 
\dot{\theta}\cos\phi &0 &-\dot{\phi} \\
-\dot{\theta}\sin\phi &\dot{\phi} &0
\end{bmatrix}
\end{align*}
\begin{align*}
\dot{\widehat{\Omega}} &= 
\begin{bmatrix}
0 & -(\ddot{\theta}\cos\phi-\dot{\theta}\dot{\phi}\sin\phi) & (\ddot{\theta}\sin\phi+\dot{\theta}\dot{\phi}\cos\phi) \\ 
(\ddot{\theta}\cos\phi-\dot{\theta}\dot{\phi}\sin\phi) &0 &-\ddot{\phi} \\
-(\ddot{\theta}\sin\phi+\dot{\theta}\dot{\phi}\cos\phi) &\ddot{\phi} &0
\end{bmatrix}.
\end{align*}

Note from (\ref{eq:OmegaHatSqrd}) that
$\widehat{\Omega}^2=\Omega\Omega^T-||\Omega||^2I$ and hence that
\begin{align*}
\widehat{\Omega}^2 &=
\left[\begin{array}{ccc}  - {\dot{\theta}}^2 & \dot{\phi}\, \dot{\theta}\, \sin\phi & \dot{\phi}\, \dot{\theta}\, \cos\phi\\ \dot{\phi}\, \dot{\theta}\, \sin\phi &  - {\dot{\phi}}^2 - {\dot{\theta}}^2\, {\cos\phi}^2 & {\dot{\theta}}^2\, \cos\phi\, \sin\phi\\ \dot{\phi}\, \dot{\theta}\, \cos\phi & {\dot{\theta}}^2\, \cos\phi\, \sin\phi &  - {\dot{\phi}}^2 - {\dot{\theta}}^2\, {\sin\phi}^2 \end{array}\right]
\end{align*}

Substituting these in the Newton's equations (\ref{eq:NewtonInMovingFrames}) in the $\mathbf{b}$-frame we have
\begin{equation*}
\begin{bmatrix}
F_{N1}\\ F_{N2}\\ 0
\end{bmatrix}
=m\begin{bmatrix}
  -r\ddot{\theta}\cos{\phi} + 2r\dot{\phi}\dot{\theta}\sin{\phi}\\
 -r\dot{\phi}^2 - r\dot{\theta}^2\cos{\phi}^2 + g\sin{\phi} \\
    r\dot{\theta}^2\sin\phi\cos\phi + r\ddot{\phi} + g\cos{\phi}
\end{bmatrix}.
%\label{eq:F_values}
\end{equation*}

The first two rows can be used to find the constraint forces $F_{N1}$ and $F_{N2}$ that constrain the bead to stay on the hoop while the third row can be used to describe the motion of the bead as follows:
\begin{equation*}
r\ddot{\phi}=-\cos{\phi}\left(g+\dot{\theta}^2\sin\phi\right).
%\label{eq:charac_eqn}
\end{equation*}



Similarly to what we have done in this problem, you are invited to try out the exercises \ref{Ex:RotatingDiskMassInSlot} -- \ref{ex:MEMSgyro} at this point in order to get
an idea of how convenient it is to write down equations of motion using appropriately chosen moving frames and equation-\ref{eq:NewtonInMovingFrames}. The idea is that writing down position, velocity and the fundamental constrain force interactions are mostly convenient in a frame where the particle or the object appears fixed. Then what remains is the computation that relates this frame to an inertial frame and then use equation to write down Newton's equations in the moving frame in which the particle or the object appears to be fixed.

%%%%%%%%%%%%%%%%%%%%%%%%%%%%%%%%%%%
\subsubsection{Effects of Earth's Rotation About its Axis}\label{Secn:EffectOfEarthRotation}
In the following we also use the equations (\ref{eq:AppearentForcesMoving}) to show the effects of Earths rotation on gravity as well as on the 
formation of hurricanes and the motion of a long pendulum known as the Foucault's pendulum. 

\begin{figure}[ht]
\begin{center}
\includegraphics[width=3.5in]{EarthRotatn}
\renewcommand{\baselinestretch}{1}\selectfont
\caption{Effects of Earth's Rotation about its Axis: Frame $\mathbf{c}$ is fixed on the surface of the Earth with origin coinciding at $O'$. Frame $\mathbf{b}$ is parallel to $\mathbf{c}$ with center $O$ coinciding 
with the center of the earth. Frame $\mathbf{a}$ is fixed on the earth with origin at $O$ and $\mathbf{a}_3$ aligned along the axis of rotation of the earth and $\mathbf{a}_2$ in the $\mathbf{b}_2,\mathbf{b}_3$ plane. The origin of frame $\mathbf{e}$ coincides with $O$ 
and $\mathbf{e}_3$ is always aligned along $\mathbf{a}_3$.}
\label{Fig:EarthRotan}
\renewcommand{\baselinestretch}{1.5}\selectfont
\end{center}
\end{figure}

Let us consider the effect that the Earth's rotation about its axis has on the motion of a particle as observed in an Earth fixed frame at a point $O'$ (with latitude $\alpha$) on the 
surface of the Earth. Consider figure \ref{Fig:EarthRotan}. Let $\mathbf{c}(t)$ be an ortho-normal frame fixed on Earth with center at $O'$. The frame is oriented such that $\mathbf{c}_2$ is aligned 
perpendicular to the Earth (vertical direction), $\mathbf{c}_3$ is aligned in the South - North direction (towards north along the latitude), and $\mathbf{c}_1$ is aligned in the East - West direction 
(towards west along the longitude). Let $b$ be a frame that is parallel to $\mathbf{c}$ and fixed on Earth with origin at the center of the Earth $O$. Let $a$ be a frame with origin at $O$ and 
$\mathbf{a}_3$ aligned along the axis of rotation of the Earth, $\mathbf{a}_1 \equiv \mathbf{b}_1$ and $O'$ lies in the $\mathbf{a}_2 - \mathbf{a}_3$ plane as shown in figure-\ref{Fig:EarthRotan}.
Thus if $\mathbf{b}(t)=\mathbf{a}(t)R_1(\alpha)$
where $\alpha$ is the latitude angle and is a constant and $R_1(\alpha)$ is one of the three basic rotations given in (\ref{eq:RotatedFrames}).
Let $e$ be a frame with origin, $O$, at the center of Earth and parallel to a frame fixed on the sun. The frame is oriented such that the $\mathbf{e}_3$ direction coincides with the axis of 
rotation of the Earth, ie $\mathbf{a}_3\equiv \mathbf{e}_3$ and $\mathbf{a}(t)=\mathbf{e}\,R_3(\theta)$ where $\theta$ is the angle of rotation of the Earth about its axis.
\\
\\
We are interested in analyzing the motion of a particle $P$ as observed in the frame $\mathbf{c}(t)$. Let the representation of the point $P$ in the $\mathbf{c}(t)$ frame be $X_p(t)$, ie. $O'P=\mathbf{c}(t)X_p(t)
$. Since $OO'=\mathbf{c}(t)o$ and $\mathbf{c}$ is parallel to $\mathbf{b}$ we have that $OP=OO'+O'P=\mathbf{b}(t)(o+X_p)$.  Since by construction $O'$ is a distance $r$ ($r$ is the radius of the Earth) away on the 
$\mathbf{b}_2$ axis, $o=[0\:\:\:r\:\:\:0]^T$ and is a constant.


Thus if $x(t)$ is the representation of $P$ in $\mathbf{e}$ we have
\[
\mathbf{e}\,x(t)=\mathbf{b}(t)\,(o+X_p(t))=\mathbf{e}R_3R_1\underbrace{(o+X_p(t))}_{X}=\mathbf{e}RX.
\]
Thus $x=RX(t)$
where $R=R_3R_1$ and $X=o+X_p$.
From Newton's equations in the rotating frame, (\ref{eq:NewtonInMovingFrames}), we have
\begin{equation}\label{eq:AppearentForcesEarth}
m\, \ddot{X}(t)=-m\;\widehat{\Omega}^2(t) X(t)-2m\;\widehat{\Omega}(t)\dot{X}(t)- m\;\dot{\widehat{\Omega}}(t) X(t)+R^Tf(t).
\end{equation}
Let us decompose the total force in to two components $f=f^g+f^e$ where $f^g$ is the gravitational force and $f^e$ is the additional external forces.
Gravity acts in the $OP$ direction. Thus
\[
f^g=-\frac{mg}{||x||}x=-\frac{mg}{||X||}RX,
\]
and hence
\[
R^Tf^g=-\frac{mg}{||X||}X.
\]
Substituting this in (\ref{eq:AppearentForcesEarth}) and noting that the Earth is rotating at a constant rate (hence $\dot{\widehat{\Omega}}=0$)  we have
\begin{eqnarray}
\ddot{X} &=&-\widehat{\Omega}^2X-2\widehat{\Omega}\dot{X}-g\frac{X}{||X||}+\frac{1}{m}R^Tf^e,\label{eq:AppearentForcesEarth1}\\
 &=&-2\widehat{\Omega}\dot{X}-\left(gI_{3\times 3}+||X||\widehat{\Omega}^2 \right)\frac{X}{||X||}+\frac{1}{m}R^Tf^e.\label{eq:AppearentForcesEarth2}
\end{eqnarray}
\emph{Observe that these equations describe the motion of a particle as observed in a frame fixed on Earth with origin coinciding with the center of the Earth.}
From (\ref{eq:AppearentForcesEarth1}) it can be seen that the motion of the particle in the Earth fixed frame, in addition to the gravity and external forces, is also influenced by the 
Centrifugal and Coriolis terms that arise due to the ignorance of the rotation of the Earth. In equation (\ref{eq:AppearentForcesEarth2}) the Centrifugal term has been combined with the gravity term. 
This allows one to see that the effective gravity felt by an observer will change with the latitude of the location of the observer. We will explain this in a bit more detail at the end of this section.



Recall $X=o+X_p$ where $o$ is a constant and  $X_p$ is the representation of the point $P$ in the Earth fixed frame $c$ fixed on the surface of the Earth at $O'$. Then $\dot{X}=
\dot{X}_p$ and $\ddot{X}=\ddot{X}_p$ thus from (\ref{eq:AppearentForcesEarth2}) we have that
\begin{align}
\ddot{X}_p&=-2\widehat{\Omega}\dot{X}_p-\left(gI_{3\times 3}+||X_p+o||\widehat{\Omega}^2 \right)\frac{X_p+o}{||X_p+o||}+\frac{1}{m}R^Tf^e\nonumber\\
&=-2\widehat{\Omega}\dot{X}_p-\widehat{\Omega}^2{X}_p-\frac{g}{||X_p+o||}(X_p+o)-\widehat{\Omega}^2 o+\frac{1}{m}R^Tf^e\label{eq:AppearentForcesEarth3}
\end{align}
describes the motion of a point particle $m$ \emph{as observed in the Earth fixed frame with origin $O'$ on the surface of the Earth. This is the case that applies to us when we 
observe particle motion.} Since compared to $X_p$ the quantity $o$ is very large (since $r$ is very large) we can approximate $||(X_p+o)||\approx r$  and $(X_p+o)/||(X_p+o)||\approx o$ and then 
(\ref{eq:AppearentForcesEarth3}) approximates to
\begin{equation}
\ddot{X}_p=-2\widehat{\Omega}\dot{X}_p-\widehat{\Omega}^2{X}_p-\left(gI_{3\times 3}+r\widehat{\Omega}^2 \right)\chi+\frac{1}{m}R^Tf^e,\label{eq:AppearentForcesEarth4}
\end{equation}
where $\chi=[0\:\:\:1\:\:\:0]^T$.
These equations can be used to describe many natural phenomena. For instance it explains why a Hurricane formed in the Nothern hemisphere rotates in a counter--clockwise 
direction and a Hurricane formed in the Southern hemisphere rotates in a  clockwise direction (see figure \ref{Fig:Hurricane}).
You are asked to show this in exercise \ref{ex:Hurricane}. It can also be used to show the precession of the oscillating plane in the Foucault's pendulum (see figure-\ref{Fig:Foucault}).

\begin{figure}[ht]
\begin{center}
\begin{tabular}{cc}
\includegraphics[width=1in]{Raoni_2021-06-29_1728Z.jpg} & \includegraphics[width=1.35in]{Polar_low}\\
(a) \href{https://en.wikipedia.org/wiki/South_Atlantic_tropical_cyclone#/media/File:Raoni_2021-06-29_1728Z.jpg}{\copyright Wikipedia}. Subtropical Storm Raoni.  & (b)  \href{https://en.wikipedia.org/wiki/Polar_low#/media/File:Polar_low.jpg}{$\copyright$Wikipedia.} A Northern Polar Hurricane.
\end{tabular}
\caption{Figures show the clockwise and anti-clockwise rotation of respectively a southern hemisphere and northern hemisphere formed hurricane. } \label{Fig:Hurricane}
\end{center}
\end{figure}


The rotational velocity of the Earth,
$\Omega_e =
\frac{2\pi}{23h \:56m \:4 s}
= 7.292 \times 10^{-5} rad/s$, is very small and thus $\widehat{\Omega}$ and $\widehat{\Omega}^2$ are very small and for most applications these effects can be neglected and then the equations (\ref{eq:AppearentForcesEarth4})
reduce to the usual equations of projectile motion given by
\begin{equation}\label{eq:ApproximateEarth}
\ddot{X}_p=-g\chi+\frac{1}{m}R^Tf^e.
\end{equation}
Explicitly written down they are:
\begin{eqnarray}
\ddot{X}_{p_1} &=& \frac{1}{m}F_1,\label{eq:Earth4}\\
\ddot{X}_{p_2} &=& -g+\frac{1}{m}F_2,\label{eq:Earth5}\\
\ddot{X}_{p_3} &=& \frac{1}{m}F_3.\label{eq:Earth6}
\end{eqnarray}

\begin{figure}[ht]
\begin{center}
\includegraphics[width=4in]{FoucaultPendulum} 
\caption{The shift in the plane of oscillation of the Foucault Pendulum. Figure taken from \url{https://www.schoolphysics.co.uk/age14-16/Astronomy/text/Foucaults_pendulum/index.html}} \label{Fig:Foucault}
\end{center}
\end{figure}


Let us now explicitly consider the effects of the rotation of Earth. Let 
\begin{align*}
\gamma(t)&\triangleq-\left(gI_{3\times 3}+r\widehat{\Omega}^2 \right)\chi+\frac{1}{m}R^Tf^e,
\end{align*} 
and then (\ref{eq:AppearentForcesEarth4}) can be expressed as
\begin{equation}
\ddot{X}_p=-\widehat{\Omega}^2{X}_p-2\widehat{\Omega}\dot{X}_p+\gamma(t),\label{eq:AppearentForcesEarth5}
\end{equation}
Defining $Z_p=[X_p^T\:\:\:\:\dot{X}_p^T]^T$ we can arrange this equation as
\[
\dot{Z}_p=AZ_p+B(t),
\]
where
\[
A=\begin{bmatrix}0 & I_{3\times 3}\\-\widehat{\Omega}^2 & -2\widehat{\Omega}\end{bmatrix},
\:\:\:\:
B(t)=\begin{bmatrix}0 \\ \gamma(t)\end{bmatrix}.
\]
From linear systems theory we find that the solution to this differential equation is explicitly given by
\begin{equation}
{Z}_p(t)=e^{At}{Z}_p(0)+\int_0^te^{A(t-\tau)}B(\tau)\,d\tau.\label{eq:AppearentForcesEarth55}
\end{equation}



To compute $e^{At}$ we need $\widehat{\Omega}$ that is given by $R^T\dot{R}=\widehat{\Omega}$. Since $R=R_3R_1$ differentiating we have that
\[
\dot{R}=\dot{R}_3R_1=R_3\widehat{\Omega}_eR_1=R_3R_1\, R_1^T\widehat{\Omega}_eR_1=R\widehat{\Omega},
\]
Now from $\dot{R}_3=R_3\widehat{\Omega}_e$  we have that
\[
\widehat{\Omega}_e= \left[
\begin{array}{ccc}
0  & -\dot{\theta}_e & 0\\
\dot{\theta}_e & 0 & 0\\
0 & 0 & 0
\end{array}
\right],
\]
where $\dot{\theta}=\dot{\theta}_e$ is the angular velocity of Earth about its axis of rotation.
Hence we have  that
\[
\widehat{\Omega}=R_1^T\widehat{\Omega}_eR_1.
\]
 Thus we have
\[
\widehat{\Omega}=\dot{\theta}_e \left[
\begin{array}{ccc}
0 & -\cos{\alpha} & \sin{\alpha} \\
\cos{\alpha} & 0 & 0\\
-\sin{\alpha} & 0 & 0
\end{array}
\right],
\]
and
\[
\widehat{\Omega}^2= -\dot{\theta}^2_e\left[
\begin{array}{ccc}
1 & 0 & 0\\
0 & \cos^2{\alpha} & -\cos{\alpha}\sin{\alpha} \\
0 & -\cos{\alpha}\sin{\alpha}  & \sin^2{\alpha}
\end{array}
\right].
\]
Substituting these in (\ref{eq:AppearentForcesEarth55}) we can explicitly find $X_p(t)$ and 
$\dot{X}_p(t)$. Below we will do this for a special case where the observer is on the equator of the Earth.

\begin{example}\label{ex:Projectile}
Consider the following problem. A cannon is released from a geostationary weather balloon that is  at a point which is directly $h$ meters above a point on the equator. Where will the cannon land?

At the equator the latitude is zero, that is $\alpha=0$. Since no other external forces are present $F=0$. Also since the cannon is released carefully $\dot{X}_p(0)=[0\:\:\:0\:\:\:0]^T$ and ${X}_p(0)=[0\:\:\:h\:\:\:0]^T$
In this case
\[
\widehat{\Omega}= \dot{\theta}_e\left[
\begin{array}{ccc}
0 & -1 & 0 \\
1 & 0 & 0\\
0 & 0 & 0
\end{array}
\right],\:\:\:\:\:\:
\widehat{\Omega}^2= -\dot{\theta}^2_e\left[
\begin{array}{ccc}
1 & 0 & 0\\
0 & 1 & 0 \\
0 & 0  & 0
\end{array}
\right],\:\:\:
\gamma=-(g-r\dot{\theta}_e^2)\begin{bmatrix}0 \\ 1\\ 0\end{bmatrix}.
\]

{\tiny
\[
e^{At}=\left[\begin{array}{cccccc} \cos\!\left(t\, \dot{\theta}_e\right) + t\, \dot{\theta}_e\, \sin\!\left(t\, \dot{\theta}_e\right) & \sin\!\left(t\, \dot{\theta}_e\right) - t\, \dot{\theta}_e\, \cos\!\left(t\, \dot{\theta}_e\right) & 0 & t\, \cos\!\left(t\, \dot{\theta}_e\right) & t\, \sin\!\left(t\, \dot{\theta}_e\right) & 0\\ t\, \dot{\theta}_e\, \cos\!\left(t\, \dot{\theta}_e\right) - \sin\!\left(t\, \dot{\theta}_e\right) & \cos\!\left(t\, \dot{\theta}_e\right) + t\, \dot{\theta}_e\, \sin\!\left(t\, \dot{\theta}_e\right) & 0 & - t\, \sin\!\left(t\, \dot{\theta}_e\right) & t\, \cos\!\left(t\, \dot{\theta}_e\right) & 0\\ 0 & 0 & 1 & 0 & 0 & t\\ t\, \dot{\theta}_e^2\, \cos\!\left(t\, \dot{\theta}_e\right) & t\, \dot{\theta}_e^2\, \sin\!\left(t\, \dot{\theta}_e\right) & 0 & \cos\!\left(t\, \dot{\theta}_e\right) - t\, \dot{\theta}_e\, \sin\!\left(t\, \dot{\theta}_e\right) & \sin\!\left(t\, \dot{\theta}_e\right) + t\, \dot{\theta}_e\, \cos\!\left(t\, \dot{\theta}_e\right) & 0\\ - t\, \dot{\theta}_e^2\, \sin\!\left(t\, \dot{\theta}_e\right) & t\, \dot{\theta}_e^2\, \cos\!\left(t\, \dot{\theta}_e\right) & 0 &  - \sin\!\left(t\, \dot{\theta}_e\right) - t\, \dot{\theta}_e\, \cos\!\left(t\, \dot{\theta}_e\right) & \cos\!\left(t\, \dot{\theta}_e\right) - t\, \dot{\theta}_e\, \sin\!\left(t\, \dot{\theta}_e\right) & 0\\ 0 & 0 & 0 & 0 & 0 & 1 \end{array}\right]
\]
}
Then from (\ref{eq:AppearentForcesEarth55}) and the initial conditions we have
\[
X_p(t)=h\begin{bmatrix}\sin\!\left(t\, \dot{\theta}_e\right) - t\, \dot{\theta}_e\, \cos\!\left(t\, \dot{\theta}_e\right)\\
\cos\!\left(t\, \dot{\theta}_e\right) + t\, \dot{\theta}_e\, \sin\!\left(t\, \dot{\theta}_e\right)\\ 0
\end{bmatrix}
-(g-r\dot{\theta}_e^2)\int_0^t\begin{bmatrix}(t-\tau)\sin{(\dot{\theta}_e(t-\tau))} \\ (t-\tau)\cos{(\dot{\theta}_e(t-\tau))} \\0\end{bmatrix}d\tau
\]
Using a power series expansion of the $\sin$ and $\cos$ terms and neglecting terms of $\dot{\theta}_e^3$ and higher we have
\begin{align*}
X_p(t)&=h\begin{bmatrix}0\\
\left(1-\frac{t^2\dot{\theta}_e^2}{2}\right)\\ 0
\end{bmatrix}
-(g-r\dot{\theta}_e^2)\int_0^t\begin{bmatrix}\dot{\theta}_e(t-\tau)^2 \\ (t-\tau)-\frac{\dot{\theta}_e^2(t-\tau)^3}{2} \\0\end{bmatrix}d\tau\\
&=h\begin{bmatrix}0\\
\left(1-\frac{t^2\dot{\theta}_e^2}{2}\right)\\ 0
\end{bmatrix}
-(g-r\dot{\theta}_e^2)\begin{bmatrix}\frac{\dot{\theta}_et^3}{3} \\ \frac{t^2}{2}-\frac{\dot{\theta}_e^2t^4}{8} \\0\end{bmatrix}.
\end{align*}
Let $T$ be the time it takes for the cannon to land on the ground. That is $X_{p_2}(T)\approx 0$. Thus from the second line of the matrix expression above we have
\[
T\approx\sqrt{\frac{2h}{(g-r\dot{\theta}_e^2)}}.
\]
At this time instant the first line of the matrix expression above gives us that
\[
X_{p_1}(T)\approx -\frac{(g-r\dot{\theta}_e^2)\dot{\theta}_e}{3}T^3=-\frac{2h\dot{\theta}_e}{3}\sqrt{\frac{2h}{(g-r\dot{\theta}_e^2)}}
\]
%%%%%%%%%%%%%%%%%%%%%



That is, a cannon dropped from a vertical height, $h$, from a point $O'$ on the equator will land at a distance ${(2h\dot{\theta}_e/3)}\sqrt{{2h}/{(g-r\dot{\theta}_e^2)}}$ to the East from $O'$. Approximately if 
$h=500\,m$ then $T\approx 10s$ and the cannon will fall $1\,cm$ towards the East. Can you explain why it would fall to the East instead of the West?
\end{example}




%%%%%%%%%%%%%%%%%%%%%%%%%%%%%%%%








%%%%%%%%%%%%%%%%%%%%%%%%%%%%%%%%%%%%%%%%%%%
\section{Rigid Body Motion}\label{Secn:RigidBodyKinematics}
In this section we consider the motion of a collection of non co-linear interacting particles $P_1,P_2,\cdots,P_n$ where the interactions ensure that the relative distance between any two particles remain the same at all time. Such a set of particles is called a \textit{rigid body}. Rigid or not we have shown in section \ref{Secn:InteractingParticleMotion} that 
the rate of change total linear momentum of the particles of the body is equal to the total resultant external forces acting on the body and that the rate of change total  angular momentum of the particles of the body about the center of mass of the body is equal to the total resultant moment of the external forces acting on the body. 
From (\ref{eq:AngularMomentum_i}) and (\ref{eq:BodyPi}) we find that even though writing down the total angular momentum in an inertial frame $\mathbf{e}$ is not straightforward it can be more conveniently expressed in a body fixed frame $\mathbf{b}$. Below we will see that this allows one to write down the equations of motion of a rigid body in a much more tractable form.

%\begin{figure}[ht]
%\begin{center}
%\begin{tabular}{cc}
%\includegraphics[width=1.5in]{segway_thumb} & \includegraphics[width=3.1in]{PeradeniyaCodeGenQuad10}\\
%(a) & (b)\\
%\includegraphics[width=2.8in]{11mars_rover_h} & \includegraphics[width=3in]{rover-on-floor-2}\\
%(c) & (d)
%\end{tabular}
%\renewcommand{\baselinestretch}{1}\selectfont
%\caption{Two examples of rigid body motion. Figure (a) is a picture of a Segway, figure (b) is Quadrotor UAV, figure (c) is a model of the Mars 
%rover land robot and figure (d) is a picture of a mobile land robot with a manipulator.}
%\label{Fig:RigidBodyExamples}
%\renewcommand{\baselinestretch}{1.5}\selectfont
%\end{center}
%\end{figure}




%\section{Description of Rigid Body Motion}\label{eq:RigidBodyMotion}
\begin{figure}[ht]
\begin{center}
\includegraphics[width=3.5in]{RigidBody}
\renewcommand{\baselinestretch}{1}\selectfont
\caption{The motion of a set of particles that  appear to be fixed with respect to the moving frame $\mathbf{b}$.}
\label{Fig:RigidBody0}
\renewcommand{\baselinestretch}{1.5}\selectfont
\end{center}
\end{figure}
Let $\mathbf{e}$ be an inertial frame with origin $O$ and and let $\mathbf{b}(t)$ be an ortho-normal frame with origin $O'$ in which all the particles $P_i$ appear to be fixed as illustrated for example in figure \ref{Fig:RigidBody0}. We will call $\mathbf{b}(t)$ the body frame. Let $OO'=\mathbf{e}o(t)$. The position of the $i^\mathrm{th}$ point $P_i$ at a time $t$, is given by $x_i(t)$ with respect to the frame $\mathbf{e}$ and by $X_i$ 
with respect to the body frame $\mathbf{b}(t)$. Observe that since all points on the body appear to be fixed with respect to the body frame $\mathbf{b}(t)$, the representation $X_i$ is independent of time. Therefore specifying $\mathbf{b}(t)$ amounts to specifying the configuration 
of the rigid body. Since $\mathbf{b}(t)$ is uniquely related to the inertial frame $\mathbf{e}$ by the rotational matrix $R(t)$, where $\mathbf{b}(t)=\mathbf{e}\,R(t)$, and the position of its origin $o(t)$, the specification of $(o(t),R(t))$ amounts to the unique specification of the configuration of the rigid body with respect to the inertial frame $\mathbf{e}$. Similarly any $(o,R)$ where $o\in \mathbb{R}^3$ and $R\in \mathrm{SO}(3)$ defines a unique configuration of the rigid body. Thus the configuration space of rigid body motion is $ \mathbb{R}^3\times \mathrm{SO}(3)$ where we have denoted the space of $3\times 3$ special orthogonal matrices by $\mathrm{SO}(3)$. Recall that it was shown in section-\ref{Secn:PropertiesOfRotations} that the space $\mathrm{SO}(3)$ is a three dimensional space and hence it follows that a rigid body has 6 DOF. 

The pair $(o,R)\in \mathbb{R}^3 \times \mathrm{SO}(3)$ can also be identified with a unique $4\times 4$ matrix
\begin{align}
E&=\begin{bmatrix}
R & o \\ 0 & 1\end{bmatrix}.
\end{align}
The spcace of all such $4\times 4$ matrices are referred to as the space of special euclidean group of matrices that is denote by $SE(3)$. 

\subsection{Rigid Body Kinematics}

Recall that the quantity $\Omega$ where $\widehat{\Omega}=R^T\dot{R}$ corresponds to an instantaneous rotation of the frame $\mathbf{b}$ (and hence the body) about the axis $\Omega$ as expressed in the body frame $\mathbf{b}$ by an 
amount equal to the magnitude $||\Omega||$ and thus that $\Omega$ can be defined to be the \textit{body angular velocity} of the rigid body. The $\mathbf{e}$ frame version of this quantity $\omega \triangleq R\Omega$ is defined to be the \textit{spatial angular velocity} of the rigid body.
The equation
\begin{align}
\dot{R}&=R\widehat{\Omega}
\end{align}
that defines angular velocity is usually referred to as the \textit{rigid body kinematic} equations.
Since $\widehat{\Omega}=\widehat{R^T\omega}=R^T\widehat{\omega}R$ this can also be equivalently written down as
\begin{align}
\dot{R}&=\widehat{\omega}R.
\end{align}

Recall that the total linear momentum of a set of interacting particles can be written down as
\begin{align}
p&=\sum_{i}p_i=\sum_{i}m_i\left(\dot{o}+R(\widehat{\Omega}{X}_i+\dot{X}_i)\right).
\end{align}
In the case of a rigid body $\dot{X}_i=0_{3\times 1}$ and hence
\begin{align}
p&=M(\dot{o}+R\widehat{\Omega}\bar{X})=M(\dot{o}+\widehat{\omega}R\bar{X})=M\dot{\bar{x}},
\end{align}
where $M=\sum_{i}m_i$, $\bar{X}=\sum_{i}m_i X_i/\sum_{i}m_i$ is the center of mass of the rigid body represented with repect to frame $\mathbf{b}$ fixed to the body (body frame).

Thus we also see that that the translational kinematics are given by
\begin{align}
\dot{o}&=\frac{1}{M}p-\omega \times R\bar{X}.
\end{align} 

Also recall that angular momentum of a point particle of mass $m_i$ about the origin $O'$ of a $\mathbf{b}$ frame takes the form
\begin{align}
\pi_i&= R\left(\mathbb{I}_i\Omega+m_iX_i\times( R^T\dot{o}+ \dot{X}_i)\right),
\end{align}
in the $\mathbf{e}$ frame and 
\begin{align}
\Pi_i= \left(\mathbb{I}_i\Omega+m_iX_i\times(\dot{X}_i+ R^T\dot{o})\right),
\end{align}
in the $\mathbf{b}$ frame where the quantity
\begin{align}
\mathbb{I}_i &\triangleq -m_i\widehat{X}_i^2=m_i\left(||X_i||^2I_{3\times 3} - X_iX_i^T\right),
\end{align} 
is defined as the *moment of inertia* of the particle $P_i$ about the point $O'$ in the frame $\mathbf{b}$.

Since in a rigid body the particles are fixed with respect to the frame $\mathbf{b}$ we have that $\dot{X}_i=0_{3\times 1}$ and then we have
\begin{align}
\pi_i&= R\underbrace{\left(\mathbb{I}_i\Omega+m_iX_i\times R^T\dot{o}\right)}_{\Pi_i},
\end{align}
Summing them over all the particles we have that the total angulare momentum about $O'$ is given by
\begin{align}
\pi&=\sum_{i}\pi_i= R\underbrace{\left(\mathbb{I}\Omega+M\bar{X}\times R^T\dot{o}\right)}_{\Pi},
\end{align}
where $M=\sum_{i}m_i$, $\bar{X}=\sum_{i}m_i X_i/\sum_{i}m_i$ is the representation of the center of mass of the rigid body in the $\mathbf{b}$ frame, and
\begin{align}
\mathbb{I} &\triangleq \sum_{i}-m_i\widehat{X}_i^2=\sum_{i}m_i\left(||X_i||^2I_{3\times 3} - X_iX_i^T\right),
\end{align} 
is defined as the \textit{moment of inertia} of the body about the point $O'$ with respect to the frame $\mathbf{b}$. It is easy to see that this is a symmetric positive definite matrix if all the points are not co-linear.

Since $p=M\dot{\bar{x}}=M(\dot{o}+R(\Omega\times \bar{X}))$ we see that $M\dot{o}\times \dot{\bar{x}}=\dot{o}\times p=-R(\Omega \times \bar{X})\times p$ we also see that
\begin{align}
\pi&=R\left(\mathbb{I}\Omega +M\bar{X}\times R^T\left(\frac{1}{M}p-R(\Omega\times \bar{X})\right)\right)\\
&=R\left(\mathbb{I}\Omega +\bar{X}\times R^Tp+M\bar{X}\times \bar{X} \times\Omega\right)\\
&=R\left((\mathbb{I}+M\widehat{\bar{X}}^2)\Omega +\bar{X}\times R^Tp)\right)\\
&=R\left(\mathbb{I}_c\Omega +\bar{X}\times R^Tp)\right).
\end{align}
where 
\begin{align*}
\mathbb{I}_c&\triangleq \mathbb{I}+M\widehat{\bar{X}}^2
\end{align*} 
is the inertia tensor of the body with respect to a frame that is parallel to $\mathbf{b}$ and origin coinciding with the center of mass of the object, $O_c$.



When written as 
\begin{align}
\mathbb{I}&=\mathbb{I}_c-M\widehat{\bar{X}}^2
\end{align} 
this turns out to be the \textit{parallel axis theorem}. 

Also note that since $\Omega =R^T\omega$ we have
\begin{align}
\pi&=R\left(\mathbb{I}_c\Omega +\bar{X}\times R^Tp)\right),\\
&=(R\mathbb{I}_cR^T)\omega +R\bar{X}\times p,\\
&=\mathbb{I}_c^R\omega +R\bar{X}\times p.
\end{align}
where 
\begin{align}
\mathbb{I}_c^R\triangleq R\mathbb{I}_cR^T=R(\mathbb{I}+M\widehat{\bar{X}}^2)R^T,
\end{align}
is defined to be the \textit{locked inertia tensor} in the body. It can be shown that $\mathbb{I}_c^R$ is the moment of inertia tensor of the body with respect to a frame that is parallel to $\mathbf{e}$ and  orign coinciding with the center of mass $O_c$.

\begin{svgraybox}
In summary consider a set particles which are rigid with respect to each other and $\mathbf{b}$ be a body fixed frame with origin coinciding with the point $O'$ around which the moments are defined.
Then the angular momentum of the particles about a point $O'$ can be written down in the following equivalent forms:

\begin{align}
\pi&=R\left(\mathbb{I}\Omega+M\bar{X}\times R^T\dot{o}\right)
\end{align}
where $\mathbb{I}$ is the moment of inertia tensor with respect to the frame $\mathbf{b}$ with origin $O'$,
\begin{align}
\pi&=R\left(\mathbb{I}_c\Omega+\bar{X}\times R^Tp\right)
\end{align}
where $\mathbb{I}_c$ is the moment of inertia tensor with respect to a frame that is parallel to $\mathbf{b}$ and  origin coinciding with the center of mass $O_c$, and
\begin{align}
\pi&=\mathbb{I}_c^R\omega+R\bar{X}\times p
\end{align}
where $\mathbb{I}_c^R$ is the moment of inertia tensor with respect to a frame that is parallel to $\mathbf{e}$ and  origin coinciding with the center of mass $O_c$.
\end{svgraybox}

In the case where the moments are taken about the center of mass of the body we have that
\begin{align}
\pi&=R\mathbb{I}_c\Omega=\mathbb{I}_c^R\omega.
\end{align}
%%%%%%%%

\subsection{Rigid Body Equations}
Rigid or not we have shown in the section \ref{Secn:InteractingParticleMotion} that:
\begin{align}
\dot{p}&=M\ddot{\bar{x}}=f^e,\\
\dot{\pi}&=-M\dot{o}\times \dot{\bar{x}}+\tau_e
\end{align} 
where $M=\sum_{i=1}^nm_i$ is the total mass of the particles, $\bar{x}$ is the representation of the center of mass of the set of particles in the inertial frame $\mathbf{e}$, $p=\sum_{i=1}^np_i$ is the total linear momentum of the system of particles, $f^e=\sum_{i=1}^n f_i^e$ is the total resultant of the external forces acting on the particles, $\pi =\sum_{i=1}^n\pi_i$ is the total angular momentum of the particles about $O'$, $\tau^e=\sum_{i=1}^n(x_i-o)\times f_i^e$ is the resultant force moment of the external interactions acting on the particles about the point $O'$, and $o$ is the representation of the point $O'$ in the frame $\mathbf{e}$.

\textit{Notice the extreme simplicity of the form of these governing equations however complicated the system of particles is}.

\subsubsection{Rigid body equations in the spatial frame}

Let $\bar{x}'\triangleq (\bar{x}-o)=R\bar{X}$. Differentiating this expression gives
\begin{align}
\dot{\bar{x}}'&=R(\Omega \times \bar{X})=\omega \times \bar{x}'.
\end{align}
Then the fact that $p=M\dot{\bar{x}}$ gives
\begin{align}
\dot{\pi}&=-M\dot{o}\times \dot{\bar{x}}+\tau_e=M\dot{\bar{x}}'\times \dot{\bar{x}}+\tau_e=\omega \times \bar{x}'\times p+\tau_e
\end{align} 

We also saw that the rigid body kinematic equations are given by $M\dot{o}=p-M\omega \times \bar{x}'$ and $\dot{R}=\widehat{\omega}R$.



Putting these together we arrive at the fully determined set of coupled ODEs:
\begin{svgraybox}
\begin{align}
\dot{o}&=\frac{1}{M}p-\omega \times \bar{x}',\\
\dot{R}&=\widehat{\omega}R,\\
\dot{p}&=f^e,\\
\dot{\pi}&=\omega \times \bar{x}'\times p+\tau_e
\end{align} 
where 
\begin{align}
\omega &=(\mathbb{I}_c^R)^{-1}\left(\pi-\bar{x}'\times p\right).
\end{align}
\end{svgraybox}
Solving the above equations one can obtain $R(t)$ and $o(t)$ and hence uniquely describe the motion of the rigid body.


Note that if one choses to \textit{take the moments about the center of mass}  of the rigid body then then the above equations become even simpler:

\begin{align}
\dot{o}&=\frac{1}{M}p\\
\dot{R}&=\widehat{\omega}R,\\
\dot{p}&=f^e,\\
\dot{\pi}&=\tau_e,
\end{align} 
and 
\begin{align}
\omega &=(\mathbb{I}_c^R)^{-1}\pi.
\end{align}
Equations of systems don't come any simpler than this!!!

These simple expressions are valid for any rigid body motion and we will use them heavily in our controller development and simulation of rigid body motion.

\subsubsection{Rigid body equations in the body frame}

Differentiating $p$ and $\pi$ we have 
\begin{align}
\dot{p}&=M\ddot{o}+R\left(M\,\left(\widehat{\Omega}^2(t) +\dot{\widehat{\Omega}}(t)\right)\bar{X}\right)=  R F^e=f^e,
\end{align}
and
\begin{align}
\dot{\pi}&=R\left(\mathbb{I}\dot{\Omega}-\mathbb{I}{\Omega}\times \Omega-MR^T\dot{o}\times \Omega\times \bar{X} +M\bar{X}\times R^T\ddot{o}\right)\\
&=R\left(-MR^T\dot{o}\times \Omega\times {\bar{X}}+T^e\right)=-M\dot{o}\times \dot{\bar{x}}+\tau^e.
\end{align}

Thus we have that the rigid body equations $\dot{p}=f^e$ and $\dot{\pi}=-M\dot{o}\times \dot{\bar{x}}+ \tau^e$ are equivalently expressed in the $\mathbf{b}$ frame by
\begin{align}
MR^T\ddot{o}+\left(M\,\left(\widehat{\Omega}^2(t) +\dot{\widehat{\Omega}}\right)\bar{X}\right)&= F^e,\\
\left(\mathbb{I}\dot{\Omega}-\mathbb{I}{\Omega}\times \Omega+M\bar{X}\times R^T\ddot{o}\right)&=T^e.
\end{align}
Re-arranging them we obtain
\begin{align}
MR^T\ddot{o}-M\,\widehat{\bar{X}}\dot{\Omega}&= -M\,\widehat{\Omega}^2(t) \bar{X} + F^e,\\
\mathbb{I}\dot{\Omega}&=\mathbb{I}{\Omega}\times \Omega-M\bar{X}\times R^T\ddot{o}+T^e.
\end{align}
Replacing $R^T\ddot{o}$ in the last equation with the one before that we obtain
\begin{align}
\mathbb{I}\dot{\Omega}&=\mathbb{I}{\Omega}\times \Omega-\bar{X}\times \left(M\,\widehat{\bar{X}}\dot{\Omega}-M\,\widehat{\Omega}^2 \bar{X} + F^e\right)+T^e.
\end{align}
Which gives
\begin{align}
(\mathbb{I}\,+M\,\widehat{\bar{X}}^2)\dot{\Omega}&=\mathbb{I}{\Omega}\times \Omega+M\bar{X}\times \widehat{\Omega}^2 \bar{X} - \bar{X}\times F^e+T^e.
\end{align}

Note that 
\begin{align*}
\bar{X}\times \widehat{\Omega}^2\bar{X}&=\widehat{\bar{X}}\left(\Omega\Omega^T-||\Omega||^2I_{3\times 3}\right)\bar{X}=\widehat{\bar{X}}\Omega(\Omega\cdot \bar{X})=-\widehat{\Omega}\bar{X}\widehat{X}^T\Omega\\
&=-\widehat{\Omega}\left(\bar{X}\widehat{X}^T-||\bar{X}||^2I_{3\times 3}\right)\Omega=-\Omega\times \widehat{\bar{X}}^2\Omega
\end{align*}
This when substituted in the previous expression results in
\begin{align}
\mathbb{I}_c\dot{\Omega}+\Omega \times \mathbb{I}_c{\Omega}=-\bar{X}\times F^e+T^e,
\end{align}
where 
\begin{align*}
\mathbb{I}_c&\triangleq \mathbb{I}+M\widehat{\bar{X}}^2
\end{align*} 
is the inertia tensor of the body with respect to a frame that is parallel to $\mathbf{b}$ and origin coinciding with the center of mass of the object, $O_c$, and $T^e$ is the resultant moment of the external forces about the origin $O'$ of the body frame $\mathbf{b}$. 


Let $o_c$ be the representation of the center of mass of the body with respect to the inertial frame $\mathbf{e}$. Thus $\tau_c^e=\sum_{i=1}^n(x_i-o_c)\times f_i^e$ is the resultant moment of the external forces about the center of mass of the rigid body. Note that $\bar{X}=R^T(o_c-o)$. Thus we have
\begin{align*}
\tau^e=\sum_{i=1}^n(x_i-o)\times f_i^e=\sum_{i=1}^n(x_i-o_c+o_c-o)\times f_i^e=\tau^e_c+R\bar{X}\times f^e=\tau^e_c+R(\bar{X}\times F^e).
\end{align*}
Then we have that the $\left(\mathbb{I}\dot{\Omega}-\mathbb{I}{\Omega}\times \Omega+M\bar{X}\times R^T\ddot{o}\right)=T^e$ take the equivalent form
\begin{align*}
\mathbb{I}_c\dot{\Omega}+\Omega \times \mathbb{I}_c{\Omega}=T^e_c,
\end{align*}
where $T^e_c\triangleq -\bar{X}\times F^e+T^e$ is the $\mathbf{b}$-frame representation of the resultant moments acting on the body with respect to the center of mass of the body.

Thus in summary we have that the equations of motion of the system of rigid particles are completely specified in the body frame as follows:
\begin{svgraybox}
\begin{align*}
\dot{R}&=R\widehat{\Omega},\\
MR^T\ddot{o}-M\,\widehat{\bar{X}}\dot{\Omega}&= -M\,\widehat{\Omega}^2(t) \bar{X} + F^e,\\
\mathbb{I}_c\dot{\Omega}&=\mathbb{I}_c\Omega \times {\Omega}+T^e_c.
\end{align*}
\end{svgraybox}

Again when the moments are defined about the center of mass we have the much simpler set of equations:
\begin{align}
\dot{R}&=R\widehat{\Omega},\\
M\ddot{o}&= f^e,\\
\mathbb{I}_c\dot{\Omega}&=\mathbb{I}_c\Omega \times {\Omega}+T^e_c.\label{eq:EulerEqns3D}
\end{align}
\textit{These are commonly referred to as Euler's Rigid Body Equations}.

Since the moment of inertia tensor is always symmetric and positive definite it is always possible to find a body frame $\mathbf{b}$ such that the Inertia tensor $\mathbb{I}_c$ is diagonalized. For instance in an axi-symmetric rigid body if the body frame $\mathbf{b}$ is aligned along the axes of symmetry then it can be shown that the inertia matrix $\mathbb{I}_c$ is diagonal. The diagonal elements are called the principle moments of inertia.
Let the body frame $\mathbf{b}$ be chosen such that the inertia tensor is diagonalized
\begin{align}
\mathbb{I}_c=\left[ \begin{array}{ccc} \mathbb{I}_1 & 0 & 0\\ 0 & \mathbb{I}_2 & 0\\0 & 0 & \mathbb{I}_3\end{array}\right],
\end{align}
where the principle moments of inertia are given by $\mathbb{I}_1,\mathbb{I}_2,\mathbb{I}_3$.  Based on the values of the principle moments of inertia one can classify rigid bodies into three distinct categories:
\begin{enumerate}[(a)]
\item {Asymmetric Rigid Body:} $\mathbb{I}_1>\mathbb{I}_2>\mathbb{I}_3$
\item {Axi-Symmetric Rigid Body:} $\mathbb{I}_1=\mathbb{I}_2>\mathbb{I}_3$ or
$\mathbb{I}_1>\mathbb{I}_2=\mathbb{I}_3$
\item {Symmetric Rigid Body:} $\mathbb{I}_1=\mathbb{I}_2=\mathbb{I}_3$
\end{enumerate}

When the body frame $\mathbf{b}$ coincides with the directions of the principle moments of inertia of the object, the rotation dynamics (\ref{eq:EulerEqns3D}) take the form
\begin{align}
\mathbb{I}_1\dot{\Omega}_1 & =  (\mathbb{I}_2-\mathbb{I}_3)\Omega_2 \Omega_3 + T_1^e, \label{eq:AxiSym1}\\
\mathbb{I}_2\dot{\Omega}_2 & =  (\mathbb{I}_3-\mathbb{I}_1)\Omega_3 \Omega_1 + T_2^e, \label{eq:AxiSym2}\\
\mathbb{I}_3\dot{\Omega}_3 & =  (\mathbb{I}_1-\mathbb{I}_2)\Omega_1 \Omega_2 + T_3^e, \label{eq:AxiSym3}
\end{align}



%%%%%%%%%%%
\subsubsection{Purely rotating rigid body equation}
In the case where a point in the body is fixed with respect to the spatial frame $\mathbf{e}$ we will choose the body fixed frame $\mathbf{b}$ such that its origin coincides with this fixed point $O'$. Then if the moments are defined with respect to this fixed point $O'$ the above equations reduce to:
\begin{svgraybox}
In the $\mathbf{e}$-frame:
\begin{align*}
\dot{R}&=\widehat{\omega}R,\\
\dot{\pi}&=\tau_e,
\end{align*} 
where $\omega =(\mathbb{I}^R)^{-1}\pi$.
\\
\mbox{}
\\
In the $\mathbf{b}$-frame:
\begin{align*}
\dot{R}&=R\widehat{\Omega},\\
\mathbb{I}\dot{\Omega}&=\mathbb{I}\Omega \times {\Omega}+T^e.
\end{align*}
\end{svgraybox}

\subsubsection{Body angular momentum version of rotational rigid body equations}
When the moments are defined about the center of mass for a rotating and translating rigid body and about the pivot point for a purely rotating rigid body we see that $\pi=R\mathbb{I}\Omega=\mathbb{I}^R\omega$.
where $\mathbb{I}$ is the moment of inertia tensor about the center of mass for a rotating and translating rigid body and about the pivot point for a purely rotating rigid body and expressed with respect to a body fixed frame $\mathbf{b}$ whose origin coincides with the point around which the moments are defined.

Then we can also write the rotational dynamic equations using $\Pi=R^T\pi=\mathbb{I}\Omega$ as 
\begin{align*}
\dot{R}&=R\,\widehat{\mathbb{I}^{-1}\Pi},\\
\dot{\Pi}&=\Pi \times {\mathbb{I}^{-1}\Pi}+T^e_c.
\end{align*}


When the body frame $\mathbf{b}$ coincides with the directions of the principle moments of inertia of the object this equation takes the form
\begin{align}
\dot{\Pi}_1 & =  \frac{(\mathbb{I}_2-\mathbb{I}_3)}{\mathbb{I}_2 \mathbb{I}_3}\Pi_2 \Pi_3+ T_1^e, \label{eq:AxiSym1PiFree}\\
\dot{\Pi}_2 & =  \frac{(\mathbb{I}_3-\mathbb{I}_1)}{\mathbb{I}_3 \mathbb{I}_1}\Pi_3 \Pi_1+ T_2^e, \label{eq:AxiSym2PiFree}\\
\dot{\Pi}_3 & =  \frac{(\mathbb{I}_1-\mathbb{I}_2)}{\mathbb{I}_1 \mathbb{I}_2}\Pi_1 \Pi_2+ T_3^e. \label{eq:AxiSym3PiFree}
\end{align}

%%%%%%%%%%%%%%%%%%%%%%%%
\begin{svgraybox}
It is important to note that, since $\Pi=\mathbb{I}\Omega$, in general the direction of the body angular momentum does not coincide with the direction of the body angular velocity.
For a rigid body that is moving freely in space (ie. isolated from the rest of the universe) the total spatial angular momentum $\pi$ is always constant (since $\dot{\pi}=\tau^e=0$). However, we note that, since $\Pi=R^T\pi$ the body angular momentum, $\Pi$, is not conserved. Nevertheless since $||\Pi||=||R^T\pi||=||\pi||$ we see that the magnitude of the body angular momentum is still conserved for freely rotating rigid body motion.
\end{svgraybox}
When the set of particles of the rigid body are not isolated from the rest of the universe we find that
\begin{align}\label{eq:RateMagPi}
\frac{d }{dt}||\Pi(t)||=\frac{d}{dt}\sqrt{\Pi \cdot \Pi}=\frac{{\Pi \cdot \dot{\Pi}}}{\sqrt{\Pi \cdot \Pi}}=\frac{{\Pi \cdot (\Pi\times \mathbb{I}^{-1}\Pi+T^e)}}
{\sqrt{\Pi \cdot \Pi}}=\frac{{\Pi \cdot T^e}}{||\Pi||},
\end{align}
where the last equality follows from
$
A \cdot (A \times B) = 0.
$



%%%%%%%%%%%%



%%%%%%%%%%%%%%%%%%%%%%%%%%%%%

%%%%%%%%%%%%%%%%%%%%%%%%%%%%%



\begin{svgraybox}
Thus we conclude that for an isolated rigid body  $\pi$ and $||\Pi||$ remain a constant or in other words are conserved quantities of the motion. Notice that this too is a consequence of the principle of conservation of linear momentum.
\end{svgraybox}



%%%%%%%%%%%%%%%%%%%%%%%%%%%%%
\subsection{Kinetic Energy of a Rigid Body}
Recall that in section-\ref{Secn:KineticEnergyMF} we showed that (\ref{eq:KineticEnergyOfaParticleFixedInRotFrame}) gives the kinetic energy of a particle $P_i$ in the  frame $\mathbf{b}$  by
\begin{align*}
\mathrm{KE}_i&=\frac{1}{2} \left( m||V_o||^2+2m_iV_o^T\widehat{\Omega} X_i+\Omega^T\mathbb{I}_i\Omega
\right),
\end{align*}
where 
\begin{align*}
\mathbb{I}_i &\triangleq m_i\left(||X_i||^2I_{3\times 3} - X_iX_i^T\right)
\end{align*} 
was termed the \textit{moment of inertia} of the particle $P_i$ about the origin of the frame $\mathbf{b}$. 

For a set of particles where the relative distance between any two particles remain the same (what we call a rigid body) as viewed in a frame $\mathbf{b}$ (what we will call a frame fixed to the rigid body) the total kinetic energy of all the particles are then given by
summing up the above expression over all the particles we have,
\begin{align}
\mathrm{KE}&=\frac{1}{2} \sum_{p\in \mathcal{B}}\left( m||V_o||^2+2mV_o^T\widehat{\Omega} X_i+\Omega^T\mathbb{I}_i\Omega\right),\nonumber\\
&= \frac{1}{2}\left(M||V_o||^2+2V_o^T\widehat{\Omega} \sum_{p\in \mathcal{B}}\left(mX_i\right)+\Omega^T\mathbb{I}\Omega \right)\nonumber\\
&= \frac{1}{2}\left(M||V_o||^2 +2MV_o^T\widehat{\Omega} \bar{X}+\Omega^T\mathbb{I}\Omega\right)\label{eq:KineticEnergyOfaParticleFixedInRotFrame222}
\end{align}
where $M=\sum_{i=1}^nm_i$ is the total mass, $\mathbb{I}=\sum_{i=1}^n\mathbb{I}_i$ is the total moment of inertia of the body and $\bar{X}$ is the representation of the center of mass of the body in the body frame $\mathbf{b}$.
Notice that in the special case where the body fixed frame $\mathbf{b}$ is fixed at the center of mass of the rigid body we have that $\bar{X}=0$ and hence that the following holds. 
\begin{svgraybox}
If the origin of a body fixed frame of a rigid body is fixed at the center of mass of the rigid body then the kinetic energy of the rigid body with respect to an inertial frame $\mathbf{e}$ is given by
\begin{align}
\mathrm{KE}
&= \frac{1}{2}\left(M||V_o||^2 +\Omega^T\mathbb{I}\Omega\right)= \frac{1}{2}\left(M||\dot{o}||^2 +\Omega^T\mathbb{I}\Omega\right),\label{eq:KineticEnergyOfaParticleFixedInRotFrame333}
\end{align}
where the body frame $\mathbf{b}$ is related to the inertial frame $\mathbf{e}$ in such a way that $\mathbf{b}=\mathbf{e}\,R$ and the origin of $\mathbf{b}$ has the representation $o$ with respect to the inertial frame $\mathbf{e}$. Furthermore in the second equality we have also used the fact that $\dot{o}=RV_o$ and hence that $||V_o||=||\dot{o}||$. What this says is that the kinetic energy of a rigid body is equal to the sum of its rotational kinetic energy and the center of mass kinetic energy with respect to a frame $\mathbf{b}$ that is fixed on the body with center coinciding with the center of mass of the body.
\end{svgraybox}


Let us now consider the rate of change of kinetic energy of a rigid body:
\begin{align}
\frac{d }{dt}\mathrm{KE}&=\frac{1}{2}\left(M\dot{o}^T\ddot{o}+
M\ddot{o}^T\dot{o}+{\dot{\Omega}^T \mathbb{I}{\Omega}+\Omega^T \mathbb{I}\dot{\Omega}}\right)
=M\dot{o}^T\ddot{o}+\Omega^T \mathbb{I}\dot{\Omega}\nonumber\\
&=\dot{o}\cdot M\ddot{o}+\Omega \cdot \mathbb{I}\dot{\Omega}=\dot{o}\cdot f^e+\Omega 
\cdot (\mathbb{I}{\Omega}\times \Omega+T^e)\nonumber\\
&=\dot{o}\cdot f^e+\Omega \cdot T^e\label{eq:RateOfKE}
\end{align}
Where the last equality $\Omega \cdot (\mathbb{I}{\Omega}\times \Omega+T^e)=\Omega \cdot T$ follows from the easily verifiable property
of cross products and dot products
$
A \cdot (A \times B) = 0.
$
What this says is that the rate of change of kinetic energy of a rigid body is equal to the input \textit{power} of the system given by $\dot{o}\cdot f^e+\Omega \cdot T$. Thus if there are no external forces acting on the particles of the rigid body (that is if we assume that the interactions that the particles of the rigid body have with the rest of the Universe is negligible) then the power is zero and the kinetic energy of the rigid body is conserved.
%%%%%%%%%%%%

\subsubsection*{Example: Kinetic Energy of a Falling and Rolling Disk}\label{Secn:FallingRollingDisk}
Let $\mathbf{b}$ be a frame fixed on the disk and let $\mathbf{e}$ be an Earth fixed frame. The rotation matrix $R$ that relates the two frames by, $\mathbf{b}=\mathbf{e}\,R$, can be parameterized using the 3-1-2 Euler 
angles $R=R_3(\theta)R_1(\alpha)R_2(\phi)$ and thus
{\small
\[R=
\left[ \begin {array}{ccc} \cos \theta \cos \phi -\sin \theta \sin \alpha \sin \phi \:\:\:\:\:&\:\:\:\:\:-\sin \theta \cos \alpha \:\:\:\:\:&\:\:\:\:\:\cos \theta \sin \phi +\sin \theta \sin \alpha \cos \phi \\
\noalign{\bigskip}\sin \theta \cos \phi +\cos \theta \sin \alpha \sin \phi &\cos \theta \cos \alpha &\sin \theta \sin \phi -\cos \theta \sin \alpha \cos \phi \\\noalign{\bigskip}-\cos \alpha \sin \phi 
&\sin \alpha &\cos \alpha \cos \phi \end {array} \right]
\]
}
Calculating $\widehat{\Omega}=R^T\dot{R}$ we have that
the body angular velocities of the coin are given by:
\begin{eqnarray}
{\Omega}_1 & = & \dot{\alpha}\cos{\phi}-\dot{\theta}\cos{\alpha}\sin{\phi}, \label{eq:CoinBodyAngVel1}\\
{\Omega}_2 & = & \dot{\phi}-\dot{\theta}\sin{\alpha}, \label{eq:CoinBodyAngVel2}\\
{\Omega}_3 & = & \dot{\theta}\cos{\alpha}\cos{\phi}+\dot{\alpha}\sin{\phi}. \label{eq:CoinBodyAngVel3}
\end{eqnarray}

The velocity of the center of mass of the disk expressed in the $\mathbf{e}$-frame is
\[
\dot{o}=\left[\begin{array}{c} \dot{x}\\\dot{y} \\ -r\dot{\alpha}\sin{\alpha}
\end{array}\right]
\]
Let $\mathbb{I}=\mathrm{diag}(\mathbb{I}_i,\mathbb{I}_r,\mathbb{I}_i)$ be the moment of inertia tensor of the disk and $M$ be the mass of the disk. Then the kinetic energy of the falling rolling disk is
\begin{align*}
\mathrm{KE}&=\frac{1}{2}\left(M||\dot{o}||^2+\mathbb{I}\Omega \cdot \Omega\right)\\
&=\frac{1}{2}\left(M\dot{x}^2+M\dot{y}^2+\mathbb{I}_r\dot{\phi}^2+(\mathbb{I}_i+Mr^2\sin^2{\alpha})\dot{\alpha}^2
+(\mathbb{I}_i\cos^2{\alpha}+\mathbb{I}_r\sin^2{\alpha})\dot{\theta}^2-2\mathbb{I}_r\dot{\phi}\dot{\theta}\sin{\alpha}\right).
\end{align*}

\subsubsection*{Example: Kinetic Energy of a A 3-DOF Robot Arm} \label{Secn:3DOF_RobotArm}
\begin{figure}[ht]
\begin{center}
\includegraphics[width=2.6in]{TwoLink}
\renewcommand{\baselinestretch}{1}\selectfont
\caption{A schematic representation of 3-DOF robot arm.}
\label{Fig:3DOF_RobotArm}
\end{center}
\end{figure}

Let $\mathbf{e,a,b,c}$ be three orthonormal frames such that $\mathbf{e}$ is earth fixed at the point $O_1$, origin of $\mathbf{a}$ is at $O_1$ and $\mathbf{a}=\mathbf{e}R_a$, origin of $\mathbf{b}$ is at $O_1$ and $\mathbf{b}=\mathbf{a}R_b$, and origin of $\mathbf{c}$ is at $O_2$ (the pivot point of links 1 and 2) and $\mathbf{c}=\mathbf{b}R_c$. Let $G_1,G_2$ be the center of mass of the two linkages. Let $O_1G_1=\mathbf{b}X_{g_1}$ and $O_2G_2=\mathbf{c}X_{g_2}$. Let $O_1O_2=\mathbf{b}X_{o_2}$  and Let $O_2P=\mathbf{c}X_{cp}$.  Let $O_1P=\mathbf{e}x_p$.

$R_a=R_3(\alpha),R_b=R_1(\theta),R_c=R_1{\phi}$, $X_{g_1}=[0\:\:\:L_2\:\:0]^T$, $X_{g_2}=[0\:\:\:L_4\:\:0]^T$, $X_{o_2}=[0\:\:\:L_1\:\:0]^T$ and $X_{cp}=[0\:\:\:L_3\:\:0]^T$.
Then
\[
\mathbf{e}x_p=\mathbf{b}X_{o_2}+\mathbf{c}X_{P}
\]
\[
x_p=R_{L_1}\,X_{o_2}+R_{L_2}\,X_{cp}
\]
where $R_{L_1}=R_aR_b$ and $R_{L_2}=R_aR_bR_c$.
\[
x_{g_1}=R_{L_1}X_{g_1}
\]
\[
x_{g_2}=R_{L_1}\,X_{o_2}+R_{L_2}X_{g_2}
\]
\[
\dot{R}_{L_1}=R_{L_1}\widehat{\Omega}_{L_1}
\]
\[
\dot{R}_{L_2}=R_{L_2}\widehat{\Omega}_{L_2}
\]
\[
\dot{x}_{g_1}=R_{L_1}\widehat{\Omega}_{L_1}X_{g_1}
\]
\[
\dot{x}_{g_2}=R_{L_1}\widehat{\Omega}_{L_1}X_{g_1}+R_{L_2}\widehat{\Omega}_{L_2}X_{g_2}
\]

\[
\dot{x}_{g_1}=\left[\begin{array}{c}
L_2\dot{\theta}\sin(\alpha)\sin(\theta) - L_2\dot{\alpha}\cos(\alpha)\cos(\theta)\\
- L_2\dot{\alpha}\sin(\alpha)\cos(\theta) - L_2\dot{\theta}\cos(\alpha)\sin(\theta)\\
L_2\dot{\theta}cos(\theta)
\end{array} \right]
\]
{\small
\[
\dot{x}_{g_2}=\left[\begin{array}{c}
\dot{\theta}\sin(\alpha)(L_4\sin(\phi + \theta) + L_1\sin(\theta)) - \dot{\alpha}\cos(\alpha)(L_4\cos(\phi + \theta) + L_1\cos(\theta)) + L_4\dot{\phi}\sin(\phi + \theta)\sin(\alpha)\\
- \dot{\alpha}\sin(\alpha)(L_4\cos(\phi + \theta) + L_1\cos(\theta)) - \dot{\theta}\cos(\alpha)(L_4\sin(\phi + \theta) + L_1\sin(\theta)) - L_4\dot{\phi}\sin(\phi + \theta)\cos(\alpha)\\
\dot{\theta}(L_4\cos(\phi + \theta) + L_1\cos(\theta)) + L_4\dot{\phi}\cos(\phi + \theta)
\end{array} \right]
\]
}
\[
\dot{\Omega}_{L_1}=\left[\begin{array}{c}
        \dot{\theta}\\
 \dot{\alpha}\sin(\theta)\\
 \dot{\alpha}\cos(\theta)
\end{array} \right]\:\:\:\:\:
\dot{\Omega}_{L_2}=\left[\begin{array}{c}
        \dot{\phi} + \dot{\theta}\\
 \dot{\alpha}\sin(\phi + \theta)\\
 \dot{\alpha}\cos(\phi + \theta)
\end{array} \right]
\]
The the kinetic energy of the robotic arm is given by
\[
\mathrm{KE}=\frac{1}{2}M_1||\dot{x}_{g_1}||^2+\frac{1}{2}M_2||\dot{x}_{g_1}||^2+\frac{1}{2}\mathbb{I}_{L_1}\Omega_{L_1}\cdot\Omega_{L_1}+\frac{1}{2}\mathbb{I}_{L_2}\Omega_{L_2}\cdot\Omega_{L_2},
\]
where $\mathbb{I}_{L_1}$ and $\mathbb{I}_{L_2}$ are the inertia tensors of the two links in their respective body frames fixed at their center of masses and parallel to $\mathbf{b}$ and $\mathbf{c}$ respectively.

%%%%%%%%%%%%%%%%%%%%%%%%%%


%%%%%%%%%%%%%%%%%%%%%%%%




%%%%%%%%%%%%%%%%%%%%%%%%%
%\section{Examples of Rigid Body Motion}



\subsection{Free Rigid Body Motion}\label{Secn:FreeRotating}


In this section we will analyze the dynamics (the evolution over time) of a free rotating rigid body. What we mean by free is that no external force moments are present. That is $T=0$. We have seen that it is always possible to pick a body frame $\mathbf{b}$ such that the resulting moment of inertia tensor $\mathbb{I}$ is diagonal. Thus, in what follows, without loss of generality we will consider $\mathbf{b}$ to be such a frame. 

When $T=0$ we have also seen that 
the Kinetic Energy $\mathrm{KE}$, the spatial angular momentum, $\pi$, and the magnitude of the body angular momentum, $||\Pi||$, are conserved. 
That is for free rigid body rotations when we have that:
\begin{svgraybox}
\begin{align}
\pi(t) &= R(t)\left[
\begin{array}{c}
\mathbb{I}_1\Omega_1\\\mathbb{I}_2\Omega_2\\\mathbb{I}_3\Omega_3
\end{array}
\right]=R(t)\left[
\begin{array}{c}
\Pi_1\\ \Pi_2\\ \Pi_3
\end{array}
\right]=\pi=\mathrm{constant}\:\: 3 \times 1 \:\: \mathrm{matrix},\label{eq:piConst}\\
\mathrm{KE}&=\frac{1}{2}(\mathbb{I}_1\Omega_1^2+\mathbb{I}_2\Omega_2^2+\mathbb{I}_3\Omega_3^2) =\frac{1}{2}\left(\frac{\Pi_1^2}{\mathbb{I}_1}+\frac{\Pi_2^2}{\mathbb{I}_2}+\frac{\Pi_3^2}{\mathbb{I}_3}\right)= E=\mathrm{constant},
\label{eq:KEConst0}\\
||\Pi(t)||^2 &= \mathbb{I}^2_1\Omega_1^2+\mathbb{I}^2_2\Omega_2^2+\mathbb{I}^2_3\Omega_3^2=\Pi_1^2+\Pi_2^2+\Pi_3^2=h^2=\mathrm{constant}.\label{eq:PiConst0}
\end{align}
\end{svgraybox}
Notice that the above conservation laws are more conveniently represented using the body angular momenta $\Pi$. Thus we will consider the free rotational rigid body equations in the body angular momentum variables given by
\begin{eqnarray}
\dot{R} & = & R \;\widehat{\mathbb{I}^{-1}\Pi}, \label{eq:KinematicsPiFree}\\
\dot{\Pi} & = & \Pi \times \mathbb{I}^{-1}\Pi. \label{eq:EulerEqnsPiFree}
\end{eqnarray}
In a frame $\mathbf{b}$ where the inertia tensor $\mathbb{I}$ is diagonal we have seen that these equations take the form
\begin{align}
\dot{\Pi}_1 & =  \frac{(\mathbb{I}_2-\mathbb{I}_3)}{\mathbb{I}_2 \mathbb{I}_3}\Pi_2 \Pi_3, \label{eq:AxiSym1PiFree}\\
\dot{\Pi}_2 & =  \frac{(\mathbb{I}_3-\mathbb{I}_1)}{\mathbb{I}_3 \mathbb{I}_1}\Pi_3 \Pi_1, \label{eq:AxiSym2PiFree}\\
\dot{\Pi}_3 & =  \frac{(\mathbb{I}_1-\mathbb{I}_2)}{\mathbb{I}_1 \mathbb{I}_2}\Pi_1 \Pi_2. \label{eq:AxiSym3PiFree}
\end{align}
\begin{svgraybox}
From these equations we see that ${\Pi}(t)\equiv \mu_1\triangleq h [1\:\:0\:\:0]^T$, $\bar{\Pi}(t)\equiv \mu_2\triangleq h [0\:\:1\:\:0]^T$, and $\bar{\Pi}(t)\equiv \mu_3\triangleq h [0\:\:0\:\:1]^T$ where $h=||\Pi||$ are equilibrium solutions of the rigid body equations (\ref{eq:AxiSym1PiFree}) -- (\ref{eq:AxiSym3PiFree}). Such solutions are called \emph{relative equilibria} of motion. Since $\Omega=\mathbb{I}^{-1}\Pi$ we see that these relative equilibria correspond to steady rotations of the body about its three principle axis.
\end{svgraybox}


Equation (\ref{eq:KEConst0}) defines an ellipsoid, called the \emph{energy ellipsoid}, in the body angular momentum space and (\ref{eq:PiConst0}) defines the surface of a sphere, called the \emph{angular momentum} sphere, in body angular momentum space. 
Conservation of kinetic energy implies that the solutions of (\ref{eq:AxiSym1PiFree}) -- (\ref{eq:AxiSym3PiFree}) must lie on the energy ellipsoid and the conservation of the magnitude of the body angular momentum implies that the solutions of (\ref{eq:AxiSym1PiFree}) -- (\ref{eq:AxiSym3PiFree}) must lie on the angular momentum sphere. 
\begin{svgraybox}
Thus we conclude that the solutions of (\ref{eq:AxiSym1PiFree}) -- (\ref{eq:AxiSym3PiFree}) correspond to the intersection curves of the corresponding angular momentum sphere with the energy momentum ellipsoid.  
\end{svgraybox}
It should be pointed out that a solution being periodic in angular momentum $\Pi$ space does not imply that the corresponding motion of the rigid body is also periodic. To see this 
consider the following.
Let $\tau$ be the periodicity of the solution in $\Pi$ space. That is $\Pi(\tau)=\Pi(0)$. Then from the conservation of spatial angular momentum, $\pi(t)\equiv \mu$ a constant, we have that $R^T(0)\mu=\Pi(0)=
\Pi(\tau)=R^T(\tau)\mu$ and hence $\mu=R(\tau)R^T(0)\mu$. Thus we see that $R(\tau)R^T(0)=R_{\mu}$ is a rotation about the $\mu$ axis\footnote{Recall that if $R$ 
is a rotation about the axis with representation $\mu$ in the spatial frame then $R\mu=\mu$.} and hence that in general $R(\tau)\neq R(0)$. We can write
\begin{align}
R(\tau)R^T(0)=\exp{\left(\theta\widehat{n}\right)}
\end{align}
where $n=\mu/||\mu||$. The rotation angle $\theta$ of $R(\tau)R^T(0)$ about $\mu$ is called the \emph{phase} of the rotation.

It is interesting to note that 
\begin{align}
\mathrm{KE}=\Pi(t)\cdot\Omega(t)=\pi(t)\cdot \omega(t)=\mu\cdot \omega(t)=\mathrm{constant}.
\end{align}
Thus the angle between the angular velocity and the constant angular momentum remains constant through out the motion. What one can observe is the motion of body fixed directions.  Let $\Upsilon$ be some such body fixed direction. For instance it could be the third principle direction of the body. We are interested in visualizing how the body moves with respect to the fixed spatial angular momentum direction. That is we would like to know how $R\Upsilon$ is related to $\pi$.
Since $\mu \cdot R\Upsilon=R\Pi \cdot R\Upsilon=\Pi \cdot \Upsilon$ we see that, when observed in the spatially fixed frame $\mathbf{e}$, the angle that the body fixed $\Upsilon$ axis of the rigid 
body makes with the constant spatial angular momentum vector $\mu$ is equal to the angle that this axis makes with the body angular momentum $\Pi$. 
\begin{svgraybox}
This angle  
\begin{align}
\alpha(t)=\mathrm{cos}^{-1}\left(\frac{\Upsilon \cdot \Pi}{h}\right),\label{eq:NutationAngle}
\end{align}
between the body fixed axis $\Upsilon$ and the constant spatial angular momentum $\pi$ is defined as the \textit{angle of nutation} of the axis $\Upsilon$.
\end{svgraybox}

The nature of the intersection curves obviously depend on the the three principle moments of inertia $\{\mathbb{I}_1,\mathbb{I}_2,\mathbb{I}_3\}$. For instance in the case of a perfectly spherical object,  $\mathbb{I}_1=\mathbb{I}_2=\mathbb{I}_3$, and as such the energy ellipsoid is also a sphere. Thus the intersection curves in this case degenerate to points on the sphere and all trajectories correspond to steady rotations about some axis. This is also evident from (\ref{eq:AxiSym1PiFree}) -- (\ref{eq:AxiSym3PiFree}) since we see that they reduce to $\dot{\Pi}=0$.
On the other hand if $\mathbb{I}_1>\mathbb{I}_2=\mathbb{I}_3$, as in the case of an axi-symmetric object such as a disk, the intersection curves are concentric ellipses and fixed points. The fixed points correspond to relative equilibria while the other intersection curves correspond to periodic orbits in body angular momentum space. We notice that in this case there exists two isolated relative equilibrium corresponding to the two poles of the energy ellipsoid along the major axis direction and continuum of relative equilibria coinciding with the minor circle of the energy ellipsoid. Notice that the two isolated relative equilibria are stable while the ones along the minor axis are all unstable. The same conclusions apply for a thin cylinder type axi-symmetric objects where $\mathbb{I}_1=\mathbb{I}_2>\mathbb{I}_3$.


Figure \ref{Fig:IntersectionEandPi} shows the intersection of an energy ellipsoid with two constant angular momentum spheres for the case more general case where $\mathbb{I}_1>\mathbb{I}_2>\mathbb{I}_3$. The two spheres are depicted in pink and light blue color. The radius of the pink sphere is greater than $\sqrt{2\mathbb{I}_2E}$ while the radius of the light blue sphere is less than $\sqrt{2\mathbb{I}_2E}$. The blue curve depicts the critical case where the radius of the sphere is equal to $\sqrt{2\mathbb{I}_2E}$ and these four intersection curves are not closed curves.



%%%%%%%%%%%%%%%%%%


\begin{figure}[ht]
\begin{center}
\includegraphics[width=3.2in]{AngularMomentumEllipsoid}
\renewcommand{\baselinestretch}{1}\selectfont
\caption{\href{https://en.wikipedia.org/wiki/File:Ellipsoid-KSE-4-5.svg}{\copyright Wikipedia}. The intersection of an energy ellipsoid with three constant angular momentum spheres is shown in the bottom figure for the case where $\mathbb{I}_1>\mathbb{I}_2>\mathbb{I}_3$. The three spheres are depicted by pink, blue, and light blue. The radius of the pink sphere is greater than $\sqrt{2\mathbb{I}_2E}$ while the radius of the light blue sphere is less than $\sqrt{2\mathbb{I}_2E}$. The blue curve depicts the critical case where the radius of the sphere is equal to $\sqrt{2\mathbb{I}_2E}$. }
\label{Fig:IntersectionEandPi}
\renewcommand{\baselinestretch}{1.5}\selectfont
\end{center}
\end{figure}


Using the above discussion let us investigate the qualitative behavior of a rigid body for a particular constant magnitude of the angular momentum sat $||\Pi||=h$. Figure \ref{Fig:AngularMomentumSphere} shows the intersection curves of different energy ellipsoids corresponding to different energy levels with angular momentum sphere $||\Pi||=h$. From this we see that the two relative equilibria corresponding to the body spinning along the major and minor principle axis are stable while the other relative equilibrium corresponding to the body spinning along the intermediate principle axis is unstable. All other trajectories except the four ones corresponding to the critical case are periodic trajectories in the body angular momentum space.

\begin{figure}[ht]
\begin{center}
\includegraphics[width=3.2in]{MomentumEllipsoids}
\renewcommand{\baselinestretch}{1}\selectfont
\caption{The figure shows several intersection curves of different energy ellipsoid with a fixed constant angular momentum sphere. Figure taken from \url{https://www.angelfire.com/rnb/pp0/poleshift.html}}
\label{Fig:AngularMomentumSphere}
\renewcommand{\baselinestretch}{1.5}\selectfont
\end{center}
\end{figure}

Since the trajectories in the body angular momentum space correspond to the intersection curves of the energy ellipsoid and the body angular momentum ellipsoid the trajectories can be explicitly written down or in other words the system of equations (\ref{eq:AxiSym1PiFree}) -- (\ref{eq:AxiSym3PiFree}) are  completely integrable. In the section below we find these analytical solutions for the free rigid body rotations.
%%%%%%%%%%%%%%%%%%%%%%
\subsection{Analytic Solutions of Free Rigid Body Rotations}\label{Secn:IntegrabilityFreeRotations}
Let us first consider the more general asymmetric rigid body, $\mathbb{I}_1>\mathbb{I}_2>\mathbb{I}_3$.
From the two conservation laws (\ref{eq:KEConst0}) and (\ref{eq:PiConst0}) we have that
\begin{eqnarray}
\Pi^2_1 &=&\alpha_{21}-\beta_{21} \Pi^2_2, \label{eq:Pi1}\\
\Pi^2_3 &=&\alpha_{23}-\beta_{23} \Pi^2_2,\label{eq:Pi2}
\end{eqnarray}
where
\[
\alpha_{21}=\frac{\mathbb{I}_1(h-2E\mathbb{I}_3)}{\mathbb{I}_1-\mathbb{I}_3},\:\:\:\:\alpha_{23}=\frac{\mathbb{I}_3(h-2E\mathbb{I}_1)}{\mathbb{I}_3-\mathbb{I}_1}
\]
and
\[
\beta_{21}=\frac{\mathbb{I}_1(\mathbb{I}_2-\mathbb{I}_3)}{\mathbb{I}_2(\mathbb{I}_1-\mathbb{I}_3)},\:\:\:\:\beta_{23}=\frac{\mathbb{I}_3(\mathbb{I}_2-\mathbb{I}_1)}{\mathbb{I}_2(\mathbb{I}_3-\mathbb{I}_1)}.
\]
Let $2E\mathbb{I}_3<h<2E\mathbb{I}_1$. Then both $\alpha_{21}>0$ and $\alpha_{23}>0$.
Since without loss of generality we have assumed that $\mathbb{I}_1>\mathbb{I}_2>\mathbb{I}_3$ then $\beta_{21}>0$ and $\beta_{23}>0$.

Using these (\ref{eq:AxiSym2PiFree}) can be written as
\[
\dot{\Pi}_2  =  \frac{(\mathbb{I}_1-\mathbb{I}_3)}{\mathbb{I}_1 \mathbb{I}_3}\sqrt{(\alpha_{21}-\beta_{21} \Pi^2_2)(\alpha_{23}-\beta_{23} \Pi^2_2)}.
\]
Using the variable transformation
\[u
\triangleq \sqrt{\frac{\beta_{21}}{\alpha_{21}}}\,\Pi_2,
\]
\[
\omega\triangleq \sqrt{\beta_{21}\alpha_{23}}\left(\frac{\mathbb{I}_1-\mathbb{I}_3}{\mathbb{I}_1\mathbb{I}_3}\right)=\sqrt{\frac{(\mathbb{I}_2-\mathbb{I}_3)(2E\mathbb{I}_1-h)}{\mathbb{I}_1\mathbb{I}_2\mathbb{I}_3}}
\]
and
\[
k^2\triangleq \frac{\alpha_{21}\beta_{23}}{\alpha{23}\beta_{21}}=\frac{(\mathbb{I}_1-\mathbb{I}_2)(h-2E\mathbb{I}_3)}{(\mathbb{I}_2-\mathbb{I}_3)(2E\mathbb{I}_1-h)}
\]
this reduces to
\[
\dot{u}  =  \omega\sqrt{(1-u^2)(1-k^2u^2)}.
\]
This can be explicitly integrated using quadrature as follows:
\[
\int \frac{1}{\sqrt{(1-u^2)(1-k^2u^2)}}\,du =  \int \omega \,dt=\omega t +c.
\]
The left hand side of the above integral is the inverse of the Jacobi elliptic function $\mathrm{sn}$ of modulus $k$ where $0\leq k \leq 1$,
\[
\mathrm{sn}^{-1}(u)=\int_0^u \frac{1}{\sqrt{(1-u^2)(1-k^2u^2)}}\,du.
\]
Thus we have
\[
u(t)=\mathrm{sn}(\omega t +c),
\]
and hence
\begin{align}
\Pi_2(t)&=\sqrt{\frac{\mathbb{I}_2(h-2E\mathbb{I}_3)}{(\mathbb{I}_2-\mathbb{I}_3)}}\;\mathrm{sn}(\omega t +c).
\label{eq:Pi22}\\\
\end{align}

From (\ref{eq:Pi1}) and (\ref{eq:Pi2})
we have that
\begin{align*}
\Pi^2_1 &=\alpha_{21}(1-u^2)=\alpha_{21}(1-\mathrm{sn}^2(\omega t +c))=\alpha_{21}\mathrm{cn}^2(\omega t +c), \\%\label{eq:Pi11}\\
\Pi^2_3 &=\alpha_{23}(1-k^2u^2)=\alpha_{23}(1-k^2\mathrm{sn}^2(\omega t +c))=\alpha_{23}\mathrm{dn}^2(\omega t +c),%\label{eq:Pi21}
\end{align*}
and hence
\begin{align}
\Pi_1(t) &=\sqrt{\frac{\mathbb{I}_1(h-2E\mathbb{I}_3)}{\mathbb{I}_1-\mathbb{I}_3}}\mathrm{cn}(\omega t +c), \label{eq:Pi12}\\
\Pi_3(t) &=\sqrt{\frac{\mathbb{I}_3(2E\mathbb{I}_1-h)}{\mathbb{I}_1-\mathbb{I}_3}}\mathrm{dn}(\omega t +c).\label{eq:Pi32}
\end{align}
\\
\\
Jacobi Elliptic functions sn and cn are periodic functions of period $4K$ and dn is periodic of period $2K$,
where
\[
K(k)=\mathrm{sn}^{-1}(1)=\int_0^1 \frac{1}{\sqrt{(1-u^2)(1-k^2u^2)}}\,du=\int_0^{\pi/2}
\frac{1}{\sqrt{1-k^2\sin^2{\psi}}}d\psi \approx \left(1+\frac{k^2}{4}\right)\frac{\pi}{2}.
\]
Thus $\Pi_1(t), \Pi_2(t), \Pi_3(t)$ are periodic of period
\[
T=4K(k)/\omega =\frac{2\pi}{\omega}\left(1+\frac{k^2}{4}+H.O.T.\right) \approx 2\pi \sqrt{\frac{\mathbb{I}_1\mathbb{I}_2\mathbb{I}_3}{(2E\mathbb{I}_1-h)(\mathbb{I}_2-\mathbb{I}_3)}}\left( 1+\frac{(\mathbb{I}_1-\mathbb{I}_2)(h-2E\mathbb{I}_3)}{4(\mathbb{I}_2-\mathbb{I}_3)(2E\mathbb{I}_1-h)} \right).
\]
Thus confirming that $\Pi(t)$ is periodic of period $T$. However recall that the motion of the rigid body is not periodic!!!
\\
\\
\begin{figure}[ht]
\begin{center}
\begin{tabular}{cc}
\includegraphics[width=2.8in]{NonSymmetricRigidBodySimu} & \includegraphics[width=2.8in]{NonSymmetricRigidBody_Largek_Simu}\\
(a) & (b)
\end{tabular}
\renewcommand{\baselinestretch}{1}\selectfont
\caption{Components of the angular momentum vector $\Pi (t)$ for the Non-Aix-Symmetric Rigid Body. $\mathbb{I}=\mathrm{diag}[3,2,1]$. Figure (a) corresponds to an initial condition of $
\Pi(0)=[0.6,-1,2]$ thus a $k=0.2951$ and $T=5.4$. Figure (b) corresponds to an initial condition of $\Pi(0)=[3,-1,2]$ thus $k=0.88$ and $T=7.34$.}
\label{Fig:NonSymmetricRigidBodySimu}
\renewcommand{\baselinestretch}{1.5}\selectfont
\end{center}
\end{figure}
Figure \ref{Fig:NonSymmetricRigidBodySimu} shows the simulation results of for an asymmetric rigid body of $\mathbb{I}=\mathrm{diag}[3,2,1]$.



\begin{figure}[ht]
\begin{center}
\includegraphics[width=4.2in]{BodyAngularMomentumNonSymmetric}
\renewcommand{\baselinestretch}{1}\selectfont
\caption{The motion of the tip of the minor axis for the asymmetric rigid body with $\mathbb{I}=\mathrm{diag}[3,2,1]$ and an initial conditions $\Pi(0)=[0.6,-1,2]$.}
\label{Fig:NonSymmetricRigidBodyNutation}
\renewcommand{\baselinestretch}{1.5}\selectfont
\end{center}
\end{figure}


Let us explicitly compute the  nutation angle $\alpha(t)$
between the body fixed axis $\Upsilon=[0\:\:0\:\:1]^T$ and the constant spatial angular momentum $\pi$ that is given by (\ref{eq:NutationAngle}) to be
\begin{align}
\alpha(t)=\mathrm{cos}^{-1}\left(\frac{\Upsilon \cdot \Pi}{h}\right)=\mathrm{cos}^{-1}\left(\frac{\Pi_3}{h}\right)
=\mathrm{cos}^{-1}\left(\sqrt{\frac{\mathbb{I}_3(2E\mathbb{I}_1-h)}{h^2(\mathbb{I}_1-\mathbb{I}_3)}}\mathrm{dn}(\omega t +c)\right).\label{eq:NutationAngleMinoraxis}
\end{align}
Figure \ref{Fig:NonSymmetricRigidBodyNutation} shows the The motion of the tip of the minor axis for the asymmetric rigid body with $\mathbb{I}=\mathrm{diag}[3,2,1]$ and initial conditions $\Pi(0)=[0.6,-1,2]$.
%%%%%%%%%%%%%%%%%%%%%%%%%%%%%
\subsection{Axi-symmetric Rigid Body}\label{Secn:AxiSymRigidBody}
For an axi-symmetric rigid body, $\mathbb{I}_1=\mathbb{I}_2>\mathbb{I}_3$. Then we see that $k=0$ and hence that the elliptic functions reduce to the trigonometric functions ($\mathrm{sn}\rightarrow \sin, \:\:\:\mathrm{cn}\rightarrow \cos,\:\:\: \mathrm{dn}\rightarrow 
1$). Thus we have
\[
\omega=\sqrt{\frac{(2E\mathbb{I}_1-h)(\mathbb{I}_2-\mathbb{I}_3)}{\mathbb{I}_1\mathbb{I}_2\mathbb{I}_3}},
\]
and
\begin{eqnarray}
\Pi_1(t) &=&\sqrt{\frac{\mathbb{I}_1(h-2E\mathbb{I}_3)}{\mathbb{I}_1-\mathbb{I}_3}}\cos(\omega\, t +c), \label{eq:Pi1_AxiSym}\\
\Pi_2(t) &=&\sqrt{\frac{\mathbb{I}_2(h-2E\mathbb{I}_3)}{(\mathbb{I}_2-\mathbb{I}_3)}}\;\sin(\omega\, t +c),\label{eq:Pi2_AxiSym}\\
\Pi_3(t) &=&\sqrt{\frac{\mathbb{I}_3(2E\mathbb{I}_1-h)}{\mathbb{I}_1-\mathbb{I}_3}}=\mathrm{constant}.\label{eq:Pi3_AxiSym}
\end{eqnarray}
\begin{figure}[ht]
\begin{center}
\includegraphics[width=4.6in]{AxiSymmetricRigidBodySimu}
\renewcommand{\baselinestretch}{1}\selectfont
\caption{Components of the angular momentum vector $\Pi (t)$ for the Symmetric Rigid Body. $\mathbb{I}=\mathrm{diag}[2,2,1]$ and $\Pi(0)=[0.4,-1,2]$. The period of oscillation is $T=6.23$.}
\label{Fig:AxiSymmetricRigidBodySimu}
\renewcommand{\baselinestretch}{1.5}\selectfont
\end{center}
\end{figure}
The body angular momentum $\Pi(t)$ is periodic of period
\[
T=2\pi \sqrt{\frac{\mathbb{I}_1\mathbb{I}_2\mathbb{I}_3}{(2E\mathbb{I}_1-h)(\mathbb{I}_2-\mathbb{I}_3)}}.
\]
Consider the minor axis of symmetry of the body $\Upsilon=[0\:\:\:0\:\:\:1]^T$ (as represented in the body frame).
The angle between the body angular momentum and the minor axis of symmetry of the body (\textit{angle of nutation of the minor axis}) is given by,
\[
\alpha=\mathrm{cos}^{-1}\left(\frac{\Upsilon \cdot \Pi}{h}\right)=\mathrm{cos}^{-1}\left(\frac{\Pi_3}{h}\right)
=\mathrm{cos}^{-1}\left(\sqrt{\frac{\mathbb{I}_3(2E\mathbb{I}_1-h)}{h^2(\mathbb{I}_1-\mathbb{I}_3)}}\right)=\mathrm{constant}
\]
and is a constant.



Figure \ref{Fig:AxiSymmetricRigidBodySimu} shows the simulation results of an symmetric rigid body of $\mathbb{I}=\mathrm{diag}[2,2,1]$ and an initial condition of $\Pi(0)=[0.6,-1,2]$.
The motion of the tip of the angular momentum vector as observed in the body co-ordinates for the symmetric Rigid Body is shown in figure 
\ref{Fig:AxiSymmetricRigidBodyBodyAngular}.\\ \begin{figure}[ht]
\begin{center}
\includegraphics[width=4.2in]{BodyAngularMomentumAxiSymmetric}
\renewcommand{\baselinestretch}{1}\selectfont
\caption{The motion of the tip of the minor axis for a symmetric rigid body with $\mathbb{I}=\mathrm{diag}[2,2,1]$ and an initial conditions 
$\Pi(0)=[0.6,-1,2]$.}
\label{Fig:AxiSymmetricRigidBodyBodyAngular}
\renewcommand{\baselinestretch}{1.5}\selectfont
\end{center}
\end{figure}
\\
The motion of the tip of the minor axis for a symmetric rigid body with $\mathbb{I}=\mathrm{diag}[2,2,1]$ and an initial conditions 
$\Pi(0)=[0.6,-1,2]$ is shown in figure \ref{Fig:AxiSymmetricRigidBodyBodyAngular}.



%%%%%%%%%%%%%%%%%%%%%%%%%%%%%%%%%

\subsection{The Axi-symmetric Heavy Top in a Gravitational Field}
In this section we consider the axi-symmetric top in a gravitational field. The pivot point is along the axis of symmetry. The distance from the pivot point to the center of mass is $l$. 
Then the force moment about the pivot point due to gravity is $T= mgl\;(R^T \Upsilon)\times \Upsilon$ where $\Upsilon$ is the direction of gravity as observed in the $e$ frame. Thus the equations of 
motion for the heavy top are
\begin{eqnarray}
\dot{R} &=& R \, \widehat{\Omega}\,, \label{eq:KinematicEqnsHeavyTop} \\
\mathbb{I}\dot{\Omega} &=&  \mathbb{I}\Omega \times \Omega + mgl\;(R^T \Upsilon)\times \Upsilon.
\label{eq:EulerLagrangeEqnsHeavyTop}
\end{eqnarray}
\begin{figure}[ht]
\begin{center}
\includegraphics[width=4in]{AxiSymmetricTop}
\renewcommand{\baselinestretch}{1}\selectfont
\caption{The Axi-symmetric Heavy Top. Figure copied from \cite{Greenwood}.}
\label{Fig:HeavyTop}
\renewcommand{\baselinestretch}{1.5}\selectfont
\end{center}
\end{figure}
To implement (\ref{eq:KinematicEqnsHeavyTop}) and (\ref{eq:EulerLagrangeEqnsHeavyTop})
we need to parameterize the rotation matrix $R(t)$.
Let $\mathbf{a}$ be a fixed frame and $\mathbf{b}$ be a frame fixed on the top as shown in figure \ref{Fig:HeavyTop}. The configuration of the top is given by the rotation matrix $R$ where $\mathbf{b}(t)=\mathbf{a}\,R(t)
$. Using intermediate frames $\mathbf{a}'(t)$ and $\mathbf{f}(t)$ we can parameterize $R(t)$ using the 3-1-3 Euler angles as follows. As shown in the figure we have that
\[
\mathbf{a}'(t)=\mathbf{a}\,R_3(\phi),\:\:\:\:\mathbf{f}(t)=\mathbf{a}'(t)\,R_1(\theta),\:\:\:\:\mathbf{b}(t)=\mathbf{f}(t)\,R_3(\psi),\:\:\:\:
\]
Then if $\mathbf{b}(t)=\mathbf{a}\,R(t)$
\[
R(t)=R_3(\phi)R_1(\theta)R_3(\psi).
\]
In this parameterization $\phi$ is called the angle of \textit{precession}, $\theta$ is called the angle of \textit{nutation}, and $\psi$ is called the angle of \textit{spin}.

It is interesting to note that if we let $\Gamma(t)=R^T\,\Upsilon$ (the direction of gravity as seen in the body frame) and differentiate it we can replace 
(\ref{eq:KinematicEqnsHeavyTop}) and (\ref{eq:EulerLagrangeEqnsHeavyTop}) by the equivalent set of equations,
\begin{eqnarray}
\dot{\Gamma} &=& \Gamma \times {\Omega}\,, \label{eq:KinematicEqnsHeavyTop} \\
\mathbb{I}\dot{\Omega} &=&  \mathbb{I}\Omega \times \Omega + mgl\;\Gamma \times \Upsilon.
\label{eq:EulerLagrangeEqnsHeavyTop}
\end{eqnarray}
Observe that they remove the need for Euler angles.

It can be shown that the following quantities are conserved along the solutions of the system.
\begin{eqnarray}
&& KE=\frac{1}{2}(\mathbb{I}_1\Omega_1^2+\mathbb{I}_2\Omega_2^2+\mathbb{I}_3\Omega_3^2)+mgl\,\Gamma^T \Upsilon =\frac{1}{2}\left(\frac{\Pi_1^2}{\mathbb{I}_1}+\frac{\Pi_2^2}{\mathbb{I}_2}+\frac{\Pi_3^2}{\mathbb{I}_3}\right)+mgl\,
\Gamma_3,\label{eq:KEConst}\\
&& \Pi^T\Gamma = \Pi_1\Gamma_1+\Pi_2\Gamma_2+\Pi_3\Gamma_3,\\
&& ||\Gamma||^2=\Gamma^2_1+\Gamma^2_2+\Gamma^2_3.
\end{eqnarray}

A heavy top for which $\mathbb{I}_1=\mathbb{I}_2$ is called a Lagrange top. An upright spinning Lagrange top is stable  if and only if $||\Omega||>2\sqrt{Mgl\mathbb{I}_1}/\mathbb{I}_3$ (see Section 15.10 of \cite{Marsden} for further details).



%%%%%%%%%%%%%%%%%%%%



%%%%%%%%%%%%%%%%%
\section{The Space of Rotations $SO(3)$}\label{Secn:SO3}
Up to now we have seen that much about rigid body motion can be understood without explicitly having to worry about the rotational kinematics $\dot{R}=R\widehat{\Omega}$. This is so because in all the cases we have considered so far the external moments are independent of the rotations. In general this is not so. Furthermore even in these cases where the external moments do not depend on the configuration of the body we are still interested in keeping track of the configuration of the body over time. This involves the integration of $\dot{R}=R\widehat{\Omega}=\widehat{\omega} R$. 
In what follows we will concern ourselves with this integration that will tell us what the configuration of the rigid body is at every given time instant. If one considers element by element then $\dot{R}=R\widehat{\Omega}=\widehat{\omega} R$ represents nine time varying ODEs. However since $R^TR=I_{3\times 3}$ defines six constraints we see that all the nine differential equations defined by $\dot{R}=R\widehat{\Omega}=\widehat{\omega} R$ are not independent. Thus it is important to know how to find independent equations to describe these kinematics.
This brings us to to the question of parameterizing rotations or in other words finding suitable coordinates for $SO(3)$ and thereby representing $\dot{R}=R\widehat{\Omega}=\widehat{\omega} R$ in terms of these parameterizations.

In section-\ref{Secn:PropertiesOfRotations} we saw that if $R$ can be thought of as a map that transforms points in space to other points in space in such a way that it preserves distances between points and angles between lines. That is, every $R\in \mathrm{SO}(3)$ can be viewed as a rigid rotation of space. Conversely, since we have seen that two frames with coinciding origins are related by $\mathbf{b}=\mathbf{e}\,R$, every rigid body rotation about a fixed point can also be identified with an $R\in \mathrm{SO}(3)$ by fixing an orthonormal frame $\mathbf{b}$ on the rigid body and finding its relationship with some inertial frame-$\mathbf{e}$. In what follows we will take a closer look at all possible rigid body rotations. Or in other words we will try to get some insight about the structure of the space $\mathrm{SO}(3)$.


Since $R^TR=I_{3\times 3}$ we see that all 9 elements of the ${3\times 3}$ matrix $R$ are not independent. The condition $R^TR=I_{3\times 3}$ gives rise to six constraints on the 9 elements of $R$. Thus we see that there are only 3 independent elements in $R$. Therefore we can conclude that in order to express $R$ we need to know at least three parameters. This implies that the space $\mathrm{SO}(3)$ is a 3-dimensional space. However we will see below that $SO(3)$ is quite different from the the 3 dimensional vector space $\mathbb{R}^3$ and thus that there exists no way of globally parameterizing $SO(3)$ using only three parameters.

%%%%%%%%%%%%%%%%%%%%%%%%%%%%%%%%
\subsection{Non Commutativity of Rotations}
In $\mathbb{R}^3$ the binary operation of addition that makes it a vector space is commutative.  Below we will see that the space of rotations, $SO(3)$, even though is closed under the composition of rotations is not a vector space under composition of rotations since the composition of rotations is not commutative. This indicates a quite significant difference with $\mathbb{R}^3$.

To see this let us first consider composition of rotations by considering three frames $\mathbf{e}$, $\mathbf{a}$ and $\mathbf{b}$. Let $R_{\alpha},R_{\beta}\in \mathrm{SO}(3)$. 
Consider the two rotations of space that take the frame $\mathbf{e}$ to $\mathbf{a}$ and and the frame $\mathbf{a}$ to $\mathbf{b}$ respectively such that $\mathbf{a}=\mathbf{e}R_{\alpha}$ and $\mathbf{b}=\mathbf{a}R_{\beta}$. Since $\mathbf{b}=\mathbf{a}R_{\beta}=\mathbf{e}R_{\alpha}R_{\beta}$ we see that $R=R_{\alpha}R_{\beta}$ is a rotation that takes $\mathbf{e}$ to $\mathbf{a}$ by a rotation $R_{\alpha}$ and then to $\mathbf{b}$ by a rotation $R_{\beta}$. Thus composition of rotations correspond to right multiplication by the corresponding rotation matrix. 
Also consider the two rotations of space that takes the frame $\mathbf{e}$ to $\mathbf{a}'$ and and the frame $\mathbf{a}'$ to $\mathbf{b}'$ respectively such that $\mathbf{a}'=\mathbf{e}R_{\beta}$ and $\mathbf{b}'=\mathbf{a}'R_{\alpha}$. Since $\mathbf{b}'=\mathbf{a}'R_{\alpha}=\mathbf{e}R_{\beta}R_{\alpha}$ we see that $R=R_{\beta}R_{\alpha}$ is a rotation that takes $\mathbf{e}$ to $\mathbf{a}'$ by a rotation $R_{\beta}$ and then to $\mathbf{b}'$ by a rotation $R_{\alpha}$. Since in general matrix multiplication is non-commutative $R_{\alpha}R_{\beta}\neq R_{\beta}R_{\alpha}$ and therefore the final frames in the two cases above $\mathbf{b}$ and $\mathbf{b}'$ will not be the same. 
\begin{svgraybox}
Thus in general, a rotation by $R_{\alpha}$ followed up by a rotation $R_{\beta}$ will not be the same as a rotation by $R_{\beta}$ followed up by a rotation $R_{\alpha}$. That is rotations do not commute.
\end{svgraybox}

Let us illustrate this non-commutatitivity of rotations using some specific easy to visualize  examples.
Now consider a counter clockwise rotation first about the third axis by an angle equal to $\pi/2$ and then about the first axis by an angle equal to $\pi/2$. The resultant frame 
$\mathbf{b}$ is then related to the original frame $\mathbf{e}$ by
$\mathbf{b}=\mathbf{e}(R_3(\pi/2)R_1(\pi/2))$. When you reverse the oder of rotation the resultant frame $\mathbf{b}'$ is then related to the original frame $\mathbf{e}$ by
$\mathbf{b}'=\mathbf{e}(R_1(\pi/2)R_3(\pi/2))$.
Since
\begin{align*}
R_1(\pi/2)R_3(\pi/2)&=\begin{bmatrix}0&-1&0\\0&0&-1\\1&0&0\end{bmatrix}\\
R_3(\pi/2)R_1(\pi/2)&=\begin{bmatrix}0&0&1\\1&0&0\\0&1&0\end{bmatrix}
\end{align*}
we clearly see that $\mathbf{b}'\neq\mathbf{b}$. This is illustrated in figure-\ref{Fig:RotatingBook}
\begin{figure}[ht]
\begin{center}
\includegraphics[width=3.5in]{RotatingBook} 
\renewcommand{\baselinestretch}{1}\selectfont
\caption{\href{http://www.sciencebits.com/vector_algebra}{\copyright ScienceBits}. The non-commutativity of rigid body rotations.}
\label{Fig:RotatingBook}
\renewcommand{\baselinestretch}{1.5}\selectfont
\end{center}
\end{figure}






%%%%%%%%%%%%%%%%%%%%
\subsection{Representation of Rotations}
Euler's theorem shows that every rotation $R$ can be thought of as a rotation about some axis $n$ by some angle $\theta$\footnote{You are asked to prove this in exercise \ref{ex:PropertiesCrossProduct}.}. This is depicted in figure \ref{Fig:EulersTheorem}. Therefore we see that the space of rotations, $\mathrm{SO}(3)$ can be identified with the solid ball in $\mathbb{R}^3$ with the antipodal points on the boundary identified or by the space of unit tangent vectors on $\mathbb{S}^2$ that is typically denoted by $T_0\mathbb{S}^2$. This confirms our previous conclusion that the dimension of the space $\mathrm{SO}(3)$ is three. On the other hand this also confirms our observation that the space of $SO(3)$ is quite different from $\mathbb{R}^3$. In fact this shows that $SO(3)$ is not isomorphic to $\mathbb{R}^3$. What this implies is that there exists no way of globally parameterizing $SO(3)$ using three parameters.


\begin{figure}[ht]
\begin{center}
\begin{tabular}{c}
\includegraphics[width=3.0in]{Euler_AxisAngle2.png} 
\end{tabular}
\renewcommand{\baselinestretch}{1}\selectfont
\caption{\href{https://commons.wikimedia.org/wiki/File:Euler_AxisAngle.svg}{\copyright Wikipedia}. Euler's Theorem that says that every rotation is a rotation about some axis $n$ by some angle $\theta$.}
\label{Fig:EulersTheorem}
\renewcommand{\baselinestretch}{1.5}\selectfont
\end{center}
\end{figure}


Since $SO(3)$ is three dimensional any choice of co-ordinates for parameterizing $\mathrm{SO}(3)$ will involve three components. Below we will describe a specific way of assigning coordinates, that are known as Euler angles. But since the preceding discussion shows that $SO(3)$ can not be isomorphic to $\mathbb{R}^3$ there exists no co-ordinate patch that will uniquely specify every point of $\mathrm{SO}(3)$. Thus any choice of Euler angles, or for that matter any three parameter local co-ordinate system on $\mathrm{SO}(3)$ will have points at which they become singular. Below we will begin by showing how one can parameterize $R$ using unit quaternions to avoid this problem of singularity.


%%%%%%%%%%%%%%%
\subsubsection{Quaternion Representation of Rotations}\label{Secn:Quaternion}
Recall that in section-\ref{Secn:PropertiesOfRotations} we have shown that if $\widehat{\Omega}=R^T\dot{R}$ then $\Omega$ corresponds to an instantaneous rotation about $\Omega$ at an instantaneous rate that is equal to $||\Omega||$. Thus if $\Omega$ is a constant then the solution to the following initial value problem 
\begin{align}
\dot{R}&=R\widehat{\Omega},\:\:\:\:\:\:\: R(0)=I_{3\times 3},\label{eq:RdotEqn}
\end{align}
is a pure rotation about the axis $\Omega$ at a constant angular rate of $\Omega$. 
This says that if we define $\theta(t)\triangleq t||\Omega||$ then since $\Omega$ does not depend on $t$ we see that the solution, $R(t)$, to the initial value problem (\ref{eq:RdotEqn}) is a rotation about the axis $\Omega$ by an angle equal to $\theta=t||\Omega||$.
It is easy to verify by direct computation that this solution is explicitly and uniquely given by
\begin{align}
{R}(t)&=\exp{(t\widehat{\Omega})},\label{eq:ExpSoln}
\end{align}
where $\exp{(A)}$ denotes the matrix exponential of the matrix $A$. 


Setting $t=1$ in (\ref{eq:ExpSoln}) we see that $\exp{(\widehat{\Omega})}\in \mathrm{SO}(3)$ for any $\widehat{\Omega}\in \mathrm{so}(3)$ and that it corresponds to a rotation about the axis $\Omega$ by an angle $\theta=||\Omega||$.
This process defines a map $\exp : \mathrm{so}(3) \to \mathrm{SO}(3)$.
By uniqueness of solutions of differential equations we find that the so defined map 
$\exp : \mathrm{so}(3) \to \mathrm{SO}(3)$ is locally one-to-one\footnote{Showing this is beyond the scope of this lecture notes as it requires advanced mathematical notions involving group theory.}. In exercise-\ref{ex:EveryR=Rotation} you are asked to show\footnote{Hint: First show that every $R \in\mathrm{SO}(3)$ has an eigenvalue that is equal to one. Let $V$ be the corresponding eigenvector. Recall that $R$ can be thought of as acting on space by rotations. Then  since $RV=V$ we see that $R$ is a rotation about $V$. Alternatively, consider an orthonormal basis for $\mathbb{R}^3$ with $V$ as the first basis vector, we may express $R$ in this new basis and come to the same conclusion.} that every $R \in \mathrm{SO}(3)$ can be thought of as a rotation about some axis $\Omega$ and hence that for any given $R$ there exists some $\widehat{\Omega}\in \mathrm{so}(3)$ such that $R=\exp{(\widehat{\Omega})}$. Thus we see that the map $\exp : \mathrm{so}(3) \to \mathrm{SO}(3)$ is onto as well.
\begin{svgraybox}
In summary what we have shown is that for every $\Omega$ there is a unique corresponding $R$ given by the matrix exponential $\exp{(\widehat{\Omega})}$ and that for every $R$ there also exists some $\Omega$ such that $R=\exp{(\widehat{\Omega})}$. 
\end{svgraybox}

Let us now proceed to find an explicit expression for $\exp{(\widehat{\Omega})}$. In exercise-\ref{ex:PropertiesCrossProduct} you are asked to prove that 
\[
\widehat{\Omega}^2=(\Omega\Omega^T-||\Omega||^2I).
\]
From this it follows that
\[
\widehat{\Omega}^3=-||\Omega||^2\widehat{\Omega},\:\:\:\:\:\:\:\widehat{\Omega}^4=-||\Omega||^2\widehat{\Omega}^2,\:\:\:\:\:\:\:\widehat{\Omega}^5=||\Omega||^4\widehat{\Omega},\:\:\:\:\:\:\:\cdots
\]
and therefore in exercise-\ref{ex:PropertiesCrossProduct} you asked  to show that
\begin{align}
\exp{\left({\widehat{\Omega}}\right)}&=I+\frac{\sin{||\Omega||}}{||\Omega||}\widehat{\Omega}+\frac{1}{2}\left(\frac{\sin{\frac{||\Omega||}{2}}}{{\frac{||\Omega||}{2}}}\right)^2\widehat{\Omega}^2.\label{eq:Rodrigues0}
\end{align}
This is famously known as the \textit{Rodrigues formula}. We stress again that this corresponds to a rotation about the axis $\Omega$ by an angle $\theta=||\Omega||$. Thus letting $n=\Omega/||\Omega||$ be the unit length direction along $\Omega$ we can write the above equation also as
\begin{align}
R=\exp{\left(\theta\,{\widehat{n}}\right)}&=I+\sin{\theta}\,\widehat{n}+(1-\cos{\theta})\,\widehat{n}^2.\label{eq:Rodrigues22}
\end{align}
Since we have seen that every $R \in \mathrm{SO}(3)$ can be written as $R=\exp{(\widehat{\Omega})}$ for some $\widehat{\Omega}\in \mathrm{so}(3)$ we can also conclude that every $R \in \mathrm{SO}(3)$ can be written down using the expression (\ref{eq:Rodrigues22}) for some angle $\theta$ and unit direction $n$.
Observe that specifically if $E_1=[1\:\:\:0\:\:\:0]^T,E_2=[0\:\:\:1\:\:\:0]^T,E_3=[0\:\:\:0\:\:\:1]^T$ then the Rodrigues formula (\ref{eq:Rodrigues22}) gives that 
$\exp{(\theta_i\widehat{E}_i)}=R_i(\theta_i)$ corresponds to a rotation about the axis $E_i$ by an angle equal to $\theta_i$.

Let $w=\sin{\left(\frac{\theta}{2}\right)}n$ and 
$q_0=\cos{\left(\frac{\theta}{2}\right)}$.  Observe that since $q_0^2+||w||^2=1$ the ordered quadruple of numbers $q=(q_0,w)$ represents a point on the surface of the unit sphere, $\mathbb{S}^3=\{q\in \mathbb{R}^4\:\:\: |\:\:\: ||q||=1\}$, in $\mathbb{R}^4$.  Then re-arranging the above expression we have that every $R\in \mathrm{SO}(3)$ can be written down as
\begin{align*}
R&=I+2q_0\widehat{w}+2\widehat{w}^2.
\end{align*}
for some $(q_0,w)\in\mathbb{S}^3$. 
Conversely we also see that for every $q=(q_0,w)\in \mathbb{S}^3$ there is a unique corresponding $R\in \mathrm{SO}(3)$  that is explicitly given by the above formula. Note that, since $-q=(-q_0,-w)\in \mathbb{S}^3$ also gives the same $R$ that is given by $q=(q_0,w)\in \mathbb{S}^3$, this correspondence is two-to-one. In summary, we have the following:
\begin{svgraybox}
The group of rotations $\mathrm{SO}(3)$ is isomorphic to $\mathbb{S}^3/\{1,-1\}$ where the isomorphism is explicitly given by the Rodrigues formula 
\begin{align}
R&=I+2q_0\widehat{w}+2\widehat{w}^2.\label{eq:RodriguesFormulaMain}
\end{align}
Here $R$ corresponds to a rotation about $w$ by an angle equal to $\theta =2\cos^{-1}({q_0})$.
Using algebraic manipulations it can also be shown that, in terms of $(q_0,w)\in \mathbb{S}^3$, the differential equation $\dot{R}=R\widehat{\Omega}$ becomes
\begin{align}
\left[\begin{array}{c}\dot{q}_0 \\ \dot{w}
\end{array}\right] &=
\frac{1}{2}\left[\begin{array}{c}-\Omega \cdot w \\ q_0\Omega-\Omega\times {w}
\end{array}\right].\label{eq:QuaternionEquations}
\end{align}
\end{svgraybox}

The unit sphere in $\mathbb{R}^4$, denoted by $\mathbb{S}^3=\{q\in \mathbb{R}^4 \:\:\:|
\:\:\: ||q||=1\}$ is also known as the space of \textit{unit quaternions}. We conclude this section by noting that the unit quaternion $q=(q_0,w)$ that correspond to a given rotation matrix $R$ can be found using the following two expressions obtained from (\ref{eq:RodriguesFormulaMain}):
\begin{align}
\mathrm{trace}(R)&=-1+4q_0^2=2\cos \theta +1,\\
R-R^T&=4q_0 \widehat{w}=4\cos{\left(\frac{\theta}{2}\right)} \widehat{w}.
\end{align}
The first expression determines $q_0=\cos{\frac{\theta}{2}}$ and the second expression determines $w$.


%%%%%%%%%%%%%%%%%%%%%%%%%%%%%%%%%%


%%%%%%%%%%%%%%%%%%%


\subsubsection{Euler Angle Representation of Rotations}\label{Secn:EulerAngles}

%%%%%%%%%%%%%%%%%%%%%%%%%%%%%


The following matrix $R$ obtained by composing the three consecutive counter clockwise rotations about the axis $i-j-k$ respectively 
\begin{align}\label{eq:EulerAngles}
R=R_i(\theta_1)R_j(\theta_2)R_k(\theta_3)
\end{align}
is once again a special orthogonal matrix. This corresponds to a composition of a sequence of rotations of a frame first around axis $i$ by an angle $\theta_1$, then by an angle $\theta_2$ around axis $j$ and finally by an angle $\theta_2$ around axis $k$. This provides a map from a neighborhood of the origin of $\mathbb{R}^3$ to a neighborhood of the identity in $SO(3)$. Thus when $i\neq j$ and $j\neq k$ the three numbers $(\theta_1,\theta_2,\theta_3)$ serve as a local parameterization for $SO(3)$. This is called the $i-j-k$ Euler angle parameterization of $R$. It is 
important to note that for certain angles this correspondence is not unique, meaning that there exists certain $R$ such that the corresponding choice for $(\theta_1,\theta_2,\theta_3)$ is not unique.
For example consider the parameterization of $R$ using the three angles $(\theta_1,\theta_2,\theta_3)$ corresponding to the 3-1-3 Euler angle parameterization of $R$ that is explicitly given by
\begin{align}\label{eq:EulerAngles313}
R=R_3(\theta_1)R_1(\theta_2)R_3(\theta_3)=\begin{bmatrix}
c_1c_3-c_2s_1s_3 & c_2c_3s_1+c_1s_3 & s_1s_2  \\
-c_3s_1-c_1c_2s_3 & c_1c_2c_3-s_1s_3 & c_1s_2\\
s_2s_3 & -c_3s_2 & c_2\end{bmatrix}
\end{align}
where we have used the notation $c_i\triangleq \cos{\theta_i}$, $s_i\triangleq \sin{\theta_i}$. This parameterization is shown in figure \ref{Fig:313Euler}. 
It is easy to see that  when $\theta_2=0$ or $\theta_2=\pi$ then $R=R_3(\theta_1+\theta_3)$ and hence that there exists no unique $\theta_1,\theta_3$ that describe the orientation of the resulting two frames $\mathbf{e}$ and $\mathbf{b}$. This situation is called \emph{gimbal lock} in the stellite and robotics communities.
This turns out to be a common problem for any type of Euler angles being used. Thus in particular we find that the Euler angles only provide a local isomorphism between $\mathbb{R}^3$ and $SO(3)$.

The 3-1-3 Euler angles turn out to be a popular choice of local coordinates for $SO(3)$ in the  robotics and satellite communities. In this parameterization $\theta_1$ is called 
the angle of \textit{precession}, $\theta_2$ is called the angle of \textit{nutation}, and $\theta_3$ is called the angle of \textit{spin}.
\begin{figure}[ht]
\begin{center}
\includegraphics[width=3.5in]{Eulerangles_313.png} 
\renewcommand{\baselinestretch}{1}\selectfont
\caption{\href{https://en.wikipedia.org/wiki/Euler_angles}{\copyright Wikipedia}. The 3-1-3 Euler angle parameterization on $SO(3)$.}
\label{Fig:313Euler}
\renewcommand{\baselinestretch}{1.5}\selectfont
\end{center}
\end{figure}
For illustration purposes, below, we will provide explicit expressions for the representation of the angular velocity using the 3-1-3 Euler angle representation of $R$ that is shown in figure \ref{Fig:313Euler}. Consider two frames $\mathbf{b}$ and $\mathbf{e}$ where $\mathbf{b}=\mathbf{e}\,R$. Then if 
$R^T\dot{R}=\widehat{\Omega}$, we have seen earlier that $\Omega$ is the representation of the angular velocity in the $\mathbf{b}$-frame while $\omega=R\Omega$ is the representation of the angular velocity in the $\mathbf{e}$-frame.   One can find that in terms of the 3-1-3 Euler angles they are explicitly given by:
\begin{align}
\Omega & = 
\begin{bmatrix}
\dot{\theta}_1\sin{\theta_2}\sin{\theta_3}+\dot{\theta}_2\cos{\theta_3}\\
\dot{\theta}_1\sin{\theta_2}\cos{\theta_3}-\dot{\theta}_2\sin{\theta_3}\\ \dot{\theta}_1\cos{\theta_2}+\dot{\theta}_3
\end{bmatrix},  \label{eq:BodyAngVel313Euler}
\end{align}
\begin{align}
\omega & =
\begin{bmatrix} 
\dot{\theta}_3\sin{\theta_2}\sin{\theta_1}+\dot{\theta}_2\cos{\theta_1}\\
 -\dot{\theta}_3\sin{\theta_2}\cos{\theta_1}+\dot{\theta}_2\sin{\theta_1}\\
\dot{\theta}_3\cos{\theta_2}+\dot{\theta}_1\end{bmatrix}.  \label{eq:SpatialAngVel313Euler}
\end{align}
Differentiating (\ref{eq:BodyAngVel313Euler}) we obtain the body angular acceleration
\begin{align}
\dot{\Omega}&= 
\begin{bmatrix}
\ddot{\theta}_1\sin{\theta_2}\sin{\theta_3}+\dot{\theta}_1\dot{\theta}_2\cos{\theta_2}\sin{\theta_3}+\dot{\theta}_1\dot{\theta}_3\sin{\theta_2}\cos{\theta_3}+\ddot{\theta}_2\cos{\theta_3}-\dot{\theta}_2\dot{\theta}_3\sin{\theta_3}\\
\ddot{\theta}_1\sin{\theta_2}\cos{\theta_3}+\dot{\theta}_1\dot{\theta}_2\cos{\theta_2}\cos{\theta_3}-\dot{\theta}_1\dot{\theta}_3\sin{\theta_2}\sin{\theta_3}-\ddot{\theta}_2\sin{\theta_3}-\dot{\theta}_2\dot{\theta}_3\cos{\theta_3}\\
\ddot{\theta}_1\cos{\theta_2}-\dot{\theta}_1\dot{\theta}_2\sin{\theta_2}+\ddot{\theta}_3
\end{bmatrix}.\label{eq:BodyAngAcc313Euler}
\end{align}
From (\ref{eq:BodyAngVel313Euler}) we have
\begin{align*}
\Pi=\mathbb{I}\Omega & = 
\begin{bmatrix}
\mathbb{I}_1\left(\dot{\theta}_1\sin{\theta_2}\sin{\theta_3}+\dot{\theta}_2\cos{\theta_3}\right)\\
\mathbb{I}_2\left(\dot{\theta}_1\sin{\theta_2}\cos{\theta_3}-\dot{\theta}_2\sin{\theta_3}\right)\\ 
\mathbb{I}_3\left(\dot{\theta}_1\cos{\theta_2}+\dot{\theta}_3\right)
\end{bmatrix},
\end{align*}
and from (\ref{eq:BodyAngAcc313Euler}) we have
\begin{align*}
\mathbb{I}\dot{\Omega}&= 
\begin{bmatrix}
\mathbb{I}_1\left(\ddot{\theta}_1\sin{\theta_2}\sin{\theta_3}+\dot{\theta}_1\dot{\theta}_2\cos{\theta_2}\sin{\theta_3}+\dot{\theta}_1\dot{\theta}_3\sin{\theta_2}\cos{\theta_3}+\ddot{\theta}_2\cos{\theta_3}-\dot{\theta}_2\dot{\theta}_3\sin{\theta_3}\right)\\
\mathbb{I}_2\left(\ddot{\theta}_1\sin{\theta_2}\cos{\theta_3}+\dot{\theta}_1\dot{\theta}_2\cos{\theta_2}\cos{\theta_3}-\dot{\theta}_1\dot{\theta}_3\sin{\theta_2}\sin{\theta_3}-\ddot{\theta}_2\sin{\theta_3}-\dot{\theta}_2\dot{\theta}_3\cos{\theta_3}\right)\\
\mathbb{I}_3\left(\ddot{\theta}_1\cos{\theta_2}-\dot{\theta}_1\dot{\theta}_2\sin{\theta_2}+\ddot{\theta}_3\right)
\end{bmatrix}.
\end{align*}
Thus we see that Euler's Rigid body equations (\ref{eq:EulerEqns3D}) in 3-1-3 Euler angles $(\theta_1,\theta_2,\theta_3)$ take the form
{\tiny
\begin{align*}
\mathbb{I}_1\sin{\theta_2}\sin{\theta_3}\,\ddot{\theta}_1+\mathbb{I}_1\cos{\theta_3}\,\ddot{\theta}_2 & = - \mathbb{I}_1\left(\dot{\theta}_1\dot{\theta}_2\cos{\theta_2}\sin{\theta_3}+\dot{\theta}_1\dot{\theta}_3\sin{\theta_2}\cos{\theta_3}-\dot{\theta}_2\dot{\theta}_3\sin{\theta_3}\right)+(\mathbb{I}_2-\mathbb{I}_3)\left(\dot{\theta}_1\sin{\theta_2}\cos{\theta_3}-\dot{\theta}_2\sin{\theta_3}\right)\left(\dot{\theta}_1\cos{\theta_2}+\dot{\theta}_3\right)+ T_1^e \\
\mathbb{I}_2\sin{\theta_2}\cos{\theta_3}\,\ddot{\theta}_1-\mathbb{I}_2\sin{\theta_3}\,\ddot{\theta}_2 & =  -\mathbb{I}_2\left(\dot{\theta}_1\dot{\theta}_2\cos{\theta_2}\cos{\theta_3}-\dot{\theta}_1\dot{\theta}_3\sin{\theta_2}\sin{\theta_3}-\dot{\theta}_2\dot{\theta}_3\cos{\theta_3}\right)+(\mathbb{I}_3-\mathbb{I}_1)\left(\dot{\theta}_1\cos{\theta_2}+\dot{\theta}_3\right) \left(\dot{\theta}_1\sin{\theta_2}\sin{\theta_3}+\dot{\theta}_2\cos{\theta_3}\right) + T_2^e\\
\mathbb{I}_3\cos{\theta_2}\,\ddot{\theta}_1+\mathbb{I}_3\ddot{\theta}_3& = \mathbb{I}_3\dot{\theta}_1\dot{\theta}_2\sin{\theta_2}+ (\mathbb{I}_1-\mathbb{I}_2)\left(\dot{\theta}_1\sin{\theta_2}\sin{\theta_3}+\dot{\theta}_2\cos{\theta_3}\right) \left(\dot{\theta}_1\sin{\theta_2}\cos{\theta_3}-\dot{\theta}_2\sin{\theta_3}\right) + T_3^e.
\end{align*}
}


Since we have seen that 3-1-3 Euler angles become singular when $\theta_2=0$ we see that the above equations also become ildefined at  $\theta_2=0$.In section-\ref{Secn:Quaternion} we have seen how to overcome the complications involved in parameterizing $R$ using Euler angles by instead using unit quaternions $(q_0,w)\in \mathbb{S}^3$ as given by expression (\ref{eq:RodriguesFormulaMain}). We also pointed out that in this parameterization $\dot{R}=R\widehat{\Omega}$ is given by the differential equation (\ref{eq:QuaternionEquations}) and hence one can avoid any singularities by resorting to unit quaternions instead.

%%%%%%%%%%%%%%%%%%%%%%%%%%%%%





%%%%%%%%%%%%%%%%%%%%%%%%%%%%%%%%%%%%%%%%

\subsubsection*{Example of a Forced Gyroscope}\label{Secn:ForcedGyro}
\begin{figure}[h]
\begin{center}
\includegraphics[width=2.2in]{Gyro_gimbals}
\renewcommand{\baselinestretch}{1}\selectfont
\caption{}
\label{Fig:Gyroscope}
\renewcommand{\baselinestretch}{1.5}\selectfont
\end{center}
\end{figure}

For the axi-symmetric gyroscope, $\mathbb{I}_1=\mathbb{I}_2=\mathbb{I}_l,\mathbb{I}_z$
and the rigid body equations take the form
\begin{eqnarray}
\mathbb{I}_l\dot{\Omega}_1 & = & (\mathbb{I}_l-\mathbb{I}_z)\Omega_2 \Omega_3 + T_1, \label{eq:AxiSym1}\\
\mathbb{I}_l\dot{\Omega}_2 & = & (\mathbb{I}_z-\mathbb{I}_l)\Omega_3 \Omega_1 + T_2, \label{eq:AxiSym2}\\
\mathbb{I}_z\dot{\Omega}_3 & = &  T_3. \label{eq:AxiSym3}
\end{eqnarray}


Let us parameterize the rotation matrix from the earth fixed frame $\mathbf{e}$ to a disk fixed frame $\mathbf{d}$ using the 3-1-3 Euler angles.  We will need this to express the force moment
\[T=\left[\begin{array}{c} T_1 \\ T_2\\ T_3
\end{array}\right].
\]

Let $\mathbf{e}$ be an Earth fixed frame with origin, $O$. A frame $\mathbf{b}$ is fixed to the outer gimbal of the Gyroscope, a frame $\mathbf{c}$ is fixed to the inner gimbal and a frame $\mathbf{d}$ is fixed on the 
disc. Let $\mathbf{d}(t)=\mathbf{e}\,R(t)$.
The $\mathbf{b}(t)$ frame is fixed on the outer gimbal so that $\mathbf{b}_3(t)\equiv \mathbf{e}_3(t)$. Then $\mathbf{b}(t)=\mathbf{e}\,R_3(\theta_1(t))$ where $\theta_1$ is the angle of rotation of $\mathbf{b}$ about the fixed $\mathbf{e}_3, \mathbf{b}_3$ 
axis.
The frame $\mathbf{c}(t)$ is fixed on the inner gimbal so that $\mathbf{c}_1(t)\equiv \mathbf{b}_1(t)$. Then $\mathbf{c}(t)=\mathbf{b}(t)\,R_1(\theta_2(t))$ where
$\theta_2$ is the angle of rotation of $\mathbf{c}$ about the $\mathbf{c}_1, \mathbf{b}_1$ axis.
The frame $\mathbf{d}(t)$ is fixed on the disc so that $\mathbf{d}_3(t)\equiv \mathbf{c}_3(t)$. Then $\mathbf{d}(t)=\mathbf{c}(t)\,R_3(\theta_3(t))$ where
$\theta_3$ is the angle of rotation of $\mathbf{c}$ about the $\mathbf{c}_3, \mathbf{d}_3$ axis.

Now since $\mathbf{d}(t)=\mathbf{e}\,R(t)$ we have that
\begin{equation}\label{eq:EulerAngles}
R=R_3(\theta_1)R_1(\theta_2)R_3(\theta_3)
=\left[\begin{array}{ccc}
c_1c_3-c_2s_1s_3 & c_2c_3s_1+c_1s_3 & s_1s_2  \\
-c_3s_1-c_1c_2s_3 & c_1c_2c_3-s_1s_3 & c_1s_2\\
s_2s_3 & -c_3s_2 & c_2
\end{array}
\right]
\end{equation}
and hence that $R$ is naturally parameterized by the 3-1-3 Euler angles. Recall that these are singular when $\theta_2=0,\pi$.

Neglecting the gimbal inertia, the force moments acting on the disc are calculated as follows.
The moments acting on the outer gimbal due to the fixed end.
\[
T^f = e\,T^1=e  \left[
\begin{array}{c}
T^1_1\\ T_2^1 \\ u_1
\end{array}
\right]
\]
The moments acting on the inner Gimbal due to the outer Gimbal.
\[
T^I = \mathbf{b}\,T^2=\mathbf{b} \left[
\begin{array}{c}
u_2\\ T^2_2 \\ T_3^2
\end{array}
\right]
\]
The moments acting on the disc due to the inner Gimbal.
\[
T^D = \mathbf{c}\,T^3=\mathbf{c}  \left[
\begin{array}{c}
T_1^3\\ T_2^3 \\ u_3
\end{array}
\right]=\mathbf{d}\,T
\]
Neglecting the inertia of the outer and inner gimbals we have
\[
\mathbf{e}\,T^1=-\mathbf{b}\,T^2,\:\:\:\:\mathbf{b}\,T^2=-\mathbf{c}\,T^3
\]
and hence
\begin{equation}\label{eq:GimbalMoments}
T^1=-R_3(\theta_1)\,T^2,\:\:\:\: T^2=-R_1(\theta_2)\,T^3,\:\:\:\: T^3=R_3(\theta_3)\, T,
\end{equation}
and
\[
T^1=R_3(\theta_1)R_1(\theta_2)R_3(\theta_3)\; T = R\, T.
\]

From the first set of equations in (\ref{eq:GimbalMoments}) we have that $T^2_3=-u_1$ and from the second set of equations in (\ref{eq:GimbalMoments}) we have that $T_1^3=-
u_2$. Substituting these in the second set of equations in (\ref{eq:GimbalMoments}) we have
\[
\left[
\begin{array}{c}
u_2\\ T^2_2 \\ -u_1
\end{array}
\right] =
R_2 \left[
\begin{array}{c}
u_2\\ -T_2^3 \\ -u_3
\end{array}
\right].
\]
The last equation in this set of equations give
\[
T^3_2=\frac{(u_1-\cos{\theta_2}u_3)}{\sin{\theta_2}}.
\]
Now since $T^3=R_3\,T$ the force moments on the disc expressed in $d$ is given by
\[
T=
R_3^T \left[
\begin{array}{c}
-u_2\\ \frac{(u_1-\cos{\theta_2}u_3)}{\sin{\theta_2}} \\ u_3
\end{array}
\right]=\left[
\begin{array}{ccc}
\cos{\theta_3} & \sin{\theta_3} & 0 \\
-\sin{\theta_3} & \cos{\theta_3} & 0\\
0 & 0 & 1
\end{array}
\right]\left[
\begin{array}{ccc}
0 & -1 & 0 \\
\frac{1}{\sin{\theta_2}} & 0 & -\cot{\theta_2}\\
0 & 0 & 1
\end{array}
\right]\left[
\begin{array}{c}
u_1\\ u_2 \\ u_3
\end{array}
\right].
\]

\[
T=\left[
\begin{array}{ccc}
\frac{\sin{\theta_3}}{\sin{\theta_2}} & -\cos{\theta_3} & -\sin{\theta_3}\cot{\theta_2} \\
\frac{\cos{\theta_3}}{\sin{\theta_2}} & \sin{\theta_3} & -\cos{\theta_3}\cot{\theta_2}\\
0 & 0 & 1
\end{array}
\right]\left[
\begin{array}{c}
u_1\\ u_2 \\ u_3
\end{array}
\right].
\]
Here $u_1,u_2$ and $u_3$ are the external moments applied about the rotation axis of the oute gimbal, the rotation axis of the inner gimbal and the rotation axis of the disc.

Observe that the above expression for the force moment becomes singular when $\theta_2=0$ or $\theta_2=\pi$. At these angels the external moment directions on the outer gimbal 
and the disc coincides. This situation is commonly referred to as gimbal locking.




%%%%%%%%%%%%%%%%%%%%%%


%%%%%%%%%%%%%%%%%%%%%%

\newpage
\chapter{Lagrangian Formulation of Classical Mechanics}
When one considers a system that consists of several inter connected rigid bodies the process of describing the motion of the entire system by the application of Euler's rigid body equations become cumbersome due to the fact that each body has to be considered separately. The Lagrangian formulation of Newton's laws of motion circumvent this problem. It hinges on the idea of viewing the composite system as a point that evolves in an abstract space called the {configuration space} of the system and on insisting that laws of nature be independent of the coordinates used to represent the system.
In the next several sections we will develop this concept in detail.

\section{Configuration Space and Coordinates}
The first step that an observer takes in describing the motion of a system of particles is the complete specification of each of their positions in 3D Euclidean space. This specification is referred to as the 
\textit{configuration} of the system. 
The set of all possible configurations is called the \textit{configuration space}. We will begin our study of describing the motion of particles by investigating the configuration space of a single particle that is free to move in 3D space. We have seen in the previous chapter that, in  Galilean mechanics, the configuration space of a free particle is assumed to be of the from of a 3D Euclidean space allowing us to identify a point $P$ in space with a point in $\mathbb{R}^3$ using an orthonormal frame $\mathbf{e}=[\mathbf{e}_1\:\:\:\mathbf{e}_2\:\:\:\mathbf{e}_3]$. We do this by assigning the the ordered triple of numbers $(x_1,x_2,x_3)$ to the point $P$ where $x_i$ is the perpendicular distance to the point along the $\mathbf{e}_i$ direction of the orthonormal frame.  
Observe that this identification depends entirely on the choice of the ortho-normal frame $\mathbf{e}$.
For instance consider figure-\ref{Fig:GeneralMovingFrame0} in section-\ref{Secn:RelativeMotion}. The position of the particle $P$ is described by the two different ordered triple of real numbers $(x_1,x_2,x_3)$ and $
(X_1,X_2,X_3)$ in the two different frames $\mathbf{e}$ and $\mathbf{b}$ respectively.

Orhto-normal frames are also not the only means of identifying 3D-Euclidean space with $\mathbb{R}^3$. Curvilinear frames can also be used for this identification. For example, as 
shown in figure-\ref{Fig:SphericalPolar}, we may use the quantities $(r,\theta,\phi)$ to describe a point\footnote{Observe that not every point in three dimensional Euclidian space can 
be uniquely represented using these co-ordinates. For instance none of the points on the $\mathbf{e}_3$ axis has a unique representation. Such points are called co-ordinate singularities of 
the co-ordinate system.}.
All three sets of ordered triples $(x_1,x_2,x_3)$, $(X_1,X_2,X_3)$ and $(r,\theta,\phi)$ describe the same point in 3D-Euclidean space but note that they are in general different. Thus the three descriptions 
provide three different identifications with $\mathbb{R}^3$. A particular identification of Euclidian space with $\mathbb{R}^3$
is what we refer to as a choice of \textit{co-ordinates} for the Euclidian space. Therefore all three sets of ordered triples $(x_1,x_2,x_3)$, $(X_1,X_2,X_3)$ and $(r,\theta,\phi)$ provide 
three different choices of co-ordinates for Euclidian space. However all identifications are with $\mathbb{R}^3$ (have three numbers or components) and thus we say that the 
dimension of the configuration space of the unconstrained particle is three.

When the co-ordinates are expressed using an ortho-normal frame they are called rectangular or Euclidean co-ordinates. The relationship between two different co-ordinates is 
called a \textit{co-ordinate transformation}. The rectangular or Euclidean co-ordinates of the point $P$ given by $(x_1,x_2,x_3)$ are related to the spherical polar co-ordinates $(r,\theta,\phi)$ by
\begin{figure}[ht]
\begin{center}
\includegraphics[width=3.5in]{PolarCoordinates0}
\renewcommand{\baselinestretch}{1}\selectfont
\caption{Description of the point $P$ using Spherical-Polar Co-ordinates. Figure copied from \cite{Greenwood}.}
\label{Fig:SphericalPolar}
\renewcommand{\baselinestretch}{1.5}\selectfont
\end{center}
\end{figure}
\begin{eqnarray*}
x_1 &=& r\sin{\phi}\cos{\theta},\\
x_2 &=& r\sin{\phi}\sin{\theta},\\
x_3 &=& r\cos{\phi}.
\end{eqnarray*}
Thus using spherical polar co-ordinates we can express the Euclidean representation matrix $x$ as,
\[x=
\left[ \begin{array}{c} x_1\\ x_2\\ x_3 \end{array} \right]=\left[ \begin{array}{c} r\sin{\phi}\cos{\theta}\\r\sin{\phi}\sin{\theta}\\r\cos{\phi} \end{array} \right].
\]

The everyday motion of many of the objects of interest can not be approximated or abstracted as the motion of a single unconstrained particle. Typically the motion of such objects 
can be viewed as the collective motion of a set of particles that are individually constrained through their mutual interaction or due to the external influence of the rest of the particles in the universe. 

In general when a point particle is geometrically constrained to move in space the three quantities $x_1,x_2,x_3$, comprising the components of the 
Euclidean representation matrix $x$, will not be independent. If the motion of the particle is geometrically constrained so that the quantities $(x_1,x_2,x_3)$ have to satisfy some 
scalar expression
$
h(x_1,x_2,x_3)=0
$
then the particle is said to be \textit{Holonomically} constrained. Holonomic constraints reduce the degrees of freedom. A single constraint of the form $
h(x_1,x_2,x_3)=0
$
reduces the DOF by one, from three to two. If the number of such constraints are two then the DOF of the particle reduces to one. For instance when the particle is constrained to 
move on a sphere of radius equal to one then the constraint equation in the spherical-polar co-ordinates is $h(r,\theta,\phi)=r-1=0$ and the Euclidean position representation $x$ will become $[\sin{\phi}
\cos{\theta}\:\:\:\:\sin{\phi}\sin{\theta}\:\:\:\:\cos{\phi}]^T$. Notice that the two independent measurements $(\theta,\phi)$ are sufficient to describe the Euclidean position uniquely when $
\phi\neq 0,\pi$. If the particle is constrained to move in a circle of radius one that lies in the $\mathbf{e}_1,\mathbf{e}_2$ plane then there are two constraints, $h_1(r,\theta,\phi)=r-1=0$, $h_2(r,\theta,
\phi)=\phi-\pi/2=0$ and the Euclidean position representation $x$ will become $[\cos{\theta}\:\:\:\:\sin{\theta}\:\:\:\:0]^T$. Once again notice that the single measurement $\theta$ is sufficient to describe the 
Euclidean position uniquely.
The general expression
\begin{equation}\label{eq:DOF_Point}
n=3-f,
\end{equation}
relates the DOF, $n$, of a particle moving in three dimensional space to the number of holonomic constraints, $f$, imposed on that particle.  If $N$ points are moving in space then the total degrees of freedom of the system of points is
\begin{equation}\label{eq:DOF_NPoints}
n=3N-f,
\end{equation}
where now $f$ is the total number of holonomic constraints of the system of points.
Summarizing the above discussion we have seen that if the particle is 
constrained to move such that its distance to the origin is always fixed then the configuration space can be identified with the 2D-sphere, $\mathbb{S}^2$, and if it is additionally constrained to move in a 
plane then the configuration space can be identified with the circle,  $\mathbb{S}$. A little less obvious example is the planar double pendulum discussed below.
\begin{example}
Consider the planar double pendulum shown in figure \ref{Fig:DoublePendulum0}.
The configuration space of mass $m_1$ can clearly be identified with the circle with origin at the fixed point $O$ and radius $L_1$. With respect to $m_1$ the possible configurations of 
$m_2$ lie on the circle  $\mathbb{S}$, with origin at $m_1$ and radius $L_2$. Thus if the configuration of $m_1$ (that is the location of $m_1$ on the circle of radius $L_1$ with origin fixed 
at $O$) is known then the configuration of $m_2$ can be uniquely specified if we specify the position of $m_2$ on the circle  $\mathbb{S}$, with origin at $m_1$ and radius $L_2$. Thus the 
configuration space of the two particle system is the torus $\mathbb{S}\times \mathbb{S}$.
\begin{figure}[ht]
\begin{center}
\includegraphics[width=1.2in]{DoublePendulum}
\renewcommand{\baselinestretch}{1}\selectfont
\caption{The Double Planar Pendulum}
\label{Fig:DoublePendulum0}
\renewcommand{\baselinestretch}{1.5}\selectfont
\end{center}
\end{figure}
\end{example}

The idea of co-ordinates for a general configuration space, such as a sphere, a circle or a torus, is developed in a fashion similar to what we have done so far for Euclidean space.
Roughly, we try to associate points of the configuration space with an ordered $n$-tuple of real numbers in an essentially unique way. That is we try to identify either the entire configuration 
space, or a significant portion of it with some portion of $\mathbb{R}^n$. Such an identification is called a co-ordinate patch of the configuration space\footnote{The precise 
mathematical definition of this notion is beyond the scope of these notes}.
If a particular co-ordinate patch is incapable of covering all the points of a configuration space we may require several other choices to represent these points as well. Then by 
``patching'' them together we can ``cover'' all the points in the configuration space.  An important fact about co-ordinates is that if the configuration space is ``smooth'' then what 
ever the choice of co-ordinates we use, the number $n$ in the identification is a constant. 
This number is referred to as the \textit{dimension} of the configuration space. The \textit{degrees of freedom (DOF)} of a system of particles is defined to be the dimension $n$ of the 
configuration space.

\begin{example}
If the configuration space is a circle, as is the case for a simple blob pendulum, then the single quantity $\theta$ is sufficient to uniquely describe every point except one point on the 
circle\footnote{The points described by $\theta=0$ and $\theta=2\pi$ is the same point hence the uniqueness fails at this point.}. Thus prescribing $\theta$ allows us to uniquely 
identify almost all points on the circle with $\mathbb{R}$ and provides a choice of co-ordinates for the circle. Since the identification has only one component or in other words is with 
$\mathbb{R}$ the circle is one dimensional. To account for the problematic point we can start from a different point on the circle to get a different angle correspondence. That gives us 
another choice of a co-ordinate patch for the circle. Collectively both these co-ordinate patches will cover all the points on the circle.
\end{example}


\begin{example}
Let us consider the sphere. Referring to figure \ref{Fig:SphericalPolar}
we can see that the two independent quantities $\theta$ and $\phi$ uniquely describe every a point on the sphere except the North and South poles of the sphere\footnote{Try to 
reason why this is so.}. Thus $(\theta,\phi)$ provide us with a choice of a co-ordinate patch for the sphere and covers almost all points on the sphere. That is we have prescribed a way of 
uniquely identifying any point, except the poles, on the sphere with a point in $\mathbb{R}^2$. Since the identification is with $\mathbb{R}^2$ the dimension of the sphere is two. 
\end{example}




%%%%%%%%%

\begin{example}\label{Example:ParticleOnSphere}
Consider a particle constrained to move on a sphere of radius $r$ (refer to figure \ref{Fig:SphericalPolar}).
The representation of $P$ using an ortho-normal frame $\mathbf{e}$ fixed at the center of the sphere can be expressed using the polar co-ordinates $(\theta,\phi)$ as
\[
x=r\left[ \begin{array}{c} \sin{\phi}\cos{\theta}\\\sin{\phi}\sin{\theta}\\\cos{\phi} \end{array} \right].
\]
Then the velocity in the $\mathbf{e}$ frame expressed using the polar co-ordinates is
\[
v=\dot{x}=r\left[ \begin{array}{c} \dot{\phi}\cos{\phi}\cos{\theta}-\dot{\theta}\sin{\phi}\sin{\theta}\\
\dot{\phi}\cos{\phi}\sin{\theta}+\dot{\theta}\sin{\phi}\cos{\theta}\\-\dot{\phi}\sin{\phi} \end{array} \right].
\]
and the acceleration in the $\mathbf{e}$ frame expressed using the polar co-ordinates is
\[
a=\dot{v}=r\left[ \begin{array}{c}
a\ddot{\phi}\cos{\phi}\cos{\theta}-\dot{\phi}^2\sin{\phi}\cos{\theta}
-2\dot{\phi}\dot{\theta}\cos{\phi}\sin{\theta}-\dot{\theta}^2\sin{\phi}\cos{\theta}-\ddot{\theta}\sin{\phi}\sin{\theta}\\
\ddot{\phi}\cos{\phi}\sin{\theta}-\dot{\phi}^2\sin{\phi}\sin{\theta}
+2\dot{\phi}\dot{\theta}\cos{\phi}\cos{\theta}-\dot{\theta}^2\sin{\phi}\sin{\theta}+\ddot{\theta}\sin{\phi}\cos{\theta}\\
-\ddot{\phi}\sin{\phi}-\dot{\phi}^2\cos{\phi} \end{array} \right].
\]
\\
\end{example}







%%%%%%%
\begin{svgraybox}
In general a configuration space will be referred to as a \textit{configuration manifold}. Roughly a general space $\mathcal{Q}$ will be called a manifold if $\mathcal{Q}$ can be covered by a set of open sets $\mathcal{U}_\alpha$ such that there exists a diffeomorphism $\phi_\alpha : \mathcal{U}_\alpha \to \mathcal{V}_\alpha\subset \mathbb{R}^n$ for some fixed $n$ an integer and that in any overlapping set the associated map $ \phi_\beta\cdot\phi_\alpha^{-1}: \mathcal{V}_\alpha \cap \mathcal{V}_\beta \to \mathcal{V}_\alpha \cap \mathcal{V}_\beta$ is differentiable.
The pair $(\phi_\alpha,\mathcal{U}_\alpha)$ is called a co-ordinate patch of $\mathcal{Q}$.
\end{svgraybox}
The technically precise definition of a manifold and coordinates is a bit more involved and is beyond the scope of this class.



%%%%%%%%%%%
\section{Euler-Lagrange Equations for Holonomic Systems}

In this section we provide an equivalent formulation of Newtonian mechanics that turns out to be convenient when the system consists of several moving components. In the previous section we have seen that the configuration of a system of interacting and constrained set of particles can be can be thought of as a point in some abstract space called the configuration space $\mathcal{Q}$. 

Let $q(t)$ be a smooth curve on $\mathcal{Q}$ such that $q(0)=q$. Then the velocity at $q$ given by $\dot{q}(0)$ is termed the \textit{tangent vector} to the curve $q(t)$ at $q$. Different parameterisations and different curves through $q$ give different such tangent vectors at $q$. We denote by $T_{q} \mathcal{Q}$ the space of all such tangent vectors at $q$. One can show that $T_{q} \mathcal{Q}$ is a vector space. We call this the tangent space to $\mathcal{Q}$ at $q$. The collection of all such tangent spaces to $\mathcal{Q}$ is called the tangent bundle of $\mathcal{Q}$ and is denoted by $T\mathcal{Q}$. When one writes the total kinetic energy $T$ of the system we observer that it takes the form of a positive definite quadratic form on each $T_{q} \mathcal{Q}$. That is the kinetic energy takes the form
\[
\mathrm{T}=\frac{1}{2}\dot{q}^TG(q)\dot{q}\triangleq \frac{1}{2} \langle\langle \dot{q},\dot{q}\rangle\rangle,
\]
where $G(q)$ is a symmetric positive definite quadratic form that depends smoothly on $q$. Notice that it defines an inner product on each $T_q\mathcal{Q}$ that varies smoothly on $\mathcal{Q}$.

Let $T^*_q\mathcal{Q}$ be the space of all linear functionals on $T_q\mathcal{Q}$. That is $T^*_q\mathcal{Q}$ is the collection of all linear maps $\tau_q : T_q\mathcal{Q} \to \mathbb{R}$. Specifically if $\tau_q\in T^*_q\mathcal{Q}$ then
$\tau_q(v_q)\in \mathbb{R}$ for all $v_q\in T_q\mathcal{Q}$ and $\tau_q$ is linear. We will use the notation $\tau_q(v_q)\triangleq \langle \tau_q,v_q\rangle$.
We call this space $T^*_q\mathcal{Q}$ the cotangent space of $\mathcal{Q}$ at $q$. The collection of all such cotangent spaces at different points over the configuration space is called the cotangent bundle and is denoted by $T^*\mathcal{Q}$. One can show that forces are elements of the cotangent bundle and call them generalized forces. Consider a generalized force $\tau \in  T^*\mathcal{Q}$ and a smooth curve $q : [t_1,t_2] \to \mathcal{Q}$. The quantity 
\[
W\triangleq \int_{q(t)}  \tau=\int_{t_1}^{t_2} \langle \tau,\dot{q}\rangle\,dt
\]
is defined to be the work done by the force along the trajectory $q(t)$. By definition work is path dependent. However there exists certain forces such that the work done by these forces are path independent. We call such forces \textit{conservative forces}. Since the work done by them are path independent one can show that there exists a function $U:\mathcal{Q}\to \mathbb{R}$ such that $\tau=-dU$. The function $U$ will be called the \textit{potential energy function} associated with $\tau$. 

A mechanical system on $\mathcal{Q}$ is defined by the kinetic energy, $KE$, the potential energy $U$, and the nonconservative forces $f\in T^*\mathcal{Q}$ acting on the system. In the Lagrangian formulation of mechanics one defines the Lagrangian ${L}:T\mathcal{Q}\to \mathbb{R}$ by
\[
{L}(q,\dot{q})=T-U=\frac{1}{2}\langle\langle \dot{q},\dot{q}\rangle\rangle-U(q).
\]
 
In a certain co-ordinate patch of $\mathcal{Q}$ we can express $q,\dot{q},dU$ as
$q=(q_1,q_2,\cdots,q_n)$, $\dot{q}=(\dot{q}_1,\dot{q}_2,\cdots,\dot{q}_n)$, and
$dU=(\frac{\partial U}{\partial q_1},\frac{\partial U}{\partial q_2},\cdots,\frac{\partial U}{\partial q_n})$.


Let us look a bit more carefully at the external forces acting on the system. Recall that generalized forces, $\tau$, are linear functionals acting on velocities (linear operators on $T\mathcal{Q}$) and that $\delta w=\langle \tau, \dot{q}\rangle\,\delta t=\langle \tau,\delta t\,\dot{q}\rangle=\langle \tau,\delta q\rangle$ is the infinitesimal work. This is also known as the virtual work due to the generalized force $\tau$ being displaced by a virtual change of the co-ordinates $\delta q$. 
In co-ordinates if 
$\tau=(\tau_1,\tau_2,\cdots,\tau_n)$ then one can write 
\[
\delta w=\langle \tau,\delta q\rangle=\sum_{i=1}^n\tau_i\delta q_i.
\]
Thus we can use this expression to find the generalized force components $\tau_i$ once we write down the virtual work of the system.

Using the above notations the Euler-Lagrange equations of motion for the system are:
\begin{align}
\frac{d}{dt}\left(\frac{\partial L}{\partial{\dot{q}_i}}\right)-\frac{\partial L}{\partial{{q}_i}}=\tau_i,\:\:\:\:\:\:\mbox{for}\:\:\:\:\:\: i=1,2,\cdots,n.
\end{align}

Let us look at what these equations say about motion  when no non-conservative forces are present (that is $\tau\equiv 0$). In the absence of non-conservative forces, the integral 
\[
\mathcal{A}=\int_{t_1}^{t_2}L(q(t),\dot{q}(t))\,dt
\]
can be given the interpretation of \textit{total change} occurring in the system during the time interval $[t_1,t_2]$. It is commonly referred to as the total action during that time interval. What the Lagrangian equation then says is that the system evolves in such a way hat this total change (the total action) is a minimum. In other words, of all the possible paths of motion the actual motion is the one that minimizes the total change in the system. Dealing with these issues requires us a little bit of the theory of calculus of variations and is beyond the scope of this course.

\section{Examples}
\subsection{Inverted Pendulum on a Cart}
\begin{figure}[ht]
\begin{center}
 \includegraphics[width=2in]{PendulumCart}\\
\renewcommand{\baselinestretch}{1}\selectfont
\caption{Inverted Pendulum on a Cart.}
\label{Fig:InvertedPendulum}
\renewcommand{\baselinestretch}{1.5}\selectfont
\end{center}
\end{figure}


The configuration space is $\mathcal{Q}=\mathbb{R}\times \mathbb{S}$ with co-ordinates $(x,\phi)$.
The kinetic energy of the system is
\begin{equation}\label{eq:KE}
T=\frac{1}{2}\left((M+m)\,\dot{x}^2+(mL^2+I)\,\dot{\phi}^2+2mL\,\dot{\phi}\,\dot{x}\cos{\phi}\right).
\end{equation}
Here $M$ is the mass of the cart, $m$ is the mass of the pendulum, $L$ is the distance from the pivot point to the center of mass of the pendulum,
$I$ is the moment of inertia of the pendulum and $g$ is the gravitational acceleration.
\\
\\
Potential energy of the system is given by 
\begin{equation}\label{eq:PE}
U=mgL\cos\phi.
\end{equation}
The virtual work due to a virtual displacement of the co-ordinates $(\delta x,\delta \theta)$ is $\delta w=f \delta x$.
Thus the generalized external nonconservative forces are given by
$f=(u,0)$.
The Lagrangian is $L(x,\theta,\dot{x},\dot{\theta})=T-U$ and thus the Euler-Lagrange equations of the system are
\begin{align*}
\frac{d}{dt}\left(\frac{\partial L}{\partial{\dot{x}}}\right)-\frac{\partial L}{\partial{x}}&=u,\\
\frac{d}{dt}\left(\frac{\partial L}{\partial{\dot{\theta}}}\right)-\frac{\partial L}{\partial{\theta}}&=0,
\end{align*}

Which after re-arranging become:
\begin{eqnarray}
\ddot{x}&=&\frac{\left(\left( m\,{L}^{2}+I \right) u+{m}^{2}{L}^{3}\sin \left( { {\phi}} \right) {{ \dot{\phi}}}^{2}+m\,L\,I\,\sin \left( { {\phi}} \right) {{ \dot{\phi}}}^{2}-{m}^{2}{L}^{2}g\cos \left( { {\phi}} \right) \sin \left( { {\phi}} \right)\right)}{\left((m+M)I+M\,m\,{L}^{2}+{m}^{2}{L}^{2} \sin^2 {\phi}\right)} ,\label{eq:xddot}\\
\ddot{\phi}&=&\frac{ \left( -m^2\,L^2\,\sin \left( {\phi} \right)\cos \left( {\phi} \right) {{\dot{\phi}}}^{2}+mL(m+M)g\sin{\phi}-mL\,u\cos{\phi} \right)}{\left((m+M)I+M\,m\,{L}^{2}+{m}^{2}{L}^{2} \sin^2 {\phi}\right)}.\label{eq:thetaddot}
\end{eqnarray}
%%%%%%%%%%%%%%%%%%

\subsection{Falling and rolling disk}\label{Secn:FallingRollingDisk}
In section-\ref{Secn:FallingRollingDisk}
we have shown that the kinetic energy of the falling rolling disk is
\[
\mathrm{KE}=\frac{1}{2}\left(M\dot{x}^2+M\dot{y}^2+\mathbb{I}_r\dot{\phi}^2+(\mathbb{I}_p+Mr^2\sin^2{\alpha})\dot{\alpha}^2
+(\mathbb{I}_p\cos^2{\alpha}+\mathbb{I}_r\sin^2{\alpha})\dot{\theta}^2-2\mathbb{I}_r\dot{\phi}\dot{\theta}\sin{\alpha}\right)
\]
where $\mathbb{I}=\mathrm{diag}(\mathbb{I}_p,\mathbb{I}_r,\mathbb{I}_p)$ is the moment of inertia of the disk in a  body fixed frame and $M$ is the mass of the disk.
The potential energy is
\[
\mathrm{PE}=Mgr\cos{\alpha}.
\]

Thus the Lagrangian is
{
\[
L=
\frac{1}{2}\left(M\dot{x}^2+M\dot{y}^2+I_r\dot{\phi}^2+(I_p+Mr^2\sin^2{\alpha})\dot{\alpha}^2
+(I_p\cos^2{\alpha}+I_r\sin^2{\alpha})\dot{\theta}^2-2I_r\dot{\phi}\dot{\theta}\sin{\alpha}\right)-Mgr\cos{\alpha}.
\]
}
The external generalized forces acting on the system are
\[
\omega_1=\tau_{\phi}d\phi,\:\:\:\:\:\:\omega_2=\tau_{\alpha}d\alpha.
\]
The non-holonomic constraint 1-forms are
\begin{eqnarray}
\omega_2 &=&\sin{\theta}dx-\cos{\theta}dy,\label{eq:Constraint1}\\
\omega_3 &=&\cos{\theta}dx+\sin{\theta}dy-rd\phi.\label{eq:Constraint2}
\end{eqnarray}
The non-holonomic constraints are given by
\begin{eqnarray*}
\omega_2(\dot{q}) &=&\sin{\theta}\dot{x}-\cos{\theta}\dot{y}=0,\\
\omega_3(\dot{q}) &=&\cos{\theta}\dot{x}+\sin{\theta}\dot{y}-r\dot{\phi}=0.
\end{eqnarray*}
Which results in
\begin{eqnarray}
\dot{x} &=&r\dot{\phi}\cos{\theta}\label{eq:dotx}\\
\dot{y} &=&r\dot{\phi}\sin{\theta}.\label{eq:doty}
\end{eqnarray}

Using the Euler-Lagrange equations for the falling and rolling disk given by,
\[
\frac{d}{dt}\left(\frac{\partial L}{\partial \dot{q}_i}\right)-\frac{\partial L}{\partial q_i}=\omega_1+\omega_2+\lambda_3\omega_3+\lambda_2\omega_4,
\]
we have
{\small
\begin{eqnarray}
(I_p\cos^2\alpha+I_r\sin^2\alpha)\ddot{\theta}-I_r\sin\alpha \ddot{\phi} &=&-2(I_r-I_p)\dot{\alpha}\dot{\theta}\sin\alpha\cos\alpha+I_r\dot{\alpha}\dot{\phi}\cos\alpha\\
(I_p+Mr^2\sin^2{\alpha})\ddot{\alpha} &=& -Mr^2\dot{\alpha}^2\sin\alpha \cos\alpha+(I_r-I_p)\dot{\theta}^2\sin\alpha\cos\alpha-I_r\dot{\phi}\dot{\theta}\cos\alpha+Mgr\sin\alpha+
\tau_{\alpha},\nonumber\\
&& \\
(I_r+Mr^2)\ddot{\phi}-I_r\sin\alpha\ddot{\theta} &=& I_r\dot{\theta}\dot{\alpha}\cos\alpha+\tau_{\phi}
\end{eqnarray}
}




%%%%%%%%%%%%%%%%%%%%%%
\subsection{Double Pendulum}
\[
\left(\begin{array}{c} \mathrm{d\theta_1}\\ -\frac{3\, \mathrm{L_1}\, \mathrm{m_2}\, \sin\!\left(2\, \mathrm{\theta_1} - 2\, \mathrm{\theta_2}\right)\, {\mathrm{d\theta_1}}^2 + 6\, \mathrm{L_2}\, \mathrm{m_2}\, \sin\!\left(\mathrm{\theta_1} - \mathrm{\theta_2}\right)\, {\mathrm{d\theta_2}}^2 + 3\, g\, \mathrm{m_1}\, \sin\!\left(\mathrm{\theta_1}\right) - 3\, g\, \mathrm{m_2}\, \cos\!\left(\mathrm{\theta_1} - \mathrm{\theta_2}\right)\, \sin\!\left(\mathrm{\theta_2}\right)}{2\, \mathrm{L_1}\, \left( - 3\, \mathrm{m_2}\, {\cos\!\left(\mathrm{\theta_1} - \mathrm{\theta_2}\right)}^2 + \mathrm{m_1} + 3\, \mathrm{m_2}\right)}\\ \mathrm{d\theta_2}\\ \frac{\frac{3\, \mathrm{L_2}\, \mathrm{m_2}\, \sin\!\left(2\, \mathrm{\theta_1} - 2\, \mathrm{\theta_2}\right)\, {\mathrm{d\theta_2}}^2}{2} + \sin\!\left(\mathrm{\theta_1} - \mathrm{\theta_2}\right)\, \left(\mathrm{L_1}\, {\mathrm{d\theta_1}}^2\, \mathrm{m_1} + 3\, \mathrm{L_1}\, {\mathrm{d\theta_1}}^2\, \mathrm{m_2}\right) + \frac{g\, \mathrm{m_1}\, \sin\!\left(\mathrm{\theta_2}\right)}{4} - \frac{3\, g\, \mathrm{m_2}\, \sin\!\left(\mathrm{\theta_2}\right)}{2} + \frac{3\, g\, \mathrm{m_1}\, \sin\!\left(2\, \mathrm{\theta_1} - \mathrm{\theta_2}\right)}{4}}{\mathrm{L_2}\, \left(3\, \mathrm{m_2}\, {\sin\!\left(\mathrm{\theta_1} - \mathrm{\theta_2}\right)}^2 + \mathrm{m_1}\right)} \end{array}\right)
\]

%%%%%%%%%%%%%%%%%%%




\newpage
\chapter{Exercises}\label{secn:Exercises}

\section{Exercises on Particle Motion}\label{Secn:ExercisesParticleMotn}





\begin{exercise}\label{HomgenetyInertialFrame}
Let $\mathbf{e}$ and $\mathbf{e}'$ be two inertial observers and let $A$ be some space-time event. Let the quadruple $(t,x)\in \mathbb{R}^4$, where $t\in \mathbb{R}$ and  $x\in \mathbb{R}^3$, be the representation of the space-time event $A$ that corresponds to $\mathbf{e}$ while let  $(\tau,\xi)\in \mathbb{R}^4$ where $\tau\in \mathbb{R}$ and  $\xi\in \mathbb{R}^3$ be the representation of the space-time event $A$ that corresponds to $\mathbf{e}'$. 
When comparing the motion described in the two frames we need to know how the two representations (coordinates) are related to each other. Specifically we will show that inertial observers must necessarily be translating at constant velocity with respect to each other without rotations. We do this by showing the following:
\begin{enumerate}[(a)]
\item The assumption that time is homogeneous and that all intervals of time are inertial observer invariant means that necessarily $\tau=t+a$ where $a$ is a constant.
\item Homogeneity of space implies that necessarily  $\xi=\alpha+\beta t+R x$ where $\alpha,\beta$ are constant $3\times 1$ matrices and $R$ is a constant $3\times 3$ matrix. 
\item The assumption that space intervals are inertial observer independent implies that $R$ is an orthonormal constant transformation (that is $R^TR=RR^T=I$).
\item Let $O'$ be the origin of the orthonormal frame used by $\mathbf{e}'$ to make spatial measurements. If the space-time event $O'$ has the representation $(t,o)$ according to the observer $\mathbf{e}$ then since $v=\dot{o}=-R^{T}\beta=\mathrm{constant}$ we see that the velocity of the $\mathbf{e}'$ frame with respect to the $\mathbf{e}$ given by $v=\dot{o}$ must be a constant.  If both clocks of $\mathbf{e}$ and $\mathbf{e}'$ are synchronized (that is $a=0$) and if a certain space-time event $A$ has the representation $(t,x)$ according to $\mathbf{e}$ then the space-time event $A$ has the representation $(t,R(x-vt))$ according to $\mathbf{e}'$.
\end{enumerate}
\end{exercise}

\begin{exercise}\label{eq:NewtonsLaws}
Using the principle of conservation of linear momentum in inertial frames, of a system of interacting and isolated set of particles, derive Newton's three laws of motion.
\end{exercise}

\begin{exercise}\label{ex:ParticleOnParabola}
A bead that is constrained to move on a frictionless wire that lies on a horizontal plane. The wire is bent to a shape of a parabola (ie. $y=x^2$). Find the constraint forces that keep the 
bead on the wire and the equation of motion of the bead. Assume that the bead does not interact with any other objects other than the wire. 
\end{exercise}


%\begin{exercise}
%For each of the systems shown in figures \ref{Fig:SMD_System_Exercises} to \ref{Fig:ThreeDOFTransSysm}, derive the governing differential equations using Newton's law. State all assumptions made.
%\begin{figure}[ht]
%\begin{center}
%\includegraphics[width=2in]{SMD_System}
%\renewcommand{\baselinestretch}{1}\selectfont
%\caption{A simple spring mass damper system where the mass is constrained to move horizontally. Assuming that the deflections and the velocity of the mass are small the viscous force $f_v$ exerted by the damper can be assumed to be approximately proportional to the relative displacement between the piston and the cylinder of the damper.}
%\label{Fig:SMD_System_Exercises}
%\renewcommand{\baselinestretch}{1.5}\selectfont
%\end{center}
%\end{figure}


%\begin{figure}[ht]
%\begin{center}
%includegraphics[width=3.5in]{TwoDOFTransSysm}
%\renewcommand{\baselinestretch}{1}\selectfont
%\caption{Two masses coupled by springs and dampers. The dry friction forces between the two masses and the horizontal surface given by $f_{v_1}$ and $f_{v_2}$ may be assumed to be negligibly small while the viscous friction force $f_{v_3}$ exerted by the damper can be assumed to be proportional to the relative velocity of the piston and the cylinder of the damper.}
%\label{Fig:TwoDOFTransSysm}
%\renewcommand{\baselinestretch}{1.5}\selectfont
%\end{center}
%\end{figure}


%\begin{figure}[ht]
%\begin{center}
%\includegraphics[width=3.5in]{ThreeDOFTransSysm}
%\renewcommand{\baselinestretch}{1}\selectfont
%\caption{Three masses coupled by springs. The dry friction forces between the two masses and the horizontal surface given by $f_{v_1}$ and $f_{v_2}$ may be assumed to be negligibly small while the dry friction between the masses $M_1$ and $M_2$ and  $M_1$ and $M_2$ denoted by $f_{v_3}$ and $f_{v_4}$ are of non-negligible magnitude.}
%\label{Fig:ThreeDOFTransSysm}
%\renewcommand{\baselinestretch}{1.5}\selectfont
%\end{center}
%\end{figure}
%\end{exercise}

\begin{exercise}
Consider the 1-DOF electrostatic MEMS mirror model  shown in figure \ref{Fig:1D_MEMS}. Find the governing differential equations of the system. Note that the capacitance between 
two parallel plate capacitors are given by $c(l(t))=\frac{\epsilon A}{l(t)}$ and the attractive coulomb force between the two plates are given by $f_e(t)=\frac{q(t)^2}{2\epsilon A}$ where 
$A$ is the cross sectional area of the plates.

\begin{figure}[ht]
\begin{center}
\begin{tabular}{cc}
\includegraphics[width=3.5in]{OneDMirrorModel}
\end{tabular}
\caption{1-DOF Electrostatic MEMS model. Let $l_0$ be the zero voltage gap and let $l(t)$ be the gap length, $v(t)$ be the voltage across the plate, and  $i(t)$ be the current in the circuit at a given time $t$.} \label{Fig:1D_MEMS}
\end{center}
\end{figure}
\end{exercise}


\begin{exercise}\label{ex:ParticleOnSphere}
A particle $P$ of mass $m$ is constrained to move on the surface of a sphere of radius $r$ and origin $O$ by attaching $P$ to $O$ using a tight in-elastic wire (refer to figure 
\ref{Fig:ParticleOnSphere0}). Except for the tension of the wire no other external forces act on the particle. Let frame $e$ be a frame with origin coinciding with $O$ and fixed with 
respect to the sphere. The representation of the point $P$ with respect to $e$ is $x$. Answer the following:
\begin{figure}[ht]
\begin{center}
\includegraphics[width=3in]{ParticleOnSphere_e}
\renewcommand{\baselinestretch}{1}\selectfont
\caption{}
\label{Fig:ParticleOnSphere0}
\renewcommand{\baselinestretch}{1.5}\selectfont
\end{center}
\end{figure}


\begin{enumerate}
\item \label{ex:ParticleOnSphere1} Show that the velocity of the particle is always tangential to the sphere (orthogonal to the position $x$, ie. $x^T\dot{x}=0$).
\item \label{ex:ParticleOnSphere3} Differentiating the constraint $x^T\dot{x}=0$ show that the motion of the particle in the $e$ frame is described by
\[
\ddot{x}=\frac{1}{m}f_c(t)=-\frac{||\dot{x}||^2}{r^2} x,
\]
where $f_c(t)$ is the representation of the tension force in the wire.
\item Write down the equations of motion as observed in a frame $b(t)$ that is moving with respect to the fixed frame $e$.

\end{enumerate}
\end{exercise}

\begin{exercise}\label{ex:KinematicsNKineticsInElevator}
{\sf --- Einstein's box experiment:} Explain why a person standing on a scale inside an elevator sees his or her weight doubled as the elevator accelerates up at a rate of $g$ and 
sees the weight reduced to zero if the elevator decelerates at a rate of $g$. Also show that if, for some reason, the gravitational force field vanished and the elevator was moving up at 
an acceleration of $g$ then the scale would still show the correct weight of the person.
\end{exercise}

%\begin{exercise}
%For the system shown in figure \ref{Fig:MovingFrameTranslation},
%\begin{enumerate}
%\item Derive the equations of motion as observed by an observer in a frame fixed to the surface on which the box is moving.
%\item Derive the equations of motion as observed by an observer moving with the box (ie. in the moving frame co-ordinates).
%\end{enumerate}
%\begin{figure}[ht]
%\begin{center}
%\includegraphics[width=4.5in]{MovingFrameTranslation}
%\renewcommand{\baselinestretch}{1}\selectfont
%\caption{Spring Mass Damper System in a Moving Box}
%\label{Fig:MovingFrameTranslation}
%\renewcommand{\baselinestretch}{1.5}\selectfont
%\end{center}
%\end{figure}
%\end{exercise}

\begin{exercise}\label{ex:RotationMatrix}
{\sf --- Properties of a Rotation Matrix:} Let $\mathbf{e}$ and $\mathbf{b}$ be two orthonormal, right handed oriented frames with coinciding origin $O$. Let $\mathbf{b}=\mathbf{e}R$ be the relationship between the two frames. Show that $R^TR=RR^T=I_{3\times 3}$ and $\mathrm{det}(R)=1$.
\end{exercise}

%%%%%%%%%%%%%%%%%%%%%%%%%%%%%%%

\begin{exercise}\label{ex:RotatedFrames}

The three rotated frames $\mathbf{a},\mathbf{b},\mathbf{c}$ are related to a fixed frame $\mathbf{e}$ as shown in figure \ref{Fig:TwoDRotatnFrames2}. All frames are orthonormal. Let $\mathbf{a}=\mathbf{e}\, R_1{(\theta_1)}$, $\mathbf{b}=\mathbf{e} \,R_2{(\theta_2)}$,
and $\mathbf{c}=\mathbf{e}\, R_3{(\theta_3)}$. Show using direct calculations the following

\[
R_1{(\theta_1)}=\left[\begin{array}{ccc}
1 & 0 & 0\\
0 & \cos{\theta_1} & -\sin{\theta_1}\\
0 & \sin{\theta_1} & \cos{\theta_1}
\end{array}\right],\:\:\:\:
R_2{(\theta_2)}=\left[\begin{array}{ccc}
\cos{\theta_2} & 0 & \sin{\theta_2}\\
0 & 1 & 0\\
- \sin{\theta_2}& 0 & \cos{\theta_2}
\end{array}\right],\:\:\:\:
R_3{(\theta_1)}=\left[\begin{array}{ccc}
\cos{\theta_1} & -\sin{\theta_1} & 0\\
\sin{\theta_1} & \cos{\theta_1} &0\\
0 & 0 & 1
\end{array}\right]
\]
and
\[
R_1^T\dot{R}_1=\widehat{\Omega}_1=\left[\begin{array}{ccc}
0 & 0 & 0\\
0 & 0 & -\dot{\theta_1}\\
0 & \dot{\theta_1} & 0
\end{array}\right],\:\:\:\:
R_2^T\dot{R}_2=\widehat{\Omega}_2=\left[\begin{array}{ccc}
0 & 0 & \dot{\theta_2}\\
0 & 0 & 0\\
- \dot{\theta_2}& 0 & 0
\end{array}\right],\:\:\:\:
R_3^T\dot{R}_3=\widehat{\Omega}_3=\left[\begin{array}{ccc}
0 & -\dot{\theta}_1 & 0\\
\dot{\theta}_1 & 0 &0\\
0 & 0 & 0
\end{array}\right]
\]
and
\[
\widehat{\Omega}_1^2=-\dot{\theta}_1^2\left[\begin{array}{ccc}
0 & 0 & 0\\
0 & 1 & 0\\
0 & 0 & 1
\end{array}\right],\:\:\:\:
\widehat{\Omega}_2^2=-\dot{\theta}_2^2\left[\begin{array}{ccc}
1 & 0 & 0\\
0 & 0 & 0\\
0 & 0 & 1
\end{array}\right],\:\:\:\:
\widehat{\Omega}_3^2=-\dot{\theta}_3^2\left[\begin{array}{ccc}
1 & 0 & 0\\
0 & 1 & 0\\
0 & 0 & 0
\end{array}\right].
\]

\begin{figure}[ht]
\begin{center}
\begin{tabular}{ccc}
\includegraphics[width=2in]{TwoDRotatn1V2} & \includegraphics[width=2in]{TwoDRotatn2V2} & \includegraphics[width=2in]{TwoDRotatn3V2}\\
Rotation about $\mathbf{e}_1$ & Rotation about $\mathbf{e}_2$ & Rotation about $\mathbf{e}_3$ 
\end{tabular}
\renewcommand{\baselinestretch}{1}\selectfont
\caption{Rotated Frames}
\label{Fig:TwoDRotatnFrames2}
\renewcommand{\baselinestretch}{1.5}\selectfont
\end{center}
\end{figure}

\end{exercise}
%%%%%%%%%%%%%%%%%%%%%%%%%%%%%%%

\begin{exercise}\label{ex:SkewNCross}
Show that the space of $3 \times 3$ skew-symmetric matrices can be identified with $\mathbb{R}^3$ using the identification
$\Omega \mapsto \widehat{\Omega}$ where
\begin{equation*}\label{eq:SkewSymmetric1}
\widehat{\Omega}=\left[ \begin{array}{ccc} 0 & -\Omega_3 & \Omega_2 \\ \Omega_3 & 0 & -\Omega_1 \\ -\Omega_2 & \Omega_1 & 0\end{array}\right],
\end{equation*}
and $\Omega =[\Omega_1\:\:\:\Omega_2\:\:\:\Omega_3]^T$. That is show that the map
$\:\:\:\widehat{}\::\mathbb{R}^3\to so(3)$ is a one-to-one and onto map.
\end{exercise}



\begin{exercise}\label{ex:SkewNCross1}
Show that the cross product in $\mathbb{R}^3$ satisfies
\begin{equation*}\label{eq:CrossProduct}
\Omega \times X=\widehat{\Omega}X,
\end{equation*}
where $\Omega,X\in\mathbb{R}^3$ and $\:\:\:\widehat{}\::\mathbb{R}^3\to so(3)$ is the one-to-one and onto map discussed in exercise-\ref{ex:SkewNCross}.
\end{exercise}




\begin{exercise}\label{ex:PropertiesCrossProduct} Prove the following properties of the cross product in $\mathbb{R}^3$.
\begin{enumerate}
\item $A\times B=-B\times A$
\item $A\times B\times C+B\times C\times A+C\times A\times B=0$
\item $A\cdot(B\times C)=C\cdot(A \times B)=B\cdot(C \times A)$
\item $A\times(B\times C)=(A\cdot C)B-(B\cdot A)C$ \label{ex:PropertiesCrossProduct_b}
\item $(\Omega\times X)\cdot(\omega\times X)=(\Omega \cdot \omega)||X||^2-(\Omega \cdot X)(\omega \cdot X)$
\item $\Omega \cdot (\Omega \times X)=0=X\cdot(\Omega \times X)$
\end{enumerate}
Here we use the notation $A\cdot B$ to denote the usual inner product in $\mathbb{R}^3$.
\end{exercise}

%%%%%%%%%%%%%%%%%%%%%%%%%%%%%%%%

\begin{exercise}
Using the last property proven in exercise-\ref{ex:PropertiesCrossProduct} show that $(\Omega \times X)$ is perpendicular to both $X$ and $\Omega$.
\end{exercise}
%%%%%%%%%%%%%%%%%%%%%%%%%%%%%%%%
\begin{exercise}\label{ex:XhatSqrd}
Show using exercise \ref{ex:PropertiesCrossProduct} part \ref{ex:PropertiesCrossProduct_b} (or direct verification) that
\begin{align*}
\widehat{X}^2&= XX^T-||X||^2I_{3\times3}. 
\end{align*}
for any $X\in \mathbb{R}^3$.
\end{exercise}

%%%%%%%%%%%%%%%%%%%%%%%%%%%%%%%

\begin{exercise}\label{ex:RotaedCrossProduct}
If $R\in \mathrm{SO}(3)$, and $X$ and $Y$ are $3\times 1$ matrices show that the following identities hold:
\begin{align*}
R(X\times Y)&=RX\times RY\\
\widehat{RX}&=R\widehat{X}R^T.
\end{align*}
\end{exercise}


%%%%%%%%%%%%%%%%%%%%%%%%%%%%%%%%%

%%%%%%%%%%%%%%%%%%%%%%%%%%%%%
\begin{exercise}\label{ex:InfinitesimalRotations}
Let $\mathbf{e}$ and $\mathbf{b}(t)$ be two orthonormal, right handed oriented frames with coinciding origin $O$. Frame $\mathbf{e}$ is fixed while frame $\mathbf{b}$ is varying with time. Let $\mathbf{b}(t)=\mathbf{e}R(t)$ be the relationship between the two frames. Let $R^T\dot{R}=\widehat{\Omega}$ and $\omega=R\Omega$.
Show that the frame $\mathbf{b}(t)$ is instantaneously rotating about an axis in space that has the representation $\omega$ in the $\mathbf{e}$-frame and $\Omega$ in the $\mathbf{b}(t)$-frame at a rate equal to $||\omega||=||\Omega||$. 
\end{exercise}


%%%%%%%%%%%%%%%%%%%%%%%%%%%%%%%%%
\begin{exercise} {\bf \sf Particle motion in a circle:}\label{ex:ParticleOnCircle} Consider a particle constrained to move in a circle of radius $r$. Let $\mathbf{e}=[\mathbf{e}_1\:\:\:\mathbf{e}_2\:\:\:\mathbf{e}_3]$ be an orthonormal frame fixed on the circle such that $\mathbf{e}_3$ is perpendicular to the plane of the circle. Let $x$ be the representation of the particle with respect to $e$.
\begin{enumerate}
\item Using the results of exercise \ref{ex:ParticleOnSphere} show that the velocity of the particle $\dot{x}$ is tangential to the circle.
\item Using a frame $b(t)$ that is moving with the particle (ie. the particle appears fixed in $\mathbf{b}(t)$), show that $||\dot{x}||=r\Omega$ where $\Omega$ is the angular velocity of the frame $b$ with respect to $e$.
\item \label{exx:ss}  Show that the constraint force is radial and is given by $f_c(t)=-m\,\Omega^2 x$ and
that the motion of the particle in the $e$ frame is described by
\[
\ddot{x}=-\Omega^2 x.
\]
\item Show using Newton's equations in the $\mathbf{b}(t)$ frame  that the angular velocity, $\Omega$, of the particle is a constant.

\item Show that the position components $x_1$ and $x_2$ behave sinusoidally (ie. $x_1(t)=r\sin{(\Omega t+\phi_1)}$ and $x_2(t)=r\sin{(\Omega t+\phi_2)}$ ).
\end{enumerate}
\end{exercise}



%%%%%%%%%%%%%%%%%%%%%%%%%%
\begin{exercise}\label{Ex:RotatingDiskMassInSlot}
Consider the ball of mass $m$ constrained to move as shown in figure \ref{Fig:RotatingDiskMassInSlot3}. The orthonormal frame $\mathbf{e}$ is an earth fixed frame with origin at the centre of the disk. The orthonormal frame $b$ is fixed to the disk as shown in the figure. The disk is rotating about the $\mathbf{e}_3$ axis. Gravity acts in the negative $\mathbf{e}_3$ direction and the disk and the mass lies on a smooth horizontal surface. If the un-stretched length of the spring is $y_0$ write down the equations of motion of the mass.
\begin{figure}[h]
\begin{center}
\includegraphics[width=2.5in]{disk_2D.png}
\renewcommand{\baselinestretch}{1}\selectfont
\caption{A disk lying on a horizontal plane while rotating about a vertical axis.}
\label{Fig:RotatingDiskMassInSlot3}
\renewcommand{\baselinestretch}{1.5}\selectfont
\end{center}
\end{figure}
\end{exercise}


\begin{exercise}
Consider a ball of mass $m$ constrained to move as shown in figure \ref{Fig:RotatingDiskMassInSlot2}. The orthonormal frame $\mathbf{e}$ is an earth fixed frame with origin at the centre of the disk. The orthonormal frame $\mathbf{c}$ is fixed to the disk. The disk is rotating about a vertical axis (that is about the $\mathbf{e}_3$ axis) and about an axis through its centre that is perpendicular to the disk (that is about the $\mathbf{c}_1$ axis). Write down the equations of motion of the mass.
\begin{figure}[h]
\begin{center}
\includegraphics[width=4in]{RotatingDiskMassInSlot2}
\renewcommand{\baselinestretch}{1}\selectfont
\caption{A disk rotating about a vertical axis while spinning about an axis perpendicular to the disk.}
\label{Fig:RotatingDiskMassInSlot2}
\renewcommand{\baselinestretch}{1.5}\selectfont
\end{center}
\end{figure}
\end{exercise}

\begin{exercise}
Consider the mechanical system shown in figure \ref{Fig:RotatingSpringPendulum}.  The mass 
of the point $P$ is $m$. The un-stretched length of the spring is $L_0$. Neglecting friction and the moment of inertia of the spring derive the equations of motion of the system if the arm is rotating at a constant angular velocity of $\Omega$.
\begin{figure}[h]
\begin{center}
\includegraphics[width=2.2in]{RotatingSpringPendulum}
\renewcommand{\baselinestretch}{1}\selectfont
\caption{Rotating Spring Pendulum. Figure copied from G. T. Greenwood.}
\label{Fig:RotatingSpringPendulum}
\renewcommand{\baselinestretch}{1.5}\selectfont
\end{center}
\end{figure}
\end{exercise}
%%%%%%%%%%%%%%%%%%




\begin{exercise}\label{ex:MEMSgyro}
Consider the setup shown in figure \ref{Fig:MEMSGyro}. The disk can rotate in a horizontal plane. $C$ is a parallel plate capacitor. The fixed plate is rigidly attached to the center of 
the disk $O$ while the movable plate is rigidly attached to the mass $P$. The end of the spring of stiffness $k_l$, (the point $O'$) can freely move inside the circular slot. When the 
mass is at a distance $l_0$ away from the origin the spring $k_l$ is unstretched. It can be assumed that $OPO'$ remains a straight line. Assuming small values of $\theta$ the force 
exerted by the spring $k_{\theta}$ can be assumed to equal $k_{\theta}l\theta$ and act in a direction perpendicular to $OP$ and in the plane of the disc where $l$ is the distance $OP
$ (this spring is a simplified representation of a torsional spring and hence assume that it does physically obstruct the angular motion of $OPO'$).
When a voltage is applied across the capacitor plates the resulting Coulomb forces exert a force, $f_c(t)$ on the particle. The voltage across the capacitor is varied such that 
$f_c(t)=m\sin{(\omega t)}$.

\begin{enumerate}
\item If the disc is rotating at a constant angular rate of $\Omega$ derive the equations of motion of the particle.
\item Simulate the motion of the particle using MATLAB. Use  $r=2$, $l_0=1$, $k_l/m=(2\pi\times 100)^2$, $k_{\theta}/m=(2\pi\times 10)^2$, $\omega=2\pi\times 11$, $\Omega=300\; 
\mathrm{r.p.m}$ and initial conditions $l(0)=l_0, \dot{l}(0)=0,
\theta(0)=0, \dot{\theta}(0)=0$. Show all necessary graphs and attach all MATLAB work that is needed to produce these graphs.
\item Leaving all other parameters constant and using the same initial conditions as above simulate the motion for three different values of $\Omega$ in the range of $100 \; 
\mathrm{r.p.m}$ to  $1000 \; \mathrm{r.p.m}$
\item Using above results discuss how this device may be used to measure the angular rate of change of the disk. This captures the basic operating principle of a vibrating MEMS 
gyroscope.
\item Discuss what modifications you would do to change the operating range.
\item Write a two page introduction to MEMS gyroscopes. The introduction should include methods of fabrication, device types and usages and the current state of the art.
\end{enumerate}
\begin{figure}[ht]
\begin{center}
\includegraphics[width=4.5in]{VibratingGyro}
\renewcommand{\baselinestretch}{1}\selectfont
\caption{The Basic Operating Principle of a Vibrating MEMS Gyroscope}
\label{Fig:MEMSGyro}
\renewcommand{\baselinestretch}{1.5}\selectfont
\end{center}
\end{figure}
\end{exercise}

%\begin{exercise}\label{ex:MEMSGyro}
%{\sf ---  Self study on vibrating MEMS Gyroscopes:} The operation of a vibrating MEMS Gyroscope relies on the presence Coriolis forces. Write a two page report on vibrating MEMS 
%gyroscopes. Include in the report an explanation of its working principle, its use, and methods of fabrication.
%\end{exercise}

\begin{exercise}\label{ex:Hurricane}
Explain using equation (\ref{eq:AppearentForcesEarth2}) why a Hurricane that has formed in the Northern hemisphere has a counter clockwise rotation and a Hurricane that has 
formed in the Southern hemisphere has a clockwise rotation (see figure \ref{Fig:Hurricane}). 
\end{exercise}

\begin{exercise}\label{ex:Cannon}
At a point on the equator a cannon is fired towards the West with an initial velocity of 
$100\, m/s$ and a firing angle of $45^{o}$ with the equator. Taking into consideration the rotation 
of the Earth. Find the horizontal distance to the landing point of the cannon. Compare these results with those obtained by neglecting the effects of rotation of the Earth.
\end{exercise}

\begin{exercise}\label{ex:EarthAroundSun}
Prove that the Earth's orbit around the sun is an ellipse (Kepler's First Law). You may assume that the Earth and the Sun are point particles of mass $M_E$ and $M_S$ respectively 
and that the Gravitational attraction force between two masses is of magnitude $\frac{GM_SM_E}{r^2}$ where $G$ is the universal gravitational constant and $r$ is the distance 
between the Earth and the Sun. Recall that the equation of an ellipse is given by
\[
r=r_0\frac{1+e}{1-e\cos{\theta}}
\]
where $e=\frac{r_1-r_0}{r_1+r_0}$ is the eccentricity and
$\frac{b}{a}=\sqrt{1-e^2}$ (refer to figure \ref{Fig:Ellipse}).
\begin{figure}[h]
\begin{center}
\includegraphics[width=3.5in]{Ellipse}
\renewcommand{\baselinestretch}{1}\selectfont
\caption{}
\label{Fig:Ellipse}
\renewcommand{\baselinestretch}{1.5}\selectfont
\end{center}
\end{figure}

\end{exercise}







%%%%%%%%%%%%%%%%%%%%%%%%%%%%%%%%%%%%%
%\section{Exercises on Rigid Body Motion}\label{Secn:ExercisesRigidBodyMotion}

\begin{exercise}\label{ex:InstantaneousCenter}
Let $\mathcal{B}_1$ and $\mathcal{B}_2$ be two rigid bodies moving in 2-dimensional Euclidean space.  Show that there exists a point in both bodies with zero relative velocity. That is, viewed 
from either body there is a point in the other that appears to be instantaneously  fixed .
Such a \emph{point} is called an {\textit{instantaneous center}}. 
\end{exercise}


\begin{exercise}\label{ex:KennedysTheorem}\emph{Kennedy's Theorem:}
Prove the following important result applicable for two-dimensional mechanisms. 
Three rigid bodies in relative motion with each other will have 
their respective instantaneous centers lying in a straight line.
\end{exercise}


\begin{exercise}\label{ex:ParalleAxisTheorem}

Let $\mathbb{I}$ be the moment of inertia of a rigid body in a body fixed frame $\mathbf{b}$ fixed at a point $O$. 
Let $\mathbb{I}_c$ be the moment of inertia of the same rigid body in a frame parallel to $\mathbf{b}$ but fixed at the center of mass of the body $O_c$. Let $\bar{X}$ be the representation of the center of mass in the frame $\mathbf{b}$. If the total mass of the body is $M$,
show that $\mathbb{I}= \mathbb{I}_c-M\widehat{\bar{X}}^2$ (the parallel axis theorem).
\end{exercise}

\begin{exercise}
Prove the following properties about the inertia tensor $\mathbb{I}$ of a non-degenrate 3D rigid body:
\begin{enumerate}
\item $\mathbb{I}$ is symmetric and positive definite.
\item There exists a frame $\mathbf{b}$ such that $\mathbb{I}$ is diagonal.
\end{enumerate}
\end{exercise}

%\begin{exercise}\label{ex:PerpAxisTheorem}
% Show that for an axisymmetric rigid body (a body with an axis of rotational symmetry) the principle moments of inertia are related by $\mathbb{I}_1=\mathbb{I}_2=\mathbb{I}_3/2$. 
%\end{exercise}



\begin{exercise}\label{ex:PropertiesOfI}
Prove that the inertia tensor $\mathbb{I}$ is symmetric and positive definite.
\end{exercise}

%%%%%%%%%%%%%%
\begin{exercise}\label{ex:PointRigidBody}
Consider a rigid body that is made of three point particles. The first two particles are of mass $m$ and are positioned at $X_1=[1\:\:\:1\:\:\:1]^T$ and $X_2=[-1\:\:\:-1\:\:\:1]^T$ respectively while the third particle is of mass $2m$ and is positioned at, $X_3=[0\:\:\:0\:\:\:-1]^T$ with respect to the body frame $\mathbf{b}$. Write down explicitly the inertia tensor $\mathbb{I}$ in the body frame $b$. 
\end{exercise}

\begin{exercise}\label{ex:MomentInertia}
Calculate the moment of Inertia tensor $\mathbb{I}$ of each of the solid figures shown in figure \ref{Fig:Inert}.
\begin{figure}[h]
\begin{center}
\begin{tabular}{cccc}
\includegraphics[width=1.25in]{Moment_of_inertia_rod_end}&\includegraphics[width=1.25in]{Moment_of_inertia_solid_cylinder}&\includegraphics[width=1.25in]
{Moment_of_inertia_cone}
&\includegraphics[width=1.25in]{Moment_of_inertia_hollow_sphere}
\end{tabular}
\renewcommand{\baselinestretch}{1}\selectfont
\caption{Calculate the inertia tensor of each of these solid shapes.}
\label{Fig:Inert}
\renewcommand{\baselinestretch}{1.5}\selectfont
\end{center}
\end{figure}
\end{exercise}





\begin{exercise}\label{ex:ParalleAxisTheorem}
Find the moment of inertia tensor of the shaft and arm system shown in figure \ref{Fig:RotatingSpringPendulum}. The radius of the shaft is $r_s$, the length of the shaft is $l_s$ and the radius of the arm is $r_a$.
\end{exercise}



%%%%%%%%%%%%%


\begin{exercise}\label{ex:BeadOnHoop}
Consider the mechanical system shown in the figure \ref{Fig:BeadOnHoop}. The moment of inertia of the hoop about the vertical axis of rotation is $\mathbb{I}_z$. The radius of the hoop is $r$. 
The mass of the bead on the hoop is $m$. A viscous frictional torque acts about the vertical shaft of rotation. The magnitude of this torque is proportional to the rotational speed about 
the vertical axis and $C$ is the constant of proportionality. Answer the following questions.
Derive the equations of motion of the system.
For $\mathbb{I}_z=1, m=1,r=1,C=1, g=1$ simulate using MATLAB the behaviour of the system for the case where initially the mass point $P$ is displaced by an angle $7\pi/6$ from the vertical.
\begin{figure}[h]
\begin{center}
\includegraphics[width=2in]{BallHoop2}
\renewcommand{\baselinestretch}{1}\selectfont
\caption{Bead on a Hoop.}
\label{Fig:BeadOnHoop}
\renewcommand{\baselinestretch}{1.5}\selectfont
\end{center}
\end{figure}
\end{exercise}

\begin{exercise}\label{ex:FreeForcedVibrationP}
Figure \ref{Fig:ForcedVibrationApparatus} shows a schematic representation of the Free/Forced vibration apparatus in the applied mechanics lab. The apparatus consists of a beam 
pivoted at $O$. A damper is attached to the point $A$ and a spring is attached to the point $B$. The other end of the spring, $P$, is constrained to move vertically. When the beam is 
horizontal and the point $P$ has zero displacement with respect to the reference, that is when $\theta=0$ and $y=0$, the spring is at its natural length (unstretched or 
uncompressed). The spring constant is $k$ the damping constant is $C$ and the moment of inertia of the beam about $O$ is $I$. Let $OA=L_c$ and $OB=L_k$. Neglecting the 
thickness of the beam and assuming that the angle of deflection of the beam is small find the governing equations of the motion.
\begin{figure}[ht]
\begin{center}
\begin{tabular}{cc}
\includegraphics[width=4.5in]{ForcedVibrationDeflection}
\end{tabular}
\caption{A schematic representation of the Free/Forced Vibration Apparatus in the Applied Mechanics Lab.} \label{Fig:ForcedVibrationApparatus}
\end{center}
\end{figure}
\end{exercise}



\begin{exercise}\label{ex:RotatingPend}
Consider the mechanical system shown in figure \ref{Fig:RotatingSpringPendulum}. The moment of inertia of the shaft plus the arm about the vertical axis of rotation is $\mathbb{I}_z$. The mass 
of the point $P$ is $m$. The un-stretched length of the spring is $L_0$. Neglecting friction and the moment of inertia of the spring derive the equations of motion of the system and 
simulate them using MATLAB. Also find the Kinetic Energy of the system.
\end{exercise}

%\begin{exercise}
%Consider the ball inside the swinging hoop shown in figure \ref{Fig:BallOnSwingingHoop}. The hoop has a mass distribution of $\rho$ per unit length and a radius of $r$. Neglecting 
%friction between the ball and the hoop write down the constraint forces acting on the ball. Write down Euler's rigid body equations for the hoop and using this expression eliminate the 
%constraint forces appearing in the equations for the ball and find the equations that govern the motion of the entire system.
%\begin{figure}[ht]
%\begin{center}
%\includegraphics[width=2in]{BallOnSwingingHoop}
%\renewcommand{\baselinestretch}{1}\selectfont
%\caption{Ball on Swinging Hoop. Figure copied from \cite{Greenwood}.}
%\label{Fig:BallOnSwingingHoop}
%\renewcommand{\baselinestretch}{1.5}\selectfont
%\end{center}
%\end{figure}
%\end{exercise}

%%%%%%%%%%%%%%%%%%%%%%%%%%

%\begin{exercise}\label{ex:CentrifugalGovernor}
%Consider the centrifugal governor shown in figure \ref{Fig:Governor0}. Find the equations of motion of the system. Assume that the mass of the links can be neglected and that the bottom links are thin while the upper links are of non-negligible cross section.
%\begin{figure}[ht]
%\begin{center}
%\includegraphics[width=4in]{Governor}
%\renewcommand{\baselinestretch}{1}\selectfont
%\caption{Centrifugal Governor}
%\label{Fig:Governor0}
%\renewcommand{\baselinestretch}{1.5}\selectfont
%\end{center}
%\end{figure}
%\end{exercise}
%%%%%%%%%%%%%%%%

%%%%%%%%%%%%%%%%%%%
\begin{exercise}\label{ex:EulersTheorem}
{\sl Prove Euler's Theorem:} 
\begin{enumerate}
\item The eigenvalues of $R\in SO(3)$ are always of the form $\{1,e^{i\theta},e^{-i\theta}\}$.
\item Every $R\in SO(3)$ corresponds to a rotation about some axis by some angle.
\end{enumerate}
\end{exercise}
%%%%%%%%%%%%%%%%%

\begin{exercise}\label{ex:SkewSqrd}
Answer the following:
\begin{enumerate}
\item Consider the initial value problem
\[
\dot{R}=R\widehat{\Omega},\:\:\:\:\:\:\:\:\:R(0)=I_{3\times 3},
\]
where $\widehat{\Omega}$ is a constant. Show by direct verification that
\[
R(t)=\exp{(t\widehat{\Omega})}
\]
is the unique solution of the above matrix ODE initial value problem.
\item Using the above result show that for any $\widehat{\Omega}\in so(3)$ the matrix $R=\exp{(\widehat{\Omega})}$ corresponds to a rotation  about the axis $\Omega$ by an angle equal to $||\Omega||$.
\item Using the second property property proven in exercise-\ref{ex:PropertiesCrossProduct} show that
\[
\widehat{\Omega}^2w=(\Omega\cdot w)\Omega-||\Omega||^2w=(\Omega\Omega^T-||\Omega||^2I_{3\times 3})w
\]
for any $3\times 1$ matrix $w$
and hence that
\[
\widehat{\Omega}^3=-||\Omega||^2\widehat{\Omega},\:\:\:\:\:\:\:\widehat{\Omega}^4=-||\Omega||^2\widehat{\Omega}^2,\:\:\:\:\:\:\:\widehat{\Omega}^5=||\Omega||^4\widehat{\Omega},\:\:\:\:\:\:\:\cdots
\]
\item Using the results of the question above show that
\begin{align*}
\exp{\left({\widehat{\Omega}}\right)}&=I_{3\times 3}+\frac{\sin{||\Omega||}}{||\Omega||}\widehat{\Omega}+\frac{1}{2}\left(\frac{\sin{\frac{||\Omega||}{2}}}{{\frac{||\Omega||}{2}}}\right)^2\widehat{\Omega}^2.\label{eq:Rodrigues1}
\end{align*}
This equation is known as the \textit{Rodrigues formula}.
\item Let $E_1=[1\:\:\:0\:\:\:0]^T,E_2=[0\:\:\:1\:\:\:0]^T,E_3=[0\:\:\:0\:\:\:1]^T$. Using the Rodrigues formula explicitly show that
\[
\exp{(\theta \widehat{E}_i)}=R_i(\theta),
\]
where $R_i(\theta)$ is the rotation matrix that corresponds to a rotation about the $i^\mathrm{th}$-axis by an angle equal to $\theta$. Compare your results with what you did in exercise-\ref{ex:RotatedFrames}.
\end{enumerate}
\end{exercise}

%%%%%%%%%%%%%%%%%%%%%%%%%%%%%%%%



\begin{exercise}\label{ex:EveryR=Rotation}
Show that every $R\in \mathrm{SO}(3)$ can be expressed as $R=\exp{(\widehat{\Omega})}$ for some $\widehat{\Omega}\in so(3)$.
\end{exercise}




%%%%%%%%%%%%%%%%%%
%\begin{exercise}\label{ex:GyroscopicSystem}
%Consider the Gyroscope shown in figure \ref{Fig:GyroscopicSystemWithSpring}. Using the results of section \ref{Secn:ForcedGyro} derive the equations of motion.
%\begin{figure}[ht]
%\begin{center}
%\includegraphics[width=2.5in]{GyroscopicSystem}
%\renewcommand{\baselinestretch}{1}\selectfont
%\caption{The Gimbal Gyroscope}
%\label{Fig:GyroscopicSystemWithSpring}
%\renewcommand{\baselinestretch}{1.5}\selectfont
%\end{center}
%\end{figure}
%\end{exercise}

%\begin{exercise}\label{ex:3DOFRobotArm}
%Consider the 3-DOF robotic arm shown in figure-\ref{Fig:3DOF_RobotArm0}. 
%\begin{enumerate}
%\item Find the relationship between the joint rotations and the displacement and velocities of the end point of link-2.
%\item Find the equations of motion of the robot arm.
%\item Simulate the motion of the robot arm using MATLAB.
%\end{enumerate}
%\begin{figure}[ht]
%\begin{center}
%\includegraphics[width=2.6in]{TwoLink}
%\renewcommand{\baselinestretch}{1}\selectfont
%\caption{3-DOF Robot Arm}
%\label{Fig:3DOF_RobotArm0}
%\end{center}
%\end{figure}
%\end{exercise}

\begin{exercise}
For the single bar pendulum shown in figure \ref{Fig:SingleBarPendulum} derive the equations of motion using Euler's rigid body equations and also write down the 
kinetic energy of the system.
\begin{figure}[ht]
\begin{center}
\includegraphics[width=2in]{SingleBarPendulum}
\renewcommand{\baselinestretch}{1}\selectfont
\caption{Simple Pendulum}
\label{Fig:SingleBarPendulum}
\renewcommand{\baselinestretch}{1.5}\selectfont
\end{center}
\end{figure}
\end{exercise}

\begin{exercise}
For the double bar pendulum shown in figure \ref{Fig:DoublePendulum} write down the kinetic energy of the system.
\begin{figure}[ht]
\begin{center}
\includegraphics[width=1.5in]{DoublePendulum}
\renewcommand{\baselinestretch}{1}\selectfont
\caption{Double Pendulum}
\label{Fig:DoublePendulum}
\renewcommand{\baselinestretch}{1.5}\selectfont
\end{center}
\end{figure}
\end{exercise}

\begin{exercise}\label{ex:ConstrainedParticleMotion}
Two point masses $P_1$ and $P_2$ are moving in three-dimensional Euclidean space such that $P_1$ moves on a sphere and the distance between $P_1$ and $P_2$ remains 
fixed. Specify the configuration space and a suitable set of coordinates for the system. What is the DOF of the system.
\end{exercise}


\newpage
\chapter{Answers to Selected Exercises}

\subsection*{Answer to Exercise \ref{ex:ParticleOnSphere}}
In a Euclidean frame $e$, with no external force fields present, a particle $P$ is constrained to move on a sphere.

Since the particle is constrained to move on a sphere of radius $r$ we have that
\[r^2=x_1^2+x_2^2+x_3^2=||x||^2=x^Tx=constant,\]
where $x$ is the representation of the particle in the $e$ frame.
Differentiating the constraint $x^Tx=constant$ we have that
\[2x^T\dot{x}=0,\]
thus the velocity in the $e$ frame $\dot{x}$ is orthogonal to $x$ and hence lies tangent to the sphere.
Differentiating the constraint $x^T\dot{x}=0$ we have that
\[
x^T\ddot{x}+\dot{x}^T\dot{x}=0.
\]
Since the particle is constrained to move on a sphere its acceleration perpendicular to the sphere should be zero. Thus we see that the constraint force acting on the particle should cancel this acceleration if the particle is to move on the sphere.  If no other constraints exists then the constraint force comprises only this component and hence is normal to the surface. Thus 
the representation of the force in the $e$ frame is of the form
\[
f_c=\alpha x.
\]
Thus from Newton's equations in the $e$ frame
\[
\ddot{x}=\frac{1}{m}f_c=\frac{\alpha}{m}x.
\]
Hence we have that
\[
\frac{\alpha}{m}x^Tx+\dot{x}^T\dot{x}=0,
\]
and that
\[
\alpha=-m\frac{||\dot{x}||^2}{r^2},
\]
and the constraint force is $f_c=-m\frac{||\dot{x}||^2}{r^2}x$.
Now the motion of the particle is described by
\[
\ddot{x}=\frac{1}{m}f_c=-\frac{||\dot{x}||^2}{r^2}x.
\]

\begin{figure}[ht]
\begin{center}
\includegraphics[width=3.5in]{ParticleOnSphere_Frames}
\renewcommand{\baselinestretch}{1}\selectfont
\caption{}
\label{Fig:ParticleOnSphere}
\renewcommand{\baselinestretch}{1.5}\selectfont
\end{center}
\end{figure}
Consider a moving frame $b(t)$ such that the position of the particle appears fixed in $b(t)$ and $b(t)$ is related to $e$ as shown in 
figure-\ref{Fig:ParticleOnSphere}. Note that this relationship is ill-defined when $\phi=0$ of $\phi=\pi$. 

If $b(t)=eR(t)$ we have that $a=eR_3{(\theta)}$ 
and $b=aR_2{(\phi)}$. Thus
$b=eR=eR_3{(\theta)}R_2{(\phi)}$ and $R=R_3{(\theta)}R_2{(\phi)}$.


Recall that $\dot{R}=R\widehat{\Omega}$. Differentiating $x=RX$ we  have that
\[
\dot{x}=\dot{R}X+R\dot{X}=R\widehat{\Omega}X.
\]
where we have used the fact that $\dot{X}=0$ since $P$ appears to be fixed in $b(t)$. Thus
\[
||\dot{x}||=||R\widehat{\Omega}X||=||\widehat{\Omega}X||=||\Omega\times X||.
\]


Newton's Equations in the $b$ frame are
\[
m\ddot{X}=-m\widehat{\Omega}^2X-2m\widehat{\Omega}\dot{x}-m\dot{\widehat{\Omega}}X+R^Tf_c,
\]
where
$f_c=-m\frac{||\dot{x}||^2}{r^2}x$. Since $x=RX$ and $X$ is a constant we have
\[
0=-m\widehat{\Omega}^2X-m\dot{\widehat{\Omega}}X-m\frac{||\widehat{\Omega}X||^2}{r^2}X,
\]
and that
\[
\dot{\widehat{\Omega}}X=-\widehat{\Omega}^2X-\frac{||\widehat{\Omega}X||^2}{r^2}X.
\]


Now let us explicitly write down these equations. By construction of the frames $X=[0\:\: 0\:\: r]^T$. Using results of exercise \ref{ex:RotatedFrames} we see that $a=eR_3{(\theta)}$ 
and $b=aR_2{(\phi)}$. Thus
$b=eR=eR_3{(\theta)}R_2{(\phi)}$ and $R=R_3{(\theta)}R_2{(\phi)}$.
Differentiating we have
\[
\dot{R}=\dot{R}_3{(\theta)}R_2{(\phi)}+R_3{(\theta)}\dot{R}_2{(\phi)}.
\]
Now
\[
\widehat{\Omega}=R^T\dot{R}=R^T_2{(\phi)}R^T_3{(\theta)}\dot{R}_3{(\theta)}R_2{(\phi)}+R^T_2{(\phi)}R^T_3{(\theta)}R_3{(\theta)}\dot{R}_2{(\phi)}
=R^T_2{(\phi)}\widehat{\Omega}_3R_2{(\phi)}+\widehat{\Omega}_2.
\]
Evaluating this we have
\[
\widehat{\Omega}=\left[\begin{array}{ccc}
0 & -\dot{\theta}\cos{\phi} & \dot{\phi}\\
\dot{\theta}\cos{\phi} & 0 & \dot{\theta}\sin{\phi}\\
- \dot{\phi}& -\dot{\theta}\sin{\phi} & 0
\end{array}\right].
\]
Then
\[
\dot{\widehat{\Omega}}=\left[\begin{array}{ccc}
0 & -\ddot{\theta}\cos{\phi}+\dot{\theta}\dot{\phi}\sin{\phi} & \ddot{\phi}\\
\ddot{\theta}\cos{\phi}-\dot{\theta}\dot{\phi}\sin{\phi} & 0 & \ddot{\theta}\sin{\phi}+\dot{\theta}\dot{\phi}\cos{\phi}\\
- \ddot{\phi}& -\ddot{\theta}\sin{\phi}-\dot{\theta}\dot{\phi}\cos{\phi} & 0
\end{array}\right].
\]
and
\[
\widehat{\Omega}^2=\left[\begin{array}{ccc}
-(\dot{\phi}^2+\dot{\theta}^2\cos^2{\phi}) & -\dot{\theta}\dot{\phi}\sin{\phi} & -\dot{\theta}^2\cos{\phi}\sin{\phi}\\
 -\dot{\theta}\dot{\phi}\sin{\phi} & -\dot{\theta}^2 & \dot{\theta}\dot{\phi}\cos{\phi}\\
-\dot{\theta}^2\cos{\phi}\sin{\phi}&\dot{\theta}\dot{\phi}\cos{\phi} & -(\dot{\phi}^2+\dot{\theta}^2\sin^2{\phi})
\end{array}\right].
\]
Substituting these in the Newton's equations
\[
\dot{\widehat{\Omega}}X=-\widehat{\Omega}^2X-\frac{||\widehat{\Omega}X||^2}{r^2}X.
\]
we have
\[
r\left[\begin{array}{c}
 \ddot{\phi}\\
\ddot{\theta}\sin{\phi}+\dot{\theta}\dot{\phi}\cos{\phi}\\
0
\end{array}\right]=
r\left[\begin{array}{c}
\dot{\theta}^2\cos{\phi}\sin{\phi}\\
-\dot{\theta}\dot{\phi}\cos{\phi}\\
(\dot{\phi}^2+\dot{\theta}^2\sin^2{\phi})
\end{array}\right]-
(\dot{\phi}^2+\dot{\theta}^2\sin^2{\phi})\left[\begin{array}{c}
0\\
0\\
r
\end{array}\right].
\]
Which gives us
\begin{eqnarray*}
\ddot{\phi} &=& \dot{\theta}^2\cos{\phi}\sin{\phi},\\
\sin{\phi}\ddot{\theta} & = & -2\dot{\theta}\dot{\phi}\cos{\phi}.
\end{eqnarray*}
Note that these equations are not defined when $\phi=0,\pi$ since for these values the frame $b$ is ill defined.

\subsection*{Answer to Exercise \ref{ex:KinematicsNKineticsInElevator}}
Assume $e$ is inertial and let $f^e$ be the resultant fundamental force felt in a frame $e$.
Newton's equations in the inertial $e$ frame are
\[
\ddot{x}=\frac{1}{m}f^e.
\]
Then we know that the Newton's equations in a parallely translating frame are
\begin{align}
\ddot{X}=-m\ddot{o}+f^e=\frac{1}{m}F^b,\label{eq:NewtonInTranslatingFrames}
\end{align}
where $o$ is the co-ordinates of the origin of $b(t)$ with respect to $e$.
\\
\\
For the problem of the elevator let $e$ be the frame fixed on the Earth and let $b(t)$ be the frame fixed on the elevator. The weight measured by a scale is the total force exerted by 
the scale $W$. The total fundamental force acting on the person as a result of particle interactions as expressed in the $e$ frame is
\[
f^e=-mg+W.
\]
\\
When the elevator is at rest,
\[
\ddot{o}=0,
\]
and hence from Newton's equations in the moving frame (\ref{eq:NewtonInTranslatingFrames}) we have
\[
m\ddot{X}=f^e=-mg+W,
\]
The person is at rest with respect to the elevator and therefore $X=const$ and $\ddot{X}=0$ and hence
\[
0=-mg+W,
\]
and
\[
W=mg.
\]
Thus when the elevator is at rest the weight measured by the elevator is the correct weight of the person.\\
\\
When the elevator is moving at a constant acceleration of $g$,
\[
\ddot{o}=g,
\]
and hence from Newton's equations in the elevator given by (\ref{eq:NewtonInTranslatingFrames}) we have
\[
m\ddot{X}=-mg+f^e=-mg+(-mg+W)=W-2mg,
\]
The person is at rest with respect to the elevator and therefore $X=const$ and $\ddot{X}=0$ and hence
\[
0=W-2mg,
\]
and
\[
W=2mg.
\]
Thus when the elevator is accelerating constantly at $g$, the weight measured by the scale is twice as large as the correct weight of the person.\\
\\
When the elevator is moving at a constant acceleration of $g$ but gravity is absent,
Then
\[
f^e=W.
\]
and
\[
\ddot{o}=g,
\]
and from Newton's equations in the translating frame (\ref{eq:NewtonInTranslatingFrames}) we have
\[
m\ddot{X}=-mg+f^e=-mg+W.
\]
The person is at rest with respect to the elevator and therefore $X=const$ and $\ddot{X}=0$ and hence
\[
0=-mg+W,
\]
and
\[
W=mg.
\]
Thus when the elevator is moving at a constant acceleration of $g$ but gravity is absent the weight read by the scale is the same as the correct weight of the person.
%%%%%%%%%%%%%%%%%%%%%%%%%%%%%%%%
\subsection*{Answer to Exercise \ref{ex:RotationMatrix}}
Consider the representation of a point $P$ in space. To do this we rely on the space-time assumptions of Galelian mechanics that imply space to be Euclidean. Thus we may pick another point $O$ in space and setup an orthonormal frame $e$ at $O$. Let $x$ be the representation of $P$ in $e$. That is let $OP=ex$. The Euclidian assumption of space implies that the distance from $O$ to $P$ is
\[
d(O,P)=||x||=\sqrt{\langle\langle x,x\rangle\rangle}=\sqrt{x^Tx}.
\]
Let $R\in \mathrm{SO}(3)$ be a special orthogonal matrix and let $b$ be the orthonormal frame such that $b=eR$ and the origin of $b$ coincides with that of $e$. Now let $X$ be the representation of $P$ in the $b$ frame. Then we know that $x=RX$. Then the Euclidean assumption implies that the distance from $O$ to $P$ can also be expressed as
\[
d(O,P)=||X||=\sqrt{\langle\langle X,X\rangle\rangle}=\sqrt{X^TX}.
\]
Note that this is true for all points $P$ and hence for any $X$

Thus
\begin{align*}
X^TX=||X||^2=||x||^2=x^Tx=(RX)^TRX=X^T(R^TR)X
\end{align*}
and hence we have
\begin{align*}
X^T(I_{3\times 3}-R^TR)X=0.
\end{align*}
for any $X\in\mathbb{R}^3$. Thus we have that
\[
R^TR=I_{3\times 3}.
\]
This also says that $R^T$ is the unique inverse of $R$. Hence we also have 
\[
RR^T=I_{3\times 3}.
\]

Since
\[
\det{(I_{3\times3})}=\det{(R^TR)}=\det{(R^T)}\det{(R)}=\det{(R)}\det{(R)}=(\det{(R)})^2.
\]
Thus $(\det{(R)})^2=1$ and hence $\det{(R)}=\pm1$.
Consider a continuous curve in $\mathrm{SO}(3)$ given by $C(t)$ such that $C(0)=I_{3\times 3}$ and $C(1)=R$. That is a curve that begins at $I_{3\times 3}$ and ends at $R$.
Since $\det :SO(3)\to \mathbb{R}$ is a continuous map we know that $\det{(C(t))}$ must vary continuously. But we have seen that $\det{(C(t))}=\pm 1$. Hence since $\det{(C(0))}=\det{(I_{3\times 3}})=1$ we have that $\det{(C(1))}=\det{(R)}=1$. By considering frames we can show that any $R\in \mathrm{SO}(3)$ can be continuously deformed to the case that corresponds to no rotation. That is to $I_{3\times 3}$ and thus  $\det{(R)}=1$ for all $R\in \mathrm{SO}(3)$.

%%%%%%%%%%%%%%%%%%%%%%%%%%%%%%%%





\subsection*{Answer to Exercise \ref{Ex:RotatingDiskMassInSlot}}
{\it The following typed up solution is courtesy of Kanishke Gamagedara E/09/078}\\

Assume that the earth fixed frame $e$ is inertial.\\

\begin{figure}[hbtp]
  \begin{center}
  \includegraphics[scale=.6]{disk2.png}
  \caption{Used frames}
  \label{fig:disk2}
  \end{center}
\end{figure}

Considering the figure,
\begin{align}
\mathbf{c}&=\mathbf{b}R_1(\phi) \label{eq:c=bR1}\\
\mathbf{b}&=\mathbf{e}R_3(\theta) \label{eq:b=eR3}
\end{align}

where,
\[R_1(\phi)=
\begin{bmatrix}
1&0&0\\
0 & \cos\phi & -\sin\phi\\
0 & \sin\phi & \cos\phi
\end{bmatrix}, \;\;
R_3(\theta)=
\begin{bmatrix}
\cos\theta & -\sin\theta & 0\\
\sin\theta & \cos\theta & 0\\
0 & 0 & 1\\
\end{bmatrix}.
\]
Then we also have
\[\widehat{\Omega}_1=R_1^T\dot{R}_1=
\begin{bmatrix}
0&0&0\\
0 & 0 & -\dot{\phi}\\
0 & \dot{\phi} & 0
\end{bmatrix}, \;\;
\widehat{\Omega}_3=R_3^T\dot{R}_3=\begin{bmatrix}
0 & -\dot{\theta} & 0\\
\dot{\theta} & 0 & 0\\
0 & 0 & 0\\
\end{bmatrix}.
\]

From (\ref{eq:c=bR1}) and (\ref{eq:b=eR3}), we get
$\mathbf{c}=\mathbf{e}\underbrace{R_3R_1}_{R}=\mathbf{e}R$. 


The position of the small ball $x$ can be written as,
\begin{align*}
\mathbf{e}x &= \mathbf{c}X= \mathbf{e}RX
\end{align*}

Distance to the ball from the center of the disk ($OP$) can be written using the $c$ frame as follows,
\begin{align*}
OP &= 
\underbrace{\begin{bmatrix}
c_1 & c_2 & c_3
\end{bmatrix}}_\mathbf{c}
\underbrace{
\begin{bmatrix}
0\\y\\d
\end{bmatrix}}_X=\mathbf{c}X.
\end{align*}

The same position of the ball $P$ can be written using the $\mathbf{e}$ frame as
\begin{align*}
OP=\underbrace{\begin{bmatrix}
e_1 & e_2 & e_3
\end{bmatrix}}_\mathbf{e}
\underbrace{
\begin{bmatrix}
x_1\\x_2\\x_3
\end{bmatrix}}_x=\mathbf{e}x=\mathbf{c}X=\mathbf{e}RX
\end{align*}


The position representation of the point $P$ with respect to the $e$ frame given by $x$ and the representation with respect to the frame $c$ that is given by $X$ are related by,
\begin{align}
x=RX\label{eq:xAndX}
\end{align}

Differentiating (\ref{eq:xAndX}) twice with respect to time,
\begin{equation}
\ddot{x}=R\left(\widehat{\Omega}^2X+2\widehat{\Omega}\dot{X}+\dot{\widehat{\Omega}}X+\ddot{X} \right)
\end{equation}

where,
\[ \dot{X}=\begin{bmatrix}
0\\\dot{y}\\0
\end{bmatrix} \;, \;\;\;
\ddot{X}=\begin{bmatrix}
0\\\ddot{y}\\0
\end{bmatrix} \;,
\]\\

\[\widehat{\Omega}=R^T\dot{R}=(R_1^T\widehat{\Omega}_3R_1+\widehat{\Omega}_1)=\begin{bmatrix}
0& -\dot{\theta}\cos\phi & \dot{\theta}\sin\phi \\
\dot{\theta}\cos\phi & 0 & -\dot{\phi}\\
-\dot{\theta}\sin\phi &\dot{\phi} & 0
\end{bmatrix} \;\;,\]\\

\[\dot{\widehat{\Omega}}=\begin{bmatrix}
0 & -\ddot{\theta}\cos\phi+\dot{\theta}\dot{\phi}\sin\phi & \ddot{\theta}\sin\phi + \dot{\theta}\dot{\phi}\cos\phi \\
+\ddot{\theta}\cos\phi-\dot{\theta}\dot{\phi}\sin\phi & 0 & -\ddot{\phi}\\
-\ddot{\theta}\sin\phi - \dot{\theta}\dot{\phi}\cos\phi & \ddot{\phi} & 0
\end{bmatrix}
\]\\
\[
\widehat{\Omega}^2=\begin{bmatrix}-\dot{\theta}^2 &  \dot{\phi}\dot{\theta}\sin{\phi} & \dot{\phi}\dot{\theta}\cos{\phi}\\
\dot{\phi}\dot{\theta}\sin{\phi} & - (\dot{\theta}^2\cos^2{\phi} +\dot{\phi}^2) &       \dot{\theta}^2\sin{2\phi}/2\\
\dot{\phi}\dot{\theta}\cos{\phi} & \dot{\theta}^2\cos{2\phi}/2 & - (\dot{\theta}^2\sin^2{\phi} +\dot{\phi}^2)\end{bmatrix}
\]

Assuming the Earth fixed $\mathbf{e}$-frame is inertial we have $f=m\ddot{x}$ in the $\mathbf{e}$-frame,
\begin{equation}
f=mR\left(\widehat{\Omega}^2X+2\widehat{\Omega}\dot{X}+\dot{\widehat{\Omega}}X+\ddot{X} \right) \label{eq:f=ma}
\end{equation}

The fundamental forces on the mass are ,
\begin{align*}
\mathbf{e}f&=\mathbf{c}F+\mathbf{e}f_g=\mathbf{e}R\underbrace{\begin{bmatrix}
F_{N1}\\-ky\\F_{N3}
\end{bmatrix}}_{F}
+\mathbf{e}\underbrace{\begin{bmatrix}
0\\0\\-mg
\end{bmatrix}}_{f_g}
\end{align*}
$f=RF+f_g$ and hence $R^Tf=F+R^Tf_g$.
Substituting values in (\ref{eq:F=RT(f-M)}) and simplifying, we get,
\begin{equation}
\begin{bmatrix}
F_{N1}\\-ky\\F_{N3}
\end{bmatrix}-\begin{bmatrix}
0\\mg\sin{\phi}\\mg\cos{\phi}
\end{bmatrix}=
\begin{bmatrix}
m\left(\ddot{\theta}(d\sin\phi - y\cos\phi) - 2\dot{\theta}\dot{y}\cos\phi + 2\dot{\phi}\dot{\theta}(d\cos\phi + y\sin\phi)\right)\\
m\left(\ddot{y} - d\ddot{\phi}-y(\dot{\phi}^2 + \dot{\theta}^2\cos^2\phi) + d\dot{\theta}^2\cos\phi\sin\phi\right)\\
m\left(\ddot{\phi}y + 2\dot{\phi}\dot{y} - d(\dot{\phi}^2 +\dot{\theta}^2 \sin^2\phi) + y\dot{\theta}^2(\cos{2\phi})/2\right)
\end{bmatrix} \label{eq:F=}
\end{equation}

From the second row in (\ref{eq:F=}), we can solve this system. First and third rows can be used to find $F_{N1}$ and $F_{N2}$. Simplifying the second row,
\begin{equation}
\ddot{y}+y\left(\frac{k}{m}-\dot{\phi}^2-\dot{\theta}^2\cos^2\phi\right)+d\dot{\theta}^2\sin\phi\cos\phi-d\ddot{\phi}+g\sin\phi=0
\end{equation} 




%%%%%%%%%%%%%%%%%%%%%%%%%%%%%%%
\subsection*{Answer to Exercise \ref{ex:BeadOnHoop}}
{\it The following typed up solution is courtesy of Kanishke Gamagedara E/09/078}\\

In section-\ref{Secn:BeadOnRotatingHoop} we showed that the constraint forces a bead and the 
equations of motion of the bead when it is constrained to move in a rotating hoop is given by
\begin{equation}
\begin{bmatrix}
F_{N1}\\ F_{N2}\\ 0
\end{bmatrix}
=m\begin{bmatrix}
  -r\ddot{\theta}\cos{\phi} + 2r\dot{\phi}\dot{\theta}\sin{\phi}\\
 -r\dot{\phi}^2 - r\dot{\theta}^2\cos{\phi}^2 + g\sin{\phi} \\
    r\dot{\theta}^2\sin\phi\cos\phi + r\ddot{\phi} + g\cos{\phi}
\end{bmatrix},
\label{eq:F_values0}
\end{equation}
where the first two rows can be used to find the forces $F_{N1}$ and $F_{N2}$ and the third row can be used to describe the motion of the bead as follows:
\begin{equation}
r\ddot{\phi}+\cos{\phi}\left(g+\dot{\theta}^2\sin\phi\right)=0.
\label{eq:charac_eqn0}
\end{equation}
The coordinates $\theta, \phi$ were chosen as shown in Figure \ref{fig:hoop_top} and \ref{fig:hoop_front}. 

\begin{figure}[h]
\begin{center}
\includegraphics[width=2in]{BallOnHoop}
\renewcommand{\baselinestretch}{1}\selectfont
\caption{Bead on a Hoop.}
\label{Fig:BeadOnHoop}
\renewcommand{\baselinestretch}{1.5}\selectfont
\end{center}
\end{figure}


\begin{figure}[hbtp]
\minipage{0.5\textwidth}
  \begin{center}
  \includegraphics[scale=.5]{hoop_top.png}
  \caption{Top view}
  \label{fig:hoop_top}
  \end{center}
\endminipage\hfill
\minipage{0.5\textwidth}
  \begin{center}
  \includegraphics[scale=.4]{hoop_front.png}
  \caption{Perpendicular to Hoop}
  \label{fig:hoop_front}
  \end{center}
\endminipage\hfill
\end{figure}
Note that we see from equation (\ref{eq:charac_eqn0}) that the bead dynamics are influenced by the hoop dynamics due to the presence of the $\dot{\theta}^2$ term present in (\ref{eq:charac_eqn0}). 
Below we will use Euler's rigid body equations to write down equations that describe the hoop dynamics.
The moments acting on the hoop are
\[\mathbf{c}T=\mathbf{c}
\begin{bmatrix}
T_1\\T_2\\T_3
\end{bmatrix}
=\mathbf{e}
\begin{bmatrix}
T_e^1\\T_e^2\\0
\end{bmatrix}
+\mathbf{c}
\begin{bmatrix}
0\\-F_{N1}r\sin{\phi}\\F_{N1}r\cos{\phi}
\end{bmatrix}
=\mathbf{c}\left[
R_3^T(\theta)\begin{bmatrix}
T_e^1\\T_e^2\\0
\end{bmatrix}
+\begin{bmatrix}
0\\-F_{N1}r\sin{\phi}\\F_{N1}r\cos{\phi}
\end{bmatrix}\right].
\]
From this we have
\[
T=
R_3^T(\theta)\begin{bmatrix}
T_e^1\\T_e^2\\0
\end{bmatrix}
+\begin{bmatrix}
0\\-F_{N1}r\sin{\phi}\\F_{N1}r\cos{\phi}
\end{bmatrix}.
\]
Angular velocity matrix of the hoop is given by $\widehat{\Omega}_h=R_3^T\dot{R}_3$ and hence we see that
\[ \Omega_h=
\begin{bmatrix}
0\\0\\ \dot{\theta}
\end{bmatrix}
\]
Substituting these in Euler's rigid body equations in the hoop fixed frame $\mathbf{c}$,
$\mathbb{I}\dot{\Omega}=\mathbb{I}\Omega\times\Omega+T$
where $\mathbb{I} = \mathrm{diag}(\mathbb{I}_x,\mathbb{I}_y,\mathbb{I}_z)$
we get,
\begin{equation}
\begin{bmatrix}
0\\0\\\mathbb{I}_z\ddot{\theta}
\end{bmatrix}
=
\begin{bmatrix}
0\\0\\0
\end{bmatrix}
+R_3(\theta)^T
\begin{bmatrix}
T_e^1\\T_e^2\\0
\end{bmatrix}
+
\begin{bmatrix}
0\\-F_{N1}r\sin{\phi}\\F_{N1}r\cos{\phi}
\end{bmatrix}
\label{eq:IOmegadot=()}
\end{equation}

From these equations we find
\[ \mathbb{I}_z\ddot{\theta}=F_{N1}r\cos{\phi} \]
and that from (\ref{eq:F_values0})
\begin{equation}
({\mathbb{I}_z+mr^2\cos^2{\phi}})\ddot{\theta}-{mr^2\dot{\theta}\dot{\phi}\sin(2\phi)}=0,
\label{eq:thetaddot0}
\end{equation}
and hence that the two coupled second order differential equations (\ref{eq:charac_eqn0}) and (\ref{eq:thetaddot0}) describes the dynamics of the composite bead and hoop system.


The following MATLAB file\footnote{This simulation is courtesy of K. G. B. Gamagedara (E/09/078).} numerically integrates (\ref{eq:charac_eqn0}) and (\ref{eq:thetaddot0}) for the parameters $\mathbb{I}_z=1,m=1,r=1, \rho=1, d=1, g=1$ with suitable initial conditions.


\section*{MATLAB Code}
\begin{verbatim}
function Hoop_simulation3D
m=1;
r=1;
d=1;
rho=1;
g=1;
M=pi^2*d^2*r*rho/2;
I3=M*r^2/2;
X0=[0 pi/2*3 0 0];

[T,Y] = ode45(@dydt,0:.1:10,X0);


%%%%%%%%%%%%%%%%%%%%%%%%%%%%%%%%%%%%%

function Y = dydt(T,X)
    x1=X(1);    %theta
    x2=X(2);    %theta_dot
    x3=X(3);    %phi
    x4=X(4);    %phi_dot
    
    x1dot=x2;
    x2dot=m*r^2*x2*x4*sin(2*x3)/(I3+m*r^2*cos(x3)^2);
    x3dot= x4;
    x4dot=-g*cos(x3)/r-sin(2*x3)*x2^2/2;
    
    
    Y=[x1dot; x2dot; x3dot; x4dot];
end

%%%%%%%%%%%%%%%%%%%%%%%%%%%%%%%%%%%%%

figure(1)
plot(T,Y(:,1),'r')
grid on
hold on
plot(T,Y(:,2),'g')
plot(T,Y(:,3),'k')
plot(T,Y(:,4))
hold off

%%%%%%%%%%%%%%%%%%%%%%%%%%%%%%%%%%%%%%%%
figure(2)
j=1;
while j<=length(T)
    theta=Y(j,1);
    phi=Y(j,3);
    subplot(1,2,1)
    plot3(r*cos(phi)*cos(theta),r*cos(phi)*sin(theta),...
    r*sin(phi),'Marker','square','MarkerSize',...
    2,'MarkerEdgeColor','b','MarkerFaceColor','b')
    grid on
    hold on
    draw_hoop(theta,360)
    axis(.5*[-10 10 -10 10 -10 10])
    hold off
    
    
    subplot(1,2,2)
    plot3(r*cos(phi)*cos(theta),r*cos(phi)*sin(theta),...
    	r*sin(phi),'Marker','square','MarkerSize',...	
    	2,'MarkerEdgeColor','b','MarkerFaceColor','b')
    axis(.5*[-10 10 -10 10 -10 10])
    grid on
    hold on
    view([0 0 1])
    draw_hoop(theta,360)
    hold off
    
    pause(.1)
   
    
    j=j+1;
end

    function draw_hoop(theta,reso)
        P=0:2*pi/reso:2*pi;
        k=1;
        x_h=[];
        y_h=[];
        z_h=[];
        while k<=length(P)
            p=P(k);
            x_h(k)=r*cos(p)*cos(theta);
            y_h(k)=r*cos(p)*sin(theta);
            z_h(k)=r*sin(p);
            k=k+1;
        end
        plot3(x_h,y_h,z_h,'r')
    end

end
\end{verbatim}

%%%%%%%%%%%%%%%%%%%%%%%%%

\subsection*{Answer to Exercise \ref{ex:InstantaneousCenter}}


Consider the frames $\mathbf{e}$ and $\mathbf{b}$ fixed on $\mathcal{B}_1$ and $\mathcal{B}_2$ 
respectively. $O_1$ is the origin of $e$ and $O_2$ is the origin of $b$. Now let us consider points on $\mathcal{B}_2$ as viewed from $\mathcal{B}_1$.
A point $P$ on $\mathcal{B}_2$ will have representations $x$ and $X$ with respect to $e$ and $b$ respectively. Let $O_1O_2=\mathbf{e}\,y$ and $\mathbf{b}=\mathbf{e}\,R_{\phi}$ then
\begin{eqnarray}
x &=& y+R_{\phi}X,\label{eq:RelPosition}\\
\dot{x} & = & \dot{y}+R_{\phi}\widehat{\dot{\phi}}X.\label{eq:RelPosition}
\end{eqnarray}
If there exists a point $O_{21}P$ on $\mathcal{B}_2$ that appears to be fixed in $\mathcal{B}_1$ then there exists $X_I$ such that $\dot{x}_I=0$. If we can show the existence of such 
a $X_I$ then we have shown the existence of an instantaneous center.
If $\dot{x}_I=0$, from (\ref{eq:RelPosition}) it follows that
\begin{eqnarray}
\widehat{\dot{\phi}}X_I &=& -R^T_{\phi}\;\dot{y},\\
X_I &=& \frac{1}{\dot{\phi}^2}\widehat{\dot{\phi}}R_{\phi}^T\;\dot{y}. \label{eq:InstCenter}
\end{eqnarray}
That is the point in $\mathcal{B}_2$ with representation $X_I$ with respect to $b$ and given by (\ref{eq:InstCenter}) has zero velocity with respect to $\mathcal{B}_1$. The point 
$O_{21}$ is said to be the instantaneous center of $\mathcal{B}_2$ with respect to $\mathcal{B}_1$. With respect to $\mathcal{B}_1$, every point on  $\mathcal{B}_2$ appears to 
instantaneously rotate about $P$ with angular velocity $\dot{\phi}$. You are asked to show this in the exercises.
In addition if the bodies $\mathcal{B}_1$ and $\mathcal{B}_2$ are two convex shaped rigid bodies that are moving relative to each other such that at each time instance their 
surfaces are in contact only at one point $P$ then  from the assumption that the bodies are rigid it follows that the relative velocity of the contact point should lie along the surface (that 
is perpendicular to the common surface normal through the contact point) . Thus the instantaneous center of rotation of $\mathcal{B}_2$ with respect to $\mathcal{B}_1$ (or visa 
versa) lies along the common normal to the surfaces at the contact point.

%%%%%%%%%%%%%%%%%


\subsection*{Answer to Exercise \ref{ex:KennedysTheorem}}
\begin{figure}[ht]
\begin{center}
\includegraphics[width=4in]{Kennedy}
\renewcommand{\baselinestretch}{1}\selectfont
\caption{Kennedy's Theorem}
\label{Fig:KennedysTheorem}
\renewcommand{\baselinestretch}{1.5}\selectfont
\end{center}
\end{figure}

To prove Kennedy's theorem consider three rigid bodies $\mathcal{B}_1$, $\mathcal{B}_2$ and $\mathcal{B}_3$ moving in 2-dimensional Euclidean space. Let $\mathbf{e}$, $\mathbf{b}$ and $\mathbf{c}$ be 
frames fixed on $\mathcal{B}_1$, $\mathcal{B}_2$ and $\mathcal{B}_3$ respectively. At a given particular time instant, without loss of generality, we pick the origins of $\mathbf{e}$ and $\mathbf{b}$ 
to coincide with the instantaneous center $O_{21}$ of $\mathcal{B}_2$ with respect to $\mathcal{B}_1$ and the origin $c$ to coincide with the instantaneous center $O_{31}$ of $
\mathcal{B}_3$ with respect to $\mathcal{B}_1$. What we need to show is that the instantaneous center $O_{32}$ of $\mathcal{B}_3$ with respect to $\mathcal{B}_2$ lies on the line 
joining $O_{21}$ to $O_{31}$. To do this we observe points on $\mathcal{B}_3$ with reference to $\mathcal{B}_2$. Let $P$ be a point on $\mathcal{B}_3$. The representation of $P$ 
with respect to $\mathbf{e}$ is $x$, with respect to $\mathbf{b}$ is $X_b$ and with respect to $\mathbf{c}$ is $X_c$ and let $\mathbf{b}=\mathbf{e}\,R_{\theta}$, $\mathbf{c}=\mathbf{e}\,R_{\phi}$. Let $O_{21}O_{31}=\mathbf{e}\,y$. Then
\begin{equation}\label{eq:InstCenter1}
x=R_{\theta}X_b=y+R_{\phi}X_c.
\end{equation}
Now differentiating (\ref{eq:InstCenter1}) we have
\begin{equation}\label{eq:InstCenter2}
\dot{x} =R_{\theta}\dot{X}_b+R_{\theta}\widehat{\dot{\theta}}X_b=\dot{y}+R_{\phi}\dot{X}_c+R_{\phi}\widehat{\dot{\phi}}X_c.
\end{equation}
\\
Let $P=O_{23}$. The instantaneous velocity of $O_{32}$ with respect to $\mathcal{B}_2$ (and hence the $\mathbf{b}$ frame) and $\mathcal{B}_3$
(and hence the $c$ frame) is zero. Thus $\dot{X}_b=0=\dot{X}_c$.
Since $O_{31}$ is the instantaneous center of $\mathcal{B}_3$ with respect to $\mathcal{B}_1$ the point $O_{31}$ has zero instantaneous velocity with respect to $\mathcal{B}_1$. 
Thus $\dot{y}=0$.  Substituting these in (\ref{eq:InstCenter2}) we have
\begin{equation}\label{eq:InstCenter3}
\dot{x} =R_{\theta}\widehat{\dot{\theta}}X_b=R_{\phi}\widehat{\dot{\phi}}X_c.
\end{equation}
 Using (\ref{eq:InstCenter1}) we then have
\begin{equation}\label{eq:InstCenter2b}
\widehat{\dot{\theta}}x=\widehat{\dot{\phi}}(x-y).
\end{equation}
and finally that
\begin{equation}\label{eq:InstCenter3}
x = \frac{\dot{\phi}}{(\dot{\phi}-\dot{\theta})}\:y.
\end{equation}
From (\ref{eq:InstCenter3}) we see that $O_{32}$ lies on the line joining the instantaneous centers $O_{21}$ and $O_{31}$ and we have proved Kennedy's theorem that the 
instantaneous centers of three rigid bodies moving in 2-dimensional Euclidean space lie on the same line. Furthermore (\ref{eq:InstCenter3}) gives the ratio of the distances to the 
instantaneous centers as a function of the relative angular rotations of the two bodies. Explicitly
\begin{equation}\label{eq:InstCenterRatios0}
O_{21}O_{23} = \frac{\dot{\phi}}{(\dot{\phi}-\dot{\theta})}\:O_{21}O_{31}.
\end{equation}
By substituting $O_{21}O_{31}=O_{21}O_{23}+O_{23}O_{31}$ in this we also have
\begin{equation}\label{eq:InstCenterRatios}
O_{21}O_{23}\dot{\theta} + O_{31}O_{23}\dot{\phi}=0.
\end{equation}



%%%%%%%%%%%%%%%%

%%%%%%%%%%%%%%%%%%%%%%%
\subsection*{Answer to Exercise \ref{ex:FreeForcedVibrationP}}

\begin{figure}[ht]
\begin{center}
\includegraphics[width=4.5in]{ForcedVibrationDeflectionFrames}
\caption{A schematic representation of a free vibration apparatus.} \label{Fig:ForcedV}
\end{center}
        \[\footnote{}
\]
\end{figure}
Consider figure \ref{Fig:ForcedV}. Assume that when $z\equiv 0$ and $\theta\equiv0$ the system is in equilibrium.  Let the deflection of the spring at this condition be $\Delta$.
By applying Euler's 3D rigid body equations,
\begin{equation}
\dot{R}=R\widehat{\Omega}
\end{equation}
\begin{equation}\label{eq:Euler}
\mathbb{I}\dot{\Omega}=\mathbb{I}\Omega\times\Omega+T.
\end{equation}
Here
\[
R= \left[
\begin{array}{ccc}

\cos{\theta}  & -\sin{\theta} & 0\\
\sin{\theta} & \cos{\theta} & 0\\
0 & 0 & 1
\end{array}
\right],
\]
and hence
\[
\widehat{\Omega}= \left[
\begin{array}{ccc}
0  & -\dot{\theta} & 0\\
\dot{\theta} & 0 & 0\\
0 & 0 & 0
\end{array}
\right],\:\:\:\:\:\:{\Omega}= \left[
\begin{array}{c}
0  \\
0\\
\dot{\theta}
\end{array}
\right].
\]
The inertia tensor of the beam is
\[
\mathbb{I}= \left[
\begin{array}{ccc}
\mathbb{I}_1  &0 & 0\\
0 & \mathbb{I}_2 & 0\\
0 & 0 & \mathbb{I}_3
\end{array}
\right],
\]
where $\mathbb{I}_1,\mathbb{I}_2,\mathbb{I}_3$ are the inertia components of the body about the $b_1,b_2,b_3$ axis respectively.
Thus it is easily seen that $\mathbb{I}\Omega \times \Omega=0$ and hence (\ref{eq:Euler}) reduces to:
\begin{equation}
\mathbb{I}\dot{\Omega}=T,
\end{equation}
The external Torque$T=T_c+T_k+T_g$,
where $T_c$ and $T_k$ are the force moments (torques) expressed in the $b$ frame due to the damper force and spring force respectively and $T_g$ is the force moment due to the 
gravitational forces.
\\
\\
For small angles $\sin{\theta}\approx \theta\approx 0$ and $\cos{\theta}\approx 1$. For small deflections the above small angle approximations are valid and in the $b$ frame the 
spring force has the representation,
\begin{equation}
F_k=\left[\begin{array}{c}
0 \\
-K(L_k\theta+\Delta+z)\\
0\end{array}\right]
\end{equation}
The point at which this force acts has the representation in the $b$ frame given by
\[
X_k=\left[\begin{array}{cc}
L_k\\
0\\
0\end{array}\right]\\
\]
Thus the torque in the $b$ frame due to the spring force is
\begin{equation}
T_k=X_k\times F_k=\left[\begin{array}{c}
0\\
0\\
-K (L_k\theta+\Delta+z)L_k\end{array}\right]
\end{equation}
In the $b$ frame the damper force has the representation,
\begin{equation}
F_c=\left[\begin{array}{c}
0\\
-CL_c\,\dot{\theta}\\
0\end{array}\right]
\end{equation}
The representation in the $b$ frame of the point of action of the damper force is
\[
X_c=\left[\begin{array}{c}
L_c\\
0\\
0\end{array}\right].
\]
The the force moment due to the damper force is
\begin{equation}
T_c=X_c\times F_c
=\left[\begin{array}{c}
0\\
0\\
-CL_c^2\dot{\theta}\end{array}\right].
\end{equation}


In the $b$ frame the gravitational force force has the representation,
\begin{equation}
F_g=\left[\begin{array}{c}
0\\
Mg\\
0\end{array}\right]\,
\end{equation}
where $M$ is the total mass of the beam and $L$ is the distance from $O$ to the center of mass of the beam.
The representation in the $b$ frame of the point of action of the damper force is
\[
X_g=\left[\begin{array}{c}
L\\
0\\
0\end{array}\right].
\]
The the force moment due to the damper force is
\begin{equation}
T_g=X_g\times F_g
=\left[\begin{array}{c}
0\\
0\\
MgL\end{array}\right].
\end{equation}








Also
\[
I\dot{\Omega}=\left[\begin{array}{c}
0\\
0\\
I_3\ddot{\theta}\end{array}\right].
\]

Thus Euler's rigid body equations (\ref{eq:Euler}) gives,\\
\begin{equation}
I_3\ddot{\theta}=-CL_c^2\dot{\theta}-KL_k^2\theta+(MgL-KL_k\Delta)-KL_kz.
\end{equation}
Applying rigid body equations at equilibrium conditions we have
$(MgL-KL_k\Delta)=0$ and hence for small deflections
we have the governing differential equation given below.
\begin{equation}
I_3\ddot{\theta}+CL_c^2\dot{\theta}+KL_k^2\theta =-KL_k\,z.
\end{equation}
%%%%%%%%%%%%%%%%%%%%%%%%%%%%%%
\subsection*{Answer to Exercise \ref{ex:RotatingPend}}
Fix a frame $b(t)$ on the shaft and the arm such that its origin, $O$, is at the point of intersection of the arm and the shaft with $b_1(t)$ aligned along the arm, $b_3(t)$ aligned along 
the shaft and pointing upwards and $b_2(t)$ aligned so that the frame is right hand oriented. Let $e$ be a fixed frame such that the origin coincides with $O$ and $e_3$ is aligned 
along $b_3$. Let $\phi$ be the counter clockwise angle between $e_1$ and $b_1(t)$.
Then $b(t)=e\,R(t)$ where
\[
R_3(t)=\left[
\begin{array}{ccc}
\cos{\phi} & -\sin{\phi} & 0 \\
\sin{\phi} & \cos{\phi} & 0\\
0 & 0 & 1
\end{array}
\right]
\]
and $\dot{R}_3=R_3\,\widehat{\Omega}_3$
where
\[
\widehat{\Omega}_3=\left[
\begin{array}{ccc}
0 & -\dot{\phi} & 0 \\
\dot{\phi} & 0 & 0\\
0 & 0 & 0
\end{array}
\right]
\]


Let  $\theta$ be the  counter clockwise angle between the spring and the downward vertical and $l(t)$ be the 
length of the spring. Introduce another frame $c(t)$ such that its origin also coincides with the origin of $e$ and $b(t)$ while $c_1(t)=b_1(t)$ and $c_3(t)$ is along the spring. Then we see that $c(t)=b(t)R_1(t)$ where
\[
R_1(t)=\left[
\begin{array}{ccc}
1 & 0 & 0\\
0 & \cos{(\pi+\theta)} & -\sin{(\pi+\theta)} \\
0 & \sin{(\pi+\theta)} & \cos{(\pi+\theta)} 
\end{array}
\right].
\]
Then we also have that
$\dot{R}_1=R_1\,\widehat{\Omega}_1$
where
\[
\widehat{\Omega}_1=\left[
\begin{array}{ccc}
0 & 0 & 0 \\
0 & 0 & -\dot{\theta}\\
0 & \dot{\theta} & 0
\end{array}
\right].
\]
Then we have $c(t)=b(t)R_1(t)=eR_3(t)R_1(t)=eR(t)$ and hence that $R(t)=R_3(t)R_1(t)$ and 
$\dot{R}=\dot{R}_3R_1+R_3\dot{R}_1=R_3\widehat{\Omega}_3R_1+R_3R_1\widehat{\Omega}_1=R_3R_1(R_1^T\widehat{\Omega}_3R_1+\widehat{\Omega}_1)=R\widehat{\Omega}$. Thus
\[
\widehat{\Omega}=R_1^T\widehat{\Omega}_3R_1+\widehat{\Omega}_1=\begin{bmatrix}0 & \dot{\phi}\cos{\theta} & -\dot{\phi}\sin{\theta}\\
-\dot{\phi}\cos{\theta} & 0 & -\dot{\theta}\\
\dot{\phi}\sin{\theta} & \dot{\theta} & 0\end{bmatrix}
\]

\[
\widehat{\Omega}^2=\begin{bmatrix}-\dot{\phi}^2 & -\dot{\phi}\dot{\theta}\sin{\theta} & -\dot{\phi}\dot{\theta}\cos{\theta}\\
-\dot{\phi}\dot{\theta}\sin{\theta} & -(\dot{\phi}^2\cos^2\theta+\dot{\theta}^2) & \dot{\phi}^2\sin{2\theta}/2\\
-\dot{\phi}\dot{\theta}\cos{\theta} & \dot{\phi}^2\sin{2\theta}/2 & -(\dot{\phi}^2\sin^2\theta+\dot{\theta}^2)\end{bmatrix}
\]
\[
\dot{\widehat{\Omega}}=\begin{bmatrix}0 & \ddot{\phi}\cos{\theta}-\dot{\phi}\dot{\theta}\sin{\theta} & -\ddot{\phi}\sin{\theta}-\dot{\phi}\dot{\theta}\cos{\theta}\\
-\ddot{\phi}\cos{\theta}+\dot{\phi}\dot{\theta}\sin{\theta} & 0 & -\ddot{\theta}\\
\ddot{\phi}\sin{\theta}+\dot{\phi}\dot{\theta}\cos{\theta} & \ddot{\theta} & 0\end{bmatrix}
\]


Let us consider the 
point mass.
The representation of the point mass $P$ in $e$ is $x(t)$ and in $c(t)$ is $X(t)$. Then $x(t)=R(t)\, X(t)$ and it easily seen that
\[
X(t)=\left[ \begin{array}{c}
d \\ 0 \\ l \end{array} \right].
\]
Differentiating we have
\[
\dot{X}(t)=\left[ \begin{array}{c}
0 \\
0 \\
\dot{l} \end{array} \right]\,\:\:\:\:\:
\ddot{X}(t)=\left[ \begin{array}{c}
0 \\
0 \\
\ddot{l} \end{array} \right].
\]
Recall that Newton's equations for the particle expressed using the $c$-frame quantities are
\begin{equation}\label{eq:NewtonsEqn}
F=R^T f=m\, \ddot{x}=m(\widehat{\Omega}^2X+2\widehat{\Omega}\dot{X}+\dot{\widehat{\Omega}}X+\ddot{X}),
\end{equation}
where the matrix $F$ is the $b(t)$-frame representation of the resultant of the fundamental forces acting on the particle. We find that it is
\[
F=\left[ \begin{array}{c}
N \\ mg\sin{\theta} \\ -k(l-l_0)+mg\cos{\theta}  \end{array} \right].
\]
where $N$ is the constraint force on the particle in the $c_1$ direction and constraints the particle to only move in the plane perpendicular to the shaft. Now simplifying the Newton's 
equations (\ref{eq:NewtonsEqn}) we have
\begin{eqnarray}
-d\dot{\phi}^2-2l\dot{\phi}\dot{\theta}\cos{\theta}-l\ddot{\phi}\sin{\theta}&=& N,\\
l\ddot{\theta} + 2 \dot{l}\dot{\theta}-\dot{\phi}^2\sin{\theta}\cos{\theta}\;l+ d\ddot{\phi} \cos{\theta} & = & -g \sin{\theta}.\label{eq:Govern2}\\
\ddot{l} + \frac{k}{m}(l-l_0)-(\dot{\phi}^2\sin^2{\theta}+\dot{\theta}^2) l+d\ddot{\phi} \sin{\theta}& = & g \cos{\theta}, \label{eq:Govern1}
\end{eqnarray}


Now lets consider the motion of the rigid body. The body rotates about the $e_3$ axis and  Euler's rigid body equations give
\begin{equation}\label{eq:RigidBody0}
\mathbb{I}_z\ddot{\phi}=T_3,
\end{equation}
where $T_3$ is the force moment about the $b_3=e_3$ axis. The force acting at the tip of the arm in the $e$-frame is
\[
=[b_1\:\:\:b_2\:\:\:b_3]F^b = \left[\begin{array}{ccc} b_1 & b_2 & b_3 \end{array} \right] \left[\begin{array}{c} N \\ k(l-l_0)\sin{\theta} \\ -k(l-l_0)\cos{\theta} \end{array}\right]
\]
The representation of the end point, A,  of the arm is
\[
OA=[b_1\:\:\:b_2\:\:\:b_3]X_A=\left[\begin{array}{ccc} b_1 & b_2 & b_3\end{array} \right] \left[\begin{array}{c} d \\ 0 \\0\end{array}\right],
\]
Then the force moment about the $e_3$ axis is

\[
\mbox{force moment}=\tilde{X}_A F^b= \left[\begin{array}{ccc} 0 & 0 & 0\\0 & 0 & -d\\0 & d & 0 \end{array} \right] \left[\begin{array}{c} N \\ k(l-l_0)\sin{\theta}\\ -k(l-l_0)\cos{\theta} \end{array}\right]=\left[\begin{array}{c}0 \\ kd(l-l_0)\cos{\theta}\\kd(l-l_0)\sin{\theta}\end{array}\right].
\]
Thus $T_3=kd(l-l_0)\sin{\theta}$ and the rigid body equations (\ref{eq:RigidBody}) are
\begin{equation}\label{eq:RigidBody}
\mathbb{I}_z\ddot{\phi}=kd(l-l_0)\sin{\theta}.
\end{equation}


Thus the complete governing equations for the system are (\ref{eq:Govern1}),(\ref{eq:Govern2}) and (\ref{eq:RigidBody}). These are three coupled second order equations of the 
three configuration variables.


The kinetic energy of the system is
\[
KE=KE_{rigid\:\:body}+KE_{particle}.
\]
where the kinetic energy of the rigid body is given by
\[
KE_{rigid\:\:body}=\frac{\mathbb{I}_z}{2}\dot{\phi}^2,
\]
and the kinetic energy of the particle is given by
\[
KE_{particle}=\frac{m}{2}||\dot{x}||^2=\frac{m}{2}||R(\widehat{\Omega}X+\dot{X})||^2=\frac{m}{2}||\widehat{\Omega}X+\dot{X}||^2.
\]
Upon simplification we have that
\begin{eqnarray*}
KE_{particle} & = &\frac{m}{2}\left( -X^T\widehat{\Omega}^2X+2\dot{X}\widehat{\Omega}X+\dot{X}^T\dot{X}\right)\\
& = & \frac{(\mathbb{I}_z+md^2+ml^2\sin^2{\theta})}{2}\dot{\phi}^2+\frac{m}{2}\dot{l}^2+\frac{ml^2}{2}\dot{\theta}^2 +2d \sin{\theta}\dot{\phi}\dot{l}+2d \cos{\theta}l\dot{\phi}\dot{\theta}
\end{eqnarray*}
This is a quadratic form in the velocities.
%%%%%%%%%%%%%%%%%%%%%%%%%%
\subsection*{Answer to Exercise \ref{ex:CentrifugalGovernor}}

\begin{figure}[hbtp]
  \begin{center}
  \includegraphics[scale=.7]{Governer2FramesV2017.png}
  \caption{The Centrifugal Governor}
  \label{fig:governer}
  \end{center}
\end{figure}
We will fist consider a one link of the governer.
Consider a frame $\mathbf{a}$ that is fixed to the body of the governor such that $\mathbf{a}_3$ aligns along the axis of rotation of the governor and $\mathbf{a}_2$ is perpendicular to the plane containing the links of the governor as shown in figure-\ref{fig:governer}.  Let $\mathbf{b}$ be a frame fixed to the body of the governor such that $\mathbf{b}_2=\mathbf{a}_2$ and $\mathbf{b}_1$ is along one of the links of the governor as shown in figure-\ref{fig:governer}. Let $\mathbf{e}$ be an earth fixed frame such that $\mathbf{e}_3=\mathbf{a}_3$ and $\mathbf{a}=\mathbf{e}R_3(\theta)$ where
\[
  R_3(\theta)= 
  \left[\begin{matrix}
  \cos\theta & -\sin\theta & 0 \\
  \sin\theta & \cos\theta & 0 \\
  0 & 0 & 1
 \end{matrix}\right]
\]\\




We also see that $\mathbf{b}=\mathbf{a}R_2(\alpha)$ where
\[
  R_2(\alpha)= 
  \left[\begin{matrix}
  \cos\alpha & 0 & \sin\alpha \\
  0 & 1 & 0 \\
  -\sin\alpha & 0 & \cos\alpha\\
  \end{matrix}\right]
\]
Then we see that since $\mathbf{b}=\mathbf{a}R_2=\mathbf{e}R_3R_2=\mathbf{e}R$ that $R=R_3(\theta)R_2(\alpha)$.
Thus
\[
\widehat{\Omega}=R^T\dot{R}=(R_2^T\widehat{\Omega}_3R_2+\widehat{\Omega}_2)=\begin{bmatrix}
0 &-\dot{\theta}\cos\alpha & \dot{\alpha}\\
\dot{\theta}\cos\alpha & 0 & \dot{\theta}\sin\alpha \\
-\dot{\alpha} & -\dot{\theta}\sin\alpha & 0
\end{bmatrix} \;\;,\]\\

\[
\dot{\widehat{\Omega}}=\begin{bmatrix}
0 &-(\ddot{\theta}\cos\alpha-\dot{\theta}\dot{\alpha}\sin\alpha) & \ddot{\alpha}
\\
(\ddot{\theta}\cos\alpha-\dot{\theta}\dot{\alpha}\sin\alpha) & 0 & 
(\ddot{\theta}\sin\alpha+\dot{\theta}\dot{\alpha}\cos\alpha) \\
-\ddot{\alpha} & -(\ddot{\theta}\sin\alpha+\dot{\theta}\dot{\alpha}\cos\alpha) & 0
\end{bmatrix} \;\;,\]\\
\[
\widehat{\Omega}^2=\begin{bmatrix}-(\dot{\alpha}^2+\dot{\theta}^2\cos^2\alpha) &  -\dot{\alpha}\dot{\theta}\sin{\alpha} & -\dot{\theta}^2\sin{\alpha}\cos{\alpha}\\
-\dot{\alpha}\dot{\theta}\sin{\alpha} & \dot{\alpha}^2 &       \dot{\alpha}\dot{\theta}\cos{\alpha}\\
-\dot{\theta}^2\sin{\alpha}\cos{\alpha} & \dot{\alpha}\dot{\theta}\cos{\alpha} & - (\dot{\alpha}^2+\dot{\theta}^2\sin^2{\alpha})\end{bmatrix}
\]
Let $X$ be the representation of the rotating mass $m$ of the governor in the $\mathbf{b}$ frame. That is
\begin{align*}
X&= \left[\begin{matrix}
  L \\
  0\\
  0
 \end{matrix}\right]
\end{align*}

The fundamental forces due to particle interactions on the Mass $m$ are due to the gravitational interaction in $\mathbf{e}_3$ direction and the resultant reaction force that has the representation
$F^r=[F_1\:\:\:F_2\:\:\:F_3]^T$ in the $\mathbf{b}$-frame due to the connection to the link. Let $f^g$ be the representation of the gravitational force in the $\mathbf{e}$-frame. That is $f^g=[0\:\:\:0\:\:\:-mg]^T$.
Therefore,
\begin{align*}
\mathbf{e}f&=\mathbf{e}f^g+\mathbf{b}F^r\\
&=\mathbf{e}\underbrace{\left[\begin{matrix}
  0 \\
  0\\
  -mg
 \end{matrix}\right]}_{f^g}
+ \mathbf{b}\underbrace{\left[\begin{matrix}
  F_1 \\
  F_2\\
  F_3
 \end{matrix}\right]}_{F}=\mathbf{e} \left(\left[\begin{matrix}
  0 \\
  0\\
  -mg
 \end{matrix}\right]
+ R\left[\begin{matrix}
  F_1 \\
  F_2\\
  F_3
 \end{matrix}\right]\right)=
 \mathbf{e} R\underbrace{\left(R^T\left[\begin{matrix}
  0 \\
  0\\
  -mg
 \end{matrix}\right]
+ \left[\begin{matrix}
  F_1 \\
  F_2\\
  F_3
 \end{matrix}\right]\right)}_{F}=\mathbf{e}RF
 \end{align*}
 and hence we have that
 \[
 F=R^T\left[\begin{matrix}
  0 \\
  0\\
  -mg
 \end{matrix}\right]
+ \left[\begin{matrix}
  F_1 \\
  F_2\\
  F_3
 \end{matrix}\right]=-mg\left[\begin{matrix}
  - \sin{\alpha}\\
   0 \\
\cos{\alpha}
 \end{matrix}\right]
+ \left[\begin{matrix}
  F_1 \\
  F_2\\
  F_3
 \end{matrix}\right].
 \]
 
We know that Newtons' Equations in the moving frame $\mathbf{b}$ are,\\
\begin{align*}
F&=m(\widehat{\Omega}^2 X + 2\widehat{\Omega}\dot{X}+ \dot{\widehat{\Omega}}X)
\end{align*}
Which gives

\begin{align}
-mg\left[\begin{matrix}
  -\sin{\alpha} \\
  0\\
  \cos{\alpha}
 \end{matrix}\right]
+ \left[\begin{matrix}
  F_1 \\
  F_2\\
  F_3
 \end{matrix}\right]&=-mL\begin{bmatrix}(\dot{\alpha}^2+\dot{\theta}^2\cos^2\alpha)\\
 \dot{\alpha}\dot{\theta}\sin{\alpha}\\
 \dot{\theta}^2\sin\alpha\cos\alpha\\
 \end{bmatrix}
 -mL\begin{bmatrix} 0\\
 \dot{\alpha}\dot{\theta}\sin\alpha-\ddot{\theta}\cos\alpha\\ \ddot{\alpha}
 \end{bmatrix}\label{eq:Governor1}
\end{align}

Assume that the mass of the links are negligible and the bottom links are very thin rods. Let the inertia tensor of the upper link expressed in the link fixed body frame $\mathbf{b}$ be $\mathbb{I}=\mathrm{diag}\{\mathbb{I}_z,\mathbb{I}_l,\mathbb{I}_l\}$.

The rigid body equations for the link are given by
\[
\mathbb{I}\dot{\Omega}=\mathbb{I}\Omega \times \Omega+T
\]
where $T$ is the total force moment acting on the link. This is equal to the total moment due to the reaction force $-F^r$ acting at the end of the link and the reaction moments at the pivot point of the link. Let $M^r=[M_1\:\:\:0\:\:\:M_3]^T$ be the representation of the reaction moments in the $\mathbf{b}$-frame. Hence the total force moment is given by
\[
T=X\times (-F^r)+M^r=\begin{bmatrix}M_1\\F_3L\\M_3-F_2L\end{bmatrix}.
\]
Also we have
\[
\mathbb{I}\Omega=
\begin{bmatrix}
 -\mathbb{I}_z\dot{\theta}\sin{\alpha} \\ \mathbb{I}_l\dot{\alpha} \\ \mathbb{I}_l\dot{\theta}\cos\alpha
\end{bmatrix},\:\:\:\:
\mathbb{I}\dot{\Omega}=\begin{bmatrix}
-\mathbb{I}_z (\ddot{\theta}\sin\alpha+\dot{\theta}\dot{\alpha}\cos\alpha)\\ \mathbb{I}_l\ddot{\alpha} \\
\mathbb{I}_l(\ddot{\theta}\cos\alpha-\dot{\theta}\dot{\alpha}\sin\alpha)
\end{bmatrix}
\]
\[
\mathbb{I}\Omega\times \Omega=-\begin{bmatrix}0\\
(\mathbb{I}_l-\mathbb{I}_z)\dot{\theta}^2\cos\alpha\sin\alpha\\
-(\mathbb{I}_l-\mathbb{I}_z)\dot{\theta}\dot{\alpha}\sin\alpha\end{bmatrix}
\]
Thus from the Euler's rigid body equations we have
\begin{align}
 \begin{bmatrix}M_1\\F_3L\\M_3-F_2L\end{bmatrix}&=\begin{bmatrix}
-\mathbb{I}_z (\ddot{\theta}\sin\alpha+\dot{\theta}\dot{\alpha}\cos\alpha)\\ \mathbb{I}_l\ddot{\alpha} +(\mathbb{I}_l-\mathbb{I}_z)\dot{\theta}^2\cos\alpha\sin\alpha\\
\mathbb{I}_l(\ddot{\theta}\cos\alpha-\dot{\theta}\dot{\alpha}\sin\alpha)-(\mathbb{I}_l-\mathbb{I}_z)\dot{\theta}\dot{\alpha}\sin\alpha
\end{bmatrix} \end{align}
Substituting in (\ref{eq:Governor1}) we have

\begin{align*}
-mg\left[\begin{matrix}
  -\sin{\alpha} \\
  0\\
  \cos{\alpha}
 \end{matrix}\right]
+ \left[\begin{matrix}
  F_1 \\
   \frac{1}{L}\left((2\mathbb{I}_l-\mathbb{I}_z)\dot{\theta}\dot{\alpha}\sin\alpha-\mathbb{I}_l\ddot{\theta}\cos\alpha+M_3\right)\\
 \frac{1}{L} \left(\mathbb{I}_l\ddot{\alpha} +(\mathbb{I}_l-\mathbb{I}_z)\dot{\theta}^2\cos\alpha\sin\alpha\right)
 \end{matrix}\right]&=-mL\begin{bmatrix}(\dot{\alpha}^2+\dot{\theta}^2\cos^2\alpha)\\
2 \dot{\alpha}\dot{\theta}\sin\alpha-\ddot{\theta}\cos\alpha\\
 \dot{\theta}^2\sin\alpha\cos\alpha+\ddot{\alpha}\\
 \end{bmatrix}
 \end{align*}

Thus upon simplification we have that the motion of the governor mass is described by the following equations:
\begin{align}
-mL(\dot{\theta}^2\sin^2\alpha+\dot{\alpha}^2)-mg\sin\alpha &=F_1,\\
\frac{1}{L}\left(\mathbb{I}_l\ddot{\alpha} +(\mathbb{I}_l-\mathbb{I}_z)\dot{\theta}^2\cos\alpha\sin\alpha\right)&=F_3,\\
-\mathbb{I}_z(\ddot{\theta}\sin\alpha+\dot{\theta}\dot{\alpha}\cos\alpha )&=M_1,\\
(\mathbb{I}_l+mL^2)\ddot{\theta}\cos\alpha - (2\mathbb{I}_l+2mL^2-\mathbb{I}_z)\dot{\theta}\dot{\alpha}\sin\alpha&=M_3,\\
(\mathbb{I}_l+mL^2)\ddot{\alpha}+(\mathbb{I}_l+mL^2-\mathbb{I}_z)\dot{\theta}^2\cos\alpha\sin\alpha-mgL\cos\alpha&=0.
\end{align}
By assigning another frame $\mathbf{a}'$ to the other link in similar fashion as above as we will find that constraint forces and moments and the equation of motion of that link are also given by the same above equations. Thus if the control moment applied about the $\mathbf{a}_3$ axis on the entire governor is $\tau_u$ then we have that the equations of motion of the governor are given by
\begin{align}
\left((\mathbb{I}_l+mL^2)\cos^2\alpha+\mathbb{I}_z\sin^2\alpha)\right)\ddot{\theta}\cos\alpha - 2(\mathbb{I}_l+mL^2-\mathbb{I}_l)\dot{\theta}\dot{\alpha}\sin\alpha\cos\alpha&=\tau_u,\\
(\mathbb{I}_l+mL^2)\ddot{\alpha}+\left((\mathbb{I}_l+mL^2-\mathbb{I}_z)\dot{\theta}^2\sin\alpha-mgL\right)\cos\alpha&=0.
\end{align}

Consider the case where the governor is rotating at a constant angular rate of $\dot{\theta}(t)\equiv \Omega$. Then the constraint forces and moments reduce to
\begin{align*}
-mL({\Omega}^2\sin^2\alpha+\dot{\alpha}^2)-mg\sin\alpha &=F_1,\\
\frac{1}{L}\left(\mathbb{I}_l\ddot{\alpha} +(\mathbb{I}_l-\mathbb{I}_z){\Omega}^2\cos\alpha\sin\alpha\right)&=F_3,\\
-\mathbb{I}_z({\Omega}\dot{\alpha}\cos\alpha )&=M_1,\\
 - (2\mathbb{I}_l+2mL^2-\mathbb{I}_z){\Omega}\dot{\alpha}\sin\alpha&=M_3,
\end{align*}
and the motion of the spherical masses are described by 
\begin{align}
(\mathbb{I}_l+mL^2)\ddot{\alpha}+\left((\mathbb{I}_l+mL^2-\mathbb{I}_z){\Omega}^2\sin\alpha-mgL\right)\cos\alpha&=0.
\end{align}



Notice that this ODE admits the steady state solutions, ${\alpha}(t)\equiv \pi/2$, ${\alpha}(t)\equiv -\pi/2$ ${\alpha}(t)\equiv \bar{\alpha}$ where
\begin{align}
\sin\bar{\alpha}&=\frac{mgL}{\Omega^2(\mathbb{I}_l+mL^2-\mathbb{I}_z)}.
\end{align}




%%%%%%%%%%%%%%%%%%%
\subsection*{Answer to Exercise \ref{ex:3DOFRobotArm}}
\begin{figure}[ht]
\begin{center}
\includegraphics[width=2.6in]{TwoLink}
\renewcommand{\baselinestretch}{1}\selectfont
\caption{RLC Circuit}
\label{Fig:3DOF_RobotArm}
\end{center}
\end{figure}

Let $\mathbf{e,a,b,c}$ be three orthonormal frames such that $\mathbf{e}$ is earth fixed at the point $O_1$, origin of $\mathbf{a}$ is at $O_1$ and $\mathbf{a}=\mathbf{e}R_a$, origin of $\mathbf{b}$ is at $O_1$ and $\mathbf{b}=\mathbf{a}R_b$, and origin of $\mathbf{c}$ is at $O_2$ (the pivot point of links 1 and 2) and $\mathbf{c}=\mathbf{b}R_c$. Let $G_1,G_2$ be the center of mass of the two linkages. Let $O_1G_1=\mathbf{b}X_{g_1}$ and $O_2G_2=\mathbf{c}X_{g_2}$. Let $O_1O_2=\mathbf{b}X_{o_2}$  and Let $O_2P=\mathbf{c}X_{cp}$.  Let $O_1P=\mathbf{e}x_p$.

$R_a=R_3(\alpha),R_b=R_1(\theta),R_c=R_1{\phi}$, $X_{g_1}=[0\:\:\:L_2\:\:0]^T$, $X_{g_2}=[0\:\:\:L_4\:\:0]^T$, $X_{o_2}=[0\:\:\:L_1\:\:0]^T$ and $X_{cp}=[0\:\:\:L_3\:\:0]^T$.
Then
\[
\mathbf{e}x_p=\mathbf{b}X_{o_2}+\mathbf{c}X_{P}
\]
\[
x_p=R_{L_1}\,X_{o_2}+R_{L_2}\,X_{cp}
\]
where $R_{L_1}=R_aR_b$ and $R_{L_2}=R_aR_bR_c$.
\[
x_{g_1}=R_{L_1}X_{g_1}
\]
\[
x_{g_2}=R_{L_1}\,X_{o_2}+R_{L_2}X_{g_2}
\]
\[
\dot{R}_{L_1}=R_{L_1}\widehat{\Omega}_{L_1}
\]
\[
\dot{R}_{L_2}=R_{L_2}\widehat{\Omega}_{L_2}
\]
\[
\dot{x}_{g_1}=R_{L_1}\widehat{\Omega}_{L_1}X_{g_1}
\]
\[
\dot{x}_{g_2}=R_{L_1}\widehat{\Omega}_{L_1}X_{g_1}+R_{L_2}\widehat{\Omega}_{L_2}X_{g_2}
\]

\[
\dot{x}_{g_1}=\left[\begin{array}{c}
L_2\dot{\theta}\sin(\alpha)\sin(\theta) - L_2\dot{\alpha}\cos(\alpha)\cos(\theta)\\
- L_2\dot{\alpha}\sin(\alpha)\cos(\theta) - L_2\dot{\theta}\cos(\alpha)\sin(\theta)\\
L_2\dot{\theta}cos(\theta)
\end{array} \right]
\]
{\small
\[
\dot{x}_{g_2}=\left[\begin{array}{c}
\dot{\theta}\sin(\alpha)(L_4\sin(\phi + \theta) + L_1\sin(\theta)) - \dot{\alpha}\cos(\alpha)(L_4\cos(\phi + \theta) + L_1\cos(\theta)) + L_4\dot{\phi}\sin(\phi + \theta)\sin(\alpha)\\
- \dot{\alpha}\sin(\alpha)(L_4\cos(\phi + \theta) + L_1\cos(\theta)) - \dot{\theta}\cos(\alpha)(L_4\sin(\phi + \theta) + L_1\sin(\theta)) - L_4\dot{\phi}\sin(\phi + \theta)\cos(\alpha)\\
\dot{\theta}(L_4\cos(\phi + \theta) + L_1\cos(\theta)) + L_4\dot{\phi}\cos(\phi + \theta)
\end{array} \right]
\]
}

\[
\dot{\Omega}_{L_1}=\left[\begin{array}{c}
        \dot{\theta}\\
 \dot{\alpha}\sin(\theta)\\
 \dot{\alpha}\cos(\theta)
\end{array} \right]\:\:\:\:\:
\dot{\Omega}_{L_2}=\left[\begin{array}{c}
        \dot{\phi} + \dot{\theta}\\
 \dot{\alpha}\sin(\phi + \theta)\\
 \dot{\alpha}\cos(\phi + \theta)
\end{array} \right]
\]

\[
\mathrm{KE}=\frac{1}{2}M_1||\dot{x}_{g_1}||^2+\frac{1}{2}M_2||\dot{x}_{g_1}||^2+\frac{1}{2}I_{L_1}\Omega_{L_1}\cdot\Omega_{L_1}+\frac{1}{2}I_{L_2}\Omega_{L_2}\cdot\Omega_{L_2}
\]
\[
\mathrm{PE}=\left((L_1+L_2)\sin(\theta)+L_4\sin(\phi + \theta)\right)g
\]
Generalized external force
\[
f^e=\tau_{\alpha}\,d\alpha+\tau_{\theta}\,d\theta+\tau_{\phi}\,d\phi.
\]
%%%%%%%%%%%%%%%%%%%%%%%%%%%%%%%%%%%%%%%%%%%%%%

\begin{thebibliography}{lll}
\bibitem{MM} Christoph Schiller, \emph{Motion Mountain: The Adventures of Physics, Vol-1, - Fall, Flow, and Heat}, E-Book, www.motionmountain.net

\bibitem{HaimBaruh} Haim Baruh,
\emph{Analytical Dynamics} McGraw-Hill, 1999.

\bibitem{Abraham} R. Abraham and J. E. Marsden,
\emph{Foundations of Mechanics, Second Ed.} Westview, 1978.

\bibitem{Arnold} V. I. Arnold,
\emph{Mathematical Methods of Classical Mechanics, Second Ed.},
 Springer-Verlag, New York 1989.

\bibitem{Marsden} J. E. Marsden and T. S. Ratiu,
\emph{Introduction to Mechanics and Symmetry, Second Ed.}
Springer-Verlag, New York 1999.

\bibitem{BulloLewis} F. Bullo and A. D. Lewis,
\emph{Geometric Control of Mechanical Systems: Modeling, Analysis, and Design for Simple Mechanical
Control Systems,} Springer-Verlag, New York 2004.

\bibitem{Sussman} G. J. Sussman and J. Wisdom,
\emph{Introduction to Classical Mechanics,} (available online at http://mitpress.mit.edu/SICM/), MIT Press, 2001.

\bibitem{Greenwood} D. T. Greenwood,
\emph{Advanced Dynamics,}  Cambridge University Press, 2003.

\bibitem{Frankel} T. Frankel,
\emph{The Geometry of Physics: An Introduction,}  Cambridge University Press, $3^{rd}$ Edition, 2012.

\end{thebibliography}



%%%%%%%%%%%%

\printindex

\end{document}
